Based on the specification of the platform at hand, an equivalent thermal
\abbr{RC} circuit is constructed \cite{skadron2004}. The circuit comprises $\nn$
thermal nodes, and its structure depends on the intended level of granularity,
which impacts the resulting accuracy. For clarity, we assume that each
processing element is mapped onto one corresponding node, and the thermal
package is represented as a set of additional nodes.

The thermal dynamics of the system are modeled using the following system of
differential-algebraic equations \cite{ukhov2012, ukhov2014}:
\begin{subnumcases}{\elab{thermal-system}}
  \mC \frac{d\vs(\t)}{d\t} + \mG \vs(\t) = \mM \vp(\t) \elab{thermal-system-ode} \\
  \vq(\t) = \mM^T \vs(\t) + \vq_\ambient
\end{subnumcases}
The coefficients $\mC \in \real^{\nn \times \nn}$ and $\mG \in \real^{\nn \times
\nn}$ are a diagonal matrix of thermal capacitance and a symmetric,
positive-definite matrix of thermal conductance, respectively. The vectors
$\vp(\t) \in \real^\np$, $\vq(\t) \in \real^\np$, and $\vs(\t) \in \real^\nn$
are the power consumption (given in \eref{power}), temperature, and internal
state of the system at time $\t$, respectively; all three are random due to the
dependency on $\vu$. The vector $\vq_\ambient \in \real^\np$ contains the
ambient temperature. The matrix $\mM \in \real^{\nn \times \np}$ is a mapping
that distributes the power dissipation of the processing elements across the
thermal nodes; without loss of generality, $\mM$ is a rectangular diagonal
matrix whose diagonal elements are equal to unity.

As noted with respect to power, given a time span and a sampling interval,
\eref{thermal-system} can be utilized to compute the temperature profile $\mQ
\in \real^{\np \times \ns}$ of the system, which might serve as an $(\np \times
\ns)$-dimensional quantity of interest $\g$.

\begin{remark}
We do not cover dynamic steady-state temperature analysis \cite{ukhov2012} in
this paper; however, an extension of the proposed techniques to this scenarios
is straightforward.
\end{remark}

Given a temperature profile $\mQ = (\q_{ij})$, the maximal temperature of the
system can be estimated as follows:
\[
  \text{Maximal temperature} = \max_{i, j} \q_{ij},
\]
which is yet another potential quantity of interest $\g$.
