The power consumption of the tasks is assumed to be fixed, and it is given by a
vector $\vp = (\p_i)_{i = 1}^{\nt}$ wherein $\p_i$ is the power of the $i$th
task. Note that the boundaries of a task have not been specified, and one can
perform modeling at the level of granularity that makes the most sense for a
particular problem.

Let the vector $\vp(\t) = (\p_i(\t))_{i = 1}^{\np}$ capture the power
dissipation of the system at time $\t$. This vector should not be confused with
the vector $\vp$ introduced in the previous paragraph. The two are related as
follows:
\begin{equation} \elab{power}
  \p_i(\t) = \sum_{j = 1}^{\nt} \delta_{i\,m_j} \: \one_{[\b_j, \b_j +
    \d_j)}(\t) \: \p_j,
\end{equation}
for $i = 1, \dots, \np$, where $\delta_{ij}$ is the Kronecker delta and
$\one_A(x)$ is the indicator function of a set $A$. In words, the above equation
yields the power consumption of the task that is running on the $i$th processing
element at time $\t$, if any. Given a set of time instances, we can now use
\eref{power} to construct the corresponding power profile of the system.

Having obtained time and power, it is straightforward to calculate energy. One
can integrate \eref{power}; however, an easier approach is use $\vp$ and $\vd$
directly:
\[
  \text{Total energy} = \vp^T \vd = \sum_{j = 1}^\nt \p_j \d_j.
\]
