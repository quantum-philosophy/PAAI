Let the vector $\vp(\t) = (p_i(\t))_{i = 1}^{\np}$ capture the power dissipation
of the system at time $\t$. This vector should not be confused with the vector
$\vp = (p_i)_{i = 1}^{\nt}$ introduced in \sref{problem-formulation}. The two
are related as follows:
\begin{equation} \elab{power}
  p_i(\t) = \sum_{j = 1}^{\nt} \delta_{i\,m_j} \: \one_{[\b_j, \b_j + \d_j)}(\t) \: p_j,
\end{equation}
for $i = 1, 2, \dots, \np$, where $\delta_{ij}$ is the Kronecker delta and
$\one_A(x)$ is the indicator function of a set $A$. In words, the above equation
yields the power consumption of the task that is running on the $i$th processing
element at time $\t$, if any.

Given a set of time instances, \eref{power} readily yields the corresponding
power profile of the system.
