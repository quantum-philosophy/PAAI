In this section, we focus on the dynamic power dissipation of the platform $\platform$ running the application $\application$ with respect to $\mM$ (see \sref{application-model}). Denote $\vPP(\t, \o) \in \real^\Npe$ the vector of the power consumption of the processing elements at time $\t$. The $i$th element of $\vPP(\t, \o)$, $\PP_i(\t, \o)$, is given as
\begin{equation} \elabel{processor-power}
  \PP_i(\t, \o) = \sum_{j \in \index^\pi_i} \PT_j(\o) \; \I_{\event_j(\t)}(\o)
\end{equation}
where $\PT_j(\o)$ is the power mode (the power consumption) of the $j$th task, and $\I_\event(\o)$ denotes the indicator function of an event $A \in \F$ (see \sref{uncertainty-model}), which is equal to one if $\o \in \event$ and to zero, otherwise. The time-dependent events $\event_j(\t)$ in \eref{processor-power} are defined as
\[
  \event_j(\t) = \{ \o \in \O: \S_j(\o) \leq \t < \E_j(\o) \} \in \F
\]
Note that $\event_j(\t)$ are disjoint events with respect to $\index^\pi_i$, \ie, $\event_k(\t) \cap \event_k(\t) = \emptyset$, $j, k \in \index^\pi_i$, $j \neq k$. For brevity, we rewrite \eref{processor-power} in the matrix form:
\begin{equation} \elabel{platform-power}
  \vPP(\t, \o) = \mM \; \oDiag{\vPT(\o)} \: \I_{\m{\event}(\t)}(\o)
\end{equation}
where $\oDiag: \real^n \to \real^{n \times n}$ is the operator that constructs a diagonal matrix from a vector, and $\I_{\m{\event}(\t)}(\o)$ is a vector composed of $\I_{\event_i(t)}(\o)$, $i \in \{ 1, \dots, \Nts \}$.

Now, we perform a spectral expansion of \eref{platform-power}. Due to the presence of indicator functions, \eref{platform-power} is a highly discontinuous function. Therefore, the convergence rate of polynomial-based expansions can be severely deteriorated. To circumvent the difficulty, we employ the Wiener-Haar decomposition \cite{maitre2004} where the basis functions are the Haar wavelets (see \aref{spectral-expansions}). In this case,
\[
  \vPP(\t, \o) = \sum_{i = 1}^\infty \wcoeff{\vPP}(\t) \wfun_i(\vZ(\o))
\]
where $\wfun_i(\vZ(\o))$ are the basis funcitons, which are multivariate wavelets. Consequently, at each moment of time, we have an analytical expression for the dynamic power dissipation of the system, which can be employed to analyze the system from different perspectives.
