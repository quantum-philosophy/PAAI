The work is based on work in \cite{klimke2006} and \cite{ma2009}.

The approach belongs to the class of stochastic collocation techniques
\cite{xiu2010}. The major distinctive feature of stochastic collocation is the
usage of interpolation as a means of uncertainty quantification, which should be
contrasted with other techniques such as polynomial chaos expansions relying on
regression. The application of popular polynomial expansions is limited in this
case due to the non-smoothness of the response surface.

The remainder of the paper is organized as follows.
