The distribution of the uncertain parameters $\vu = (\u_i)$ is assumed to be
given as a set of marginal distribution functions
\[
  \{ \distribution_{\u_i}: i = 1, \dots, \nu \}
\]
and a copula \cite{nelsen2006}
\begin{align*}
  \copula_{\vu}(u_1, \dots, u_\nu) &= \probabilityMeasure(\distribution_{\u_1}(\u_1) \leq u_1, \dots, \distribution_{\u_\nu}(\u_\nu) \leq u_\nu).
\end{align*}
The copula is a uniform distribution function on $[0, 1]^\nu$ that captures the
dependencies between $(\u_i)$.

Each task $\task_i \in \tasks$ is ascribed a \rv\ $\e_i$ representing its
execution time. Let $\ve = (\e_i)_{i=1}^{\nt}$. The random vector $\ve$ is
assumed to be measurable with respect to $\sigma(\vu)$, the $\sigma$-algebra
generated by $\vu$ \cite{durrett2010}. Loosely speaking, this means that there
exists a map $f: \real^\nu \to \real^\nt$ such that $\ve = f(\vu)$; $\ve$ is
known whenever $\vu$ is known. A trivial example is a bijective mapping between
$\vu$ and $\ve$: $\u_i = \e_i$, for $i = 1, \dots, \nt$.
