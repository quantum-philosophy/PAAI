Based on the specification of the platform at hand, an equivalent thermal
\abbr{RC} circuit is constructed \cite{skadron2004}. The circuit comprises $\nn$
thermal nodes, and its structure depends on the intended level of granularity,
which impacts the resulting accuracy. For clarity, we assume that each
processing element is mapped onto one corresponding node, and the thermal
package is represented as a set of additional nodes.

The thermal dynamics of the system are modeled using the following system of
differential-algebraic equations \cite{ukhov2012, ukhov2014}:
\begin{subnumcases}{}
  \mC \frac{\d\vs(\t, \vu)}{\d\t} + \mG \vs(\t, \vu) = \mM \vp(\t, \vu) \\
  \vq(\t, \vu) = \mM^T \vs(\t, \vu) + \vq_\ambient
\end{subnumcases}
$\mC \in \real^{\nn \times \nn}$ and $\mG \in \real^{\nn \times \nn}$ are a
diagonal matrix of thermal capacitance and a symmetric, positive-definite matrix
of thermal conductance, respectively. $\vp \in \real^\np$, $\vq \in \real^\np$,
and $\vs \in \real^\nn$ are the power, temperature, and state vectors of the
system, respectively. $\vq_\ambient \in \real^\np$ is a vector of the ambient
temperature. $\mM \in \real^{\nn \times \np}$ is a matrix that distributes the
power dissipation of the processing elements across the thermal nodes; without
loss of generality, $\mM$ is a rectangular diagonal matrix whose diagonal
elements are equal to unity.
