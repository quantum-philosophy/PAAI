Based on the spacification of the platform at hand, an equivalent thermal
\abbr{RC} circuit is constructed \cite{skadron2004}. The circuit comprises
$\nn$ thermal nodes, and its structure depends on the intended level of
granularity, which impacts the resulting accuracy. For clarity, we assume that
each processing element is mapped onto one corresponding node, and the thermal
package is represented as a set of additional nodes.

The thermal dynamics of the system are modeled using the following system of
differential-algebraic equations \cite{ukhov2012, ukhov2014}:
\begin{subnumcases}{}
  \C \frac{\d\S(\t, \U)}{\d\t} + \G \S(\t, \U) = \M \P(\t, \U) \\
  \Q(\t, \U) = \M^T \S(\t, \U) + \Q_\ambient
\end{subnumcases}
$\C \in \real^{\nn \times \nn}$ and $\G \in \real^{\nn \times \nn}$ are a
diagonal matrix of thermal capacitance and a symmetric, positive-definite
matrix of thermal conductance, respectively. $\P \in \real^\np$, $\Q \in
\real^\np$, and $\S \in \real^\nn$ are the power, temperature, and state
vectors of the system, respectively. $\Q_\ambient \in \real^\np$ is a vector of
the ambient temperature. $\M \in \real^{\nn \times \np}$ is a matrix that
distributes the power dissipation of the processing elements across the thermal
nodes; without loss of generality, $\M$ is a rectangular diagonal matrix whose
diagonal elements are equal to unity.
