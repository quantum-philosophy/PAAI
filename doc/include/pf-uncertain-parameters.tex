In this paper, we focus on one of the most realistic scenarios where (a) the uncertain parameters $\vU(\o)$ are correlated; (b) the joint probability function of $\vU(\o)$ is not known---as it is typically infeasible to obtain---and only the marginal distribution functions and a matrix of correlations (see \sref{uncertainty-model}) are available. Since the mathematical framework employed in this paper operates on mutually independent \rvs, $\vU(\o)$ are to be expressed in terms of independent \rvs, which we denote by $\vZ(\o)$, to satisfy the the requirement. The procedure consists of the following steps.\footnote{However, if the joint distribution function is available, Rosenblatt's transformation \cite{rosenblatt1952} is a preferable choice.}

\step{Step 1.} The dependent parameters $\vU(\o)$ are transformed into dependent standard uniform \rvs\ $\vU'(\o)$ by applying the corresponding marginal distribution functions:
\begin{equation} \elabel{dependent-uniform}
  \vU'(\o) = \CDF_{\U_i}(\vU(\o)), \qquad \U'_i(\o) \sim \uniform(0, 1)
\end{equation}
Hereafter, $\CDF_{\U_i}(\cdot)$ being applied to a vector denotes an element-wise operation with the corresponding $\CDF_{\U_i}(\u)$, $\forall i$. In this work, all the distribution functions involved in transformations are assumed to be strictly increasing functions; therefore, such transformations as above are unique.

\step{Step 2.} $\vU'(\o)$ are then transformed into dependent standard normal \rvs\ $\vU''(\o)$ by applying the inverse probability transformation:
\begin{equation} \elabel{dependent-normal}
  \vU''(\o) = \CDF^{-1}_\normal(\vU'(\o)), \qquad \U''_i(\o) \sim \normal(0, 1)
\end{equation}
where $\CDF_\normal$ denotes the (univariate) standard normal distribution function (an element-wise operation). Since each marginal distributions is normal, the joint distribution function is multivariate normal, \ie, $\vU''(\o) \sim \normal(\v{0}_\Nup, \mLCorr_{\vU''})$ where $\v{0}_n$ denotes the $n$-dimensional zero vector, and $\mLCorr_{\vU''}$ is a Pearson correlation matrix of $\vU''(\o)$. $\mLCorr_{\vU''}$ is to be determined such that the chain of the inverse transformations, opposite to the ones in Step~1--2, leads to the given correlation matrix $\mCorr_\vU$. The following step serves this purpose.

\step{Step 3.} When $\mCorr_\vU \equiv \mNCorr_\vU$, we only need to ensure that the \rvs\ $\vU''(\o)$ have the same rank correlations as $\vU(\o)$, \ie, $\mNCorr_{\vU''} = \mNCorr_{\vU}$, since such measures of correlations are resistant to monotonic transformations, should they be linear or nonlinear. In our case, the relations between the Pearson correlation coefficient, Spearman's rho, and Kendall's tau have closed forms given in \aref{correlation-measures}. When $\mCorr_\vU \equiv \mLCorr_\vU$, extra affords are required since nonlinear transformations do not preserve linear measures of correlations, which Pearson's correlation coefficient is. The problem is solved via the Nataf transformation \cite{li2008}.

\step{Step 4.} The $\Nup$ dependent \rvs\ $\vU''(\o)$ are transformed into $\Nrv$, $\Nrv \leq \Nup$, independent standard normal \rvs\ $\vU'''(\o)$ using the principal component analysis (PCA) as
\begin{equation} \elabel{independent-normal}
  \vU'''(\o) = \oPCA{\vU''(\o)}
\end{equation}
Refer to \aref{principal-component-analysis} in the appendix for further datails.

\step{Step 5.} The independent \rvs\ $\vU'''(\o)$ are transformed into independent \rvs\ $\vZ(\o)$ that have preferable distribution functions from some perspective, \eg, in order to ease the subsequent computations. The transformation repeats Step~1--2 as follows:
\begin{equation} \elabel{independent-desired}
  \vZ(\o) = \CDF^{-1}_{\desired_i}(\CDF_\normal(\vU'''(\o)))
\end{equation}
where $\CDF_{\desired_i}$, $\forall i$, are the desired distribution functions. Without loss of generality, $\vZ(\o)$ are assumened to have zero means and unit variances. The step is optional; if it is bypassed, $\vZ(\o) \equiv \vU'''(\o)$.

The overall transformation described in this section consists of \eref{dependent-uniform} through \eref{independent-desired} and is denoted by
\[
  \vZ(\o) = \oT{\vU(\o)} = \CDF^{-1}_{\desired_i}(\CDF_\normal(\oPCA{\CDF_\normal^{-1}(\CDF_{\U_i}(\vU(\o)))}))
\]
The inverse of $\oT$ is defined as
\begin{equation} \elabel{parametrization}
  \vU(\o) = \oInvT{\vZ(\o)} = \CDF_{\U_i}^{-1}(\CDF_\normal(\oInvPCA{\CDF^{-1}_\normal(\CDF_{\desired_i}(\vZ(\o)))}))
\end{equation}

To sum up, the $\Nup$ uncertain parameters $\vU(\o)$ are now parametrized by $\Nrv$ independent \rvs\ $\vZ(\o)$. Note, however, if the marginal distribution functions of $\vU(\o)$ are normal, only the fourth step is required. Also, in the following, for convenience, we shall write $\vZ(\o)$ instead of $\vU(\o)$.
