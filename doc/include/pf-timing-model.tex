In this paper, we utilize spectral methods in order to quantify the uncertainty by expanding probabilistic quantities into infinite sequences of orthogonal functions in terms of the independent \rvs\ $\vZ(\o)$ obtained in \sref{uncertain-parameters}; refer to \aref{spectral-expansions} in the appendix for an introduction to such methods. Since our focus is on the variations due to the uncertain workload of the platform $\platform$, we begin with the stochastic timing analysis of the application $\application$.

First of all, we perform spectral expansions of the \rvs\ $\D_i(\o)$, $i \in \{ 1, \dots, \Nts \}$, which represent the execution times of the tasks $\tasks$ (see \sref{application-model}), as follows:
\[
  \D_i(\o) = \sum_{j = 1}^\infty \timeCoeff{\D}_{ij} \timeBasis_j(\vZ(\o))
\]
where $\timeCoeff{\D}_{ij}$ are the coefficients of the expansion, and $\timeBasis_j(\vZ(\o))$ are the basis functions. For the timing analysis, we utilize the polynomial chaos (PC), in which $\timeBasis_j(\vZ(\o))$ are orthogonal polynomials. The particular choice of $\timeBasis_j(\vZ(\o))$ and the computation of the corresponding coefficients are further discussed in the appendix, \aref{spectral-expansions}. Now, let $\S_j(\o)$ and $\E_j(\o) = \S_j(\o) + \D_j(\o)$ be the \rvs\ that model the start and end times of $\task_i$. Without loss of generality, assume the given task graph of $\application$ has a single root and a single sink. Starting from the root and taking into consideration the mapping $\mM$, we gradually compute spectral expansions of $\S_i(\o)$, $\forall i$:
\[
  \S_i(\o) = \sum_{j = 1}^\infty \timeCoeff{\S}_{ij} \timeBasis_j(\vZ(\o))
\]
The procedure can be views as a delay propagation though an electrical circuit, which involves two basic operations: the add and max operations. The former is trivial as one only needs to sum up the corresponding coefficients of two expansions. For instance, the expansion of $\E_i(\o)$ is
\[
  \E_i(\o) = \sum_{j = 1}^\infty \left( \timeCoeff{\S}_{ij} + \timeCoeff{\D}_{ij} \right) \timeBasis_j(\vZ(\o))
\]
which is also the expansion of the start times of $\{ \task_k: k \in \iTask_{>i}, \; \size{\iTask_{k<}} = 1 \}$ (see \sref{application-model}) where $\size{\cdot}$ denotes the number of elements in the set. The max operation is required for $\{ \S_i(\o): \size{\iTask_{>i}} > 1 \}$ and is performed via an additional spectral expansion based on the expansions of $\{ \E_j(\o): j \in \iTask_{>i} \}$. The spectral expansion of the end time $\E_\text{sink}(\o)$ of the sink node yields an analytical expression of the total execution time of the application, which can be further analyzed in a straightforward manner.
