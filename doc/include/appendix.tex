\renewcommand{\thesection}{S\arabic{section}}
\renewcommand{\thetable}{S\arabic{table}}
\renewcommand{\thefigure}{S\arabic{figure}}
\setcounter{table}{0}
\setcounter{figure}{0}

\section{Correlation Measures} \alabel{correlation-measures}
Recall that $\vU''(\o)$ has a multivariate normal distribution and assume that the given correlation matrix of $\vU(\o)$ is defined using a rank correlation coefficient, \ie, $\mCorr_\vU \equiv \mNCorr_\vU$. If $\mNCorr_\vU$ is defined in terms of Spearman's rho, then the corresponding Pearson correlation matrix is
\[
  \mLCorr_{\vU''} = 2 \sin \left( \frac{\pi}{6} \mNCorr_{\vU} \right)
\]
If $\mNCorr_\vU$ is defined in terms of Kendall's tau, then
\[
  \mLCorr_{\vU''} = \sin \left( \frac{\pi}{2} \mNCorr_{\vU} \right)
\]
Both transformations should be understood as element-wise operations.

\section{Principal Component Analysis} \alabel{principal-component-analysis}
Given the expectation vector $\vExp_\v{X}$ and covariance matrix $\mCov_\v{X}$ of a vector of correlated \rvs\ $\v{X}(\o) \in \real^n$, the PCA performs a linear transformation of $\v{X}(\o)$ into a vector of centered and normalized \rvs\ $\v{Y}(\o) \in \real^m$ such that the linear correlations are eliminated, \ie, $\vExp_\v{Y} = \v{0}_m$ and $\mCov_\v{Y} = \m{I}_{m \times m}$. The reduction is based on the specified parameter $\pPCA$, which defines the portion of the variance of $\v{X}$ that should be preserved. Denote the PCA operator by $\oPCA: \real^n \to \real^m$.

\section{Spectral Expansions} \alabel{spectral-expansions}
In order to perform the TPT analysis, we rely on spectral methods, namely, the polynomial chaos (PC), \sref{timing-model}, and the Wiener-Haar decomposition (WH), \sref{power-model} and \sref{thermal-model}. In this case, an expansion of an $n$-dimensional stochastic quantity $\v{X}(\t, \o)$, which is parametrized by $\t$, has the following form:
\[
  \v{X}(\t, \o) = \sum_{i = 1}^\infty \generalCoeff{\v{X}}_i(\t) \generalBasis_i(\vZ(\o))
\]
where $\generalCoeff{\v{X}}_i(\t)$ are the coefficients of the expansion, $\generalBasis_i(\vZ(\o))$ are the basis functions, and the equality is in the $\L{2}$ sense. Infinite expansions are to be truncated to become feasible for practical computations. Define the spectral expansion operator as follows:
\[
  \v{X}(\t, \o) \approx \oSE{p}{\v{X}}(\t, \o) = \sum\nolimits_{i \in \iExpansion_p} \generalCoeff{\v{X}}_i(\t) \generalBasis_i(\vZ(\o))
\]
where $\iExpansion_p$ is an index set constructed with respect to some parameter $p$, which defines a truncation strategy.

\subsection{Polynomial Chaos Expansion}
Since the probability distribution of the stochastic quantity, which is to be expanded, is not known \apriori, it is not possible, in general, to construct the optimal polynomial basis beforehand \cite{maitre2010}. However, to start with, one should be guided by the distribution functions of the underlying \rvs, \ie, $\CDF_{\desired_i}$ in our case (see \sref{uncertain-parameters}). Since $\CDF_{\desired_i}$ are selected irrespectively of the input distributions $\CDF_{\U_i}$, it is preferable to have the ones among the standard distributions, namely, normal, $\beta$, $\gamma$,  uniform, Poisson, binomial, negative binomial, and hypergeometric. In this case, the rule-of-thumb polynomials are known and given by the Askey scheme of orthogonal polynomials \cite{xiu2002}. Otherwise, custom orthogonal polynomials are to be constructed via, \eg, the Gram-Schmidt orthogonalization process \cite{witteveen2006}.

In this paper, $\iExpansion_p \equiv \{ i: \oTO{\generalBasis_i(\vZ(\o))} \leq p \}$ where $\oTO$ returns the total order of the given multivariate polynomial. $p \equiv \Neo$ is called the expansion order.

\subsection{Wiener-Haar Decomposition}
The basis functions $\generalBasis_i(\vZ(\o))$ are the Haar wavelet functions.

\section{Numerical Integration}
We utilize the Clenshaw-Curtis quadrature rule, wherein the integrand is first expanded into the Chebychev polynomials, and then the polynomials are integrated exactly. As any other quadrature rule, the technique boils down to computation of a weighted sum of the integrand values evaluated at a certain set of prescribed nodes. The rule is nested and, therefore, efficient when it comes to construction of sparse grid for multidimensional integration.
