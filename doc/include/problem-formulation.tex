Consider a multiprocessor system composed of two major components: a platform
and an application. The platform consists of a set of heterogeneous processing
elements $\procs = \{ \proc_i: i = 1, \dots, \np \}$ and is equipped with a
thermal package. The application is given as a directed acyclic graph $\app =
(\tasks, \deps)$ where $\tasks = \{ \task_i: i = 1, \dots, \nt \}$ is a set of
vertices representing tasks, and $\deps \subset \tasks \times \tasks$ is a set
of edges representing data dependencies between the tasks.

The mapping of the application onto the platform is assumed to be fixed and
given by a function $\mapping: \tasks \to \procs$. The power consumption of the
tasks is assumed to be fixed and given by a function $\power: \tasks \to
\real$. For example, for $\task \in \tasks$, $\mapping(\task)$ is the
processing element that the task is executed on, and $\power(\task)$ is the
amount of power that the tasks is estimated to consume. Note that the
boundaries of a task have not been specified, and one can perform modeling at
the level of granularity that makes the most sense for a particular problem.

The system depends on a set of parameters that are uncertain at the design
stage. We model such parameters as a random vector $\X = (\X_i)_{i = 1}^\nx$.
The focus of this work is on those parameters that affect the timing
characteristics of the system, namely, the start and finish time of each of the
tasks. This variability inevitably implies that the corresponding power and
temperature profiles of the system are affected as well and, thus, are
uncertain for the designer.

We pursue the following major objectives:
\begin{itemize}

  \item {\bfseries Objective~1.} Develop a framework for transient temperature
  analysis of the system $(\procs, \app)$ that takes into consideration the
  uncertainty due the parameters $\X$ affecting the timing characteristics of
  the tasks $\tasks$.

\end{itemize}
