One of the central algorithms in the field of high-dimensional integration and
interpolation is the Smolyak algorithm. Intuitively speaking, the algorithm
takes a number of small tensor-product structures and composes them in such a
way that the resulting grid has a drastically reduced number of evaluation or
collocation nodes, compared to the full tensor-product construction, while the
approximating power of such a sparse grid stays the same for a class of
functions that one is typically interested in integrating or interpolating
\cite{klimke2006}.

The Smolyak interpolant for $\f$ is as follows:
\begin{equation} \elab{smolyak}
  \smolyak{l}(\f) := \sum_{l - \nin + 1 \leq |\vi| \leq l} (-1)^{l - |\vi|} \, {\nin - 1 \choose l - |\vi|} \, \tensor{\vi}(\f)
\end{equation}
where $l \in \natural$ is the level of interpolation, and $|\vi| := i_1 + \dots
+ i_\nin$. We see that the algorithm is indeed just a peculiar composition of
cherry-picked tensor products. However, the formula has an implication of
paramount importance: the quantity of interest needs to be exercised only at the
nodes of the sparse grid underpinning \eref{smolyak}:
\begin{equation} \elab{smolyak-grid}
  \Y^l = \bigcup_{l - \nin + 1 \leq |\vi| \leq l} \X^\vi.
\end{equation}
The cardinality of the above set does not have a general closed-form formula;
however, it can be several orders of magnitude smaller than the one of the full
tensor product given in \eref{tensor-cardinality}, depending on the
dimensionality of the problem at hand.

A better intuition about the properties of Smolyak's construction can be
obtained by rewriting \eref{smolyak} in an incremental form. To this end, let
$\Delta\tensor{0}(\f) := \tensor{0}(\f)$,
\begin{equation} \elab{tensor-delta-1d}
  \Delta\tensor{i}(\f) := (\tensor{i} - \tensor{i - 1})(\f), \qquad \text{for $i > 0$},
\end{equation}
and
\begin{equation} \elab{tensor-delta}
  \Delta\tensor{\vi}(\f) := (\Delta\tensor{i_1} \otimes \cdots \otimes \Delta\tensor{i_\nin})(\f).
\end{equation}
Then, \eref{smolyak} is identical to
\begin{equation} \elab{smolyak-incremental}
  \smolyak{l}(\f) = \sum_{|\vi| \leq l} \Delta\tensor{\vi}(\f) = \smolyak{l - 1}(\f) + \sum_{|\vi| = l} \Delta\tensor{\vi}(\f).
\end{equation}
It can be seen that a Smolyak interpolant can be refined without the need of
starting from scratch: the work done to attain one accuracy level can be
entirely recycled to go one level up.

The sparsity and efficient refinement of Smolyak's approach---shown in
\eref{smolyak-grid} and \eref{smolyak-incremental}, respectively---are
remarkable properties \perse\ (taking into consideration the resulting
accuracy), but they can be taken even further. Let $\Delta\X^0 := \X^0$,
\[
  \Delta\X^i := \X^i \setminus \X^{i - 1}, \quad \text{for $i > 0$,}
\]
and
\[
  \Delta\X^\vi := \Delta\X^{i_1} \times \cdots \times \Delta\X^{i_\nin}.
\]
Then, \eref{smolyak-grid} can be rewritten as
\begin{equation} \elab{smolyak-grid-incremental}
  \Y^l = \bigcup_{|\vi| \leq l} \Delta\X^i = \Y^{l - 1} \cup \bigcup_{|\vi| = l} \Delta\X^i,
\end{equation}
which is analogous to \eref{smolyak-incremental}. Now, we note that it is
computationally beneficial to have $\X^{i - 1}$ be partially or entirely
included in $\X^i$ since, in that case, $\Y^l \setminus \Y^{l - 1}$ in
\eref{smolyak-grid-incremental} gets smaller. In words, the values of $\f$
evaluated on lower levels can be reused on higher levels if the interpolating
grid grows without destroying/abandoning its previous structure. With this in
mind, the rules used for generating successive sets of points $\X^{i_k}$ in each
dimension $k$ can be chosen to be nested, that is, in such a way that the rule
of level $i_k$ contains all the nodes of the rule of level $i_k - 1$, for $k =
2, 3, \dots, \nin$.

Lastly, we would like to impose one additional property in order to make
\eref{smolyak-incremental} well suited for practical implementations. Namely,
interpolants of higher levels are required to represent exactly interpolants
of lower levels:
\begin{equation} \elab{tensor-exactness}
  \tensor{i - 1}(\f) = \tensor{i}(\tensor{i - 1}(\f)).
\end{equation}
This can be achieved by an appropriate choice of collocation nodes and basis
functions, which will be discussed later. If the above equation holds, using
\eref{tensor-1d} and \eref{tensor-delta-1d},
\[
  \Delta\tensor{i}(\f) = \sum_{j \in \index(i)} \left(\f(\x^i_j) - \tensor{i - 1}(\f)(\x^i_j)\right) \e^i_j.
\]
We note that the above sum is over $\X^i = \{ \x^i_j: j \in \index(i) \}$, and
the difference in the sum, which we denote by
\begin{equation} \elab{surplus}
  \surplus(\x^i_j) = \f(\x^i_j) - \tensor{i - 1}(\f)(\x^i_j),
\end{equation}
is zero whenever $\x^i_j \in \X^{i - 1}$. Since $\X^{i - 1} \subset \X^i$, we
actually need to sum over the nodes in $\Delta\X^i$, which we shall write as
\begin{equation} \elab{tensor-surplus-1d}
  \Delta\tensor{i}(\f) = \sum_{\x^i_j \in \Delta\X^i} \surplus(\x^i_j) \, \e^i_j
\end{equation}
where $\e^i_j$ naturally corresponds to $\x^i_j$. The delta $\surplus(\x^i_j)$
in \eref{surplus} is referred to as a hierarchical surplus. When going from one
level of interpolation to the next one, the surplus is nothing more than the
difference between the actual value of $\f$ at a new collocation point and its
approximation by the interpolant constructed so far. Multidimensional surpluses
of $\f$ are constructed using \eref{tensor-surplus-1d} and
\eref{tensor-delta} denoted by $\Delta\f(\vx^\vi_\vj)$.

To summarize, assuming that collocation points and basis functions have been
carefully chosen according to the criteria mentioned earlier, we have obtained
an efficient algorithm for high-dimensional interpolation. The main equations
are \eref{smolyak-incremental} and \eref{tensor-surplus-1d}, which enable to
undertake the approximation in an incremental or hierarchical manner.
