The life cycle of interpolation has roughly two phases: construction and usage.
The construction phase involves sampling the target function $\f$ at collocation
points and produces a set of artifacts needed for the actual interpolation at
the usage phase. Regarding the aforementioned artifacts, it can be seen in
\eref{tensor-surplus-1d} that an approximant is entirely characterized by a set
of indexing pairs $\{ (\vi, \vj) \}$ and the corresponding surpluses $\{
\surplus(\vx^\vi_\vj) \}$. Recall that the multi-index $\vi$ captures the levels
of interpolation with respect to each dimension, and $\vj$ captures the
corresponding orders (see \sref{tensor-product}). Each pair $(\vi, \vj)$
unambiguously maps to a collocation point and a basis function, which, together
with the corresponding surplus, allow one to evaluate the interpolant at any
point of interest.

\begin{algorithm}
  \caption{\emph{Construct} an interpolant for a function.}
  \alab{construct}
  \begin{algorithmic}[1]
    \vspace{0.4em}

    \Require{target} \Comment{A function to approximate}
    \Ensure{surrogate} \Comment{The approximating object}

    \vspace{0.4em}

    \Let{level}{0}
    \Let{indices}{ComputeIndices(level)}
    \Let{surrogate}{InitializeEmpty()}

    \Loop
      \Let{nodes}{ComputeNodes(indices)}
      \Let{values}{Execute(target, nodes)}
      \Let{interpolated\_values}{Evaluate(surrogate, nodes)}
      \Let{surpluses}{$\text{values} - \text{interpolated\_values}$}
      \State Append(surrogate, indices, surpluses)
      \If{IsLimitReached(surrogate)}
        \State \textbf{break}
      \EndIf
      \For{k \textbf{in range of} indices}
        \If{IsAccurateEnough(surpluses[k])}
          \State Exclude(indices, k)
        \EndIf
      \EndFor
      \If{IsEmpty(indices)}
        \State \textbf{break}
      \EndIf
      \Let{indices}{ComputeChildren(indices)}
      \Let{level}{$\text{level} + 1$}
    \EndLoop

    \State \textbf{return} surrogate

    \vspace{0.4em}
  \end{algorithmic}
\end{algorithm}

The conceptual code corresponding to the construction phase is given in
\aref{construct} called \token{Construct}.

\begin{remark}
In the pseudocodes presented in this paper, many implementation details---such
as memory management---have been purposely omitted in order to distill the core
ideas. In addition, some of the auxiliary subroutines that the algorithms make
use of are not described as being self-explanatory.
\end{remark}

The input to \token{Construct}, \token{target}, is the function to be
approximated. The output, \token{surrogate}, is a structure containing the
artifacts mentioned earlier: indices and surpluses. The outermost loop on line~4
corresponds to the increasing level of Smolyak's interpolation, which is denoted
by $l$ in \eref{smolyak} and \eref{smolyak-incremental} and by \token{level} in
the code. The \token{indices} variable is a working set containing the indexing
pairs specific to the current level. The set is initially populated on line~2 by
the indices of the zeroth level. For the open Newton--Cotes rule, it is the zero
pair $(\v{0}, \v{0})$, which correspond to the root node $(0.5)_{i = 1}^\nin$
and the basis function $\e^\v{0}_\v{0}(\vx) = 1$. \token{ComputeNodes} on line~5
takes a set of indexing pairs and returns the corresponding nodes of the sparse
grid; see \sref{collocation-nodes}. \token{Execute} on line~6 calls the target
function at each of the given nodes and returns the corresponding values.
\token{Evaluate} on line~7 uses the surrogate constructed so far in order to
calculate an approximation to the values of \token{target} obtained on line~6;
\token{Evaluate} is \aref{evaluate}, and it will be described separately.
\token{Append} on line~9 ameliorates the surrogate by extending it with the
indices and surpluses of the current iteration/level. The check on line~10 is to
stop the algorithm when it reaches a limit on a resources such as the
construction time, interpolation level, and total number of \token{target}'s
invocations. The loop on line~14 iterates over the collocation nodes of the
current level and identifies those nodes that need refinement.

\begin{algorithm}
  \caption{\emph{Evaluate} an interpolant at a set of points.}
  \alab{evaluate}
  \begin{algorithmic}[1]
    \vspace{0.4em}

    \Require{surrogate} \Comment{An approximating object}
    \Require{points} \Comment{Points of interest}
    \Ensure{values} \Comment{The approximated values}

    \vspace{0.4em}

    \Let{indices}{GetIndices(surrogate)}
    \Let{surpluses}{GetSurpluses(surrogate)}
    \Let{values}{InitializeZeroed()}

    \For{point \textbf{in} points}
      \For{k \textbf{in range of} indices}
        \Let{weight}{ComputeBasis(indices[k], point)}
        \Let{values[k]}{$\text{values[k]} + \text{weight} \times \text{surpluses[k]}$}
      \EndFor
    \EndFor

    \State \textbf{return} values

    \vspace{0.4em}
  \end{algorithmic}
\end{algorithm}

The pseudocode of the interpolating procedure is given in \aref{evaluate}.
