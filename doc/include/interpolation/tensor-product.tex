In one dimension ($\nin = 1$), $\f$ is approximated by virtue of the following
interpolating formula:
\begin{equation} \elab{tensor-1d}
  \tensor{i}(\f) := \sum_{j \in \index(i)} \f(\x^i_j) \, \e^i_j.
\end{equation}
The superscript $i \in \natural$ is the level of interpolation; $\index(i) = \{
0, 1, \dots, \n_i - 1 \} \subset \natural$ is a set with cardinality $\n_i$ that
indexes the nodes on the corresponding level; $\X^i = \{ \x^i_j \} \subset [0,
1]$ are the nodes; and $\E^i = \{ \e^i_j \} \subset \continuous([0, 1])$ are the
basis functions. The subscript $j$ is referred to as the order of a node. The
choice of the collocation nodes and basis functions is an important concern
which will be thoroughly discussed later on.

In multiple dimensions ($\nin > 1$), $\f$ is approximated by the tensor product
of $\nin$ one-dimensional interpolants:
\begin{equation} \elab{tensor}
  \tensor{\vi}(\f) := (\tensor{i_1} \otimes \cdots \otimes \tensor{i_\nin})(\f) = \sum_{\vj \in \index(\vi)} \f(\vx^\vi_\vj) \, \e^\vi_\vj
\end{equation}
where $\vi = (i_k) \in \natural^\nin$ and $\vj = (j_k) \in \natural^\nin$ are
multi-indices specifying, respectively, the interpolation levels and node orders
for all dimensions, and $\index(\vi) := \index(i_1) \times \cdots \times
\index(i_\nin)$ is a set of multi-indices corresponding to the tensor-product
structure. Further,
\begin{align*}
  \X^\vi &= \X^{i_1} \times \cdots \times \X^{i_\nin} \\
         &= \left\{ \vx^\vi_\vj = (\x^{i_k}_{j_k}): \vj \in \index(\vi) \right\} \subset [0, 1]^\nin
\end{align*}
and
\begin{align*}
  \E^\vi &= \E^{i_1} \otimes \cdots \otimes \E^{i_\nin} \\
         &= \left\{ \e^\vi_\vj = \e^{i_1}_{j_1} \otimes \cdots \otimes \e^{i_\nin}_{j_\nin}: \vj \in \index(\vi) \right\} \subset \continuous([0, 1]^\nin)
\end{align*}
are the nodes and basis functions corresponding to the multi-index $\vi$,
respectively.
\begin{remark}
Each dimension can have its own rule defining the distribution of collocation
node with respect to each level. Similarly, the basis functions of one dimension
can differ from the basis functions of another. In order to improve readability,
this aspect is excluded from our notation.
\end{remark}
The cardinality of $\index(\vi)$, denoted by $\n_\vi$, is
\begin{equation} \elab{tensor-cardinality}
  \n_\vi = \prod_{k = 1}^\nin \n_{i_k}.
\end{equation}
The last equation elucidates the prohibited expense of the tensor-product
construction shown in \eref{tensor} for high-dimensional problems: the number of
nodes grows exponentially as $\nin$ increases. However, \eref{tensor} serves
well as a building block for more efficient algorithms, which we discuss next.
