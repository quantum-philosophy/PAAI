\subsection{System Model}

The system that we shall consider consists of two major components: a
multiprocessor platform and an application. The platform is defined as a set of
cores (processing elements) $\cores = \{ \core_i: i = 1, \dots, \cc \}$. The
application is given as a directed acyclic graph $\app = (\tasks, \deps)$ where
$\tasks = \{ \task_i: i = 1, \dots, \tc \}$ is a set of vertices representing
tasks, and $\deps \subset \tasks \times \tasks$ is a set of edges representing
data dependencies between the tasks.

The mapping of the application onto the platform is assumed to be fixed and
given by a function $\mapping: \tasks \to \cores$. The power consumption of the
tasks is assumed to be fixed and given by a function $\power: \tasks \to
\real$. For example, for $\task \in \tasks$, $\mapping(\task)$ is the core that
the task is to be executed on, and $\power(\task)$ is the amount of power that
the tasks is estimated to be consuming. Note that the boundaries of a task have
not been specified, and one can perform modeling at the level of granularity
that makes the most sense for the problem at hand.

Each task $\task \in \tasks$ is characterized by two \rvs: its start time
$\start$ and finish time $\finish$. Both variables are assumed to be measurable
with respect to $\sigma(\U)$, \ie, the $\sigma$-algebra generated by $\U$
\cite{durrett2010}. Loosely speaking, this means that $\start$ and $\finish$
are deterministic functions of $\U$; they are known whenever $\U$ is known. To
give an example, let $\U$ be a vector whose components $(\U_i)$ represent the
execution times of the tasks: $\U_i = \finish_i - \start_i$, for $i = 1, \dots,
\tc$. Then, assuming a deterministic scheduling policy, all $\start_i$ and
$\finish_i$ can be derived from $\U$.
