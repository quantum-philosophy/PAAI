Monte Carlo (\abbr{MC}) sampling \cite{xiu2010} is undoubtedly the most famous
and versatile approach to the analysis of stochastic systems. The technique was
introduced in the middle of the twentieth century and since then has had a
tremendous impact in innumerable contexts. The success of \abbr{MC} sampling is
due to the ease of implementation, independence of the stochastic
dimensionality, and asymptotic behavior of the quantities estimated using this
approach. The major problem with \abbr{MC} sampling, however, is in sampling:
one should be able to collect a sufficiently large number of realizations of the
quantity of interest in order to draw sound conclusions with respect to this
quantity. The main concern here is: How many samples can we afford drawing? How
long does it take to obtain one sample? How much does it cost? When the subject
is expensive---in any sense that matters to the problem at
hand---\abbr{MC}-based methods are rendered as slow and often infeasible.

In order to eliminate or reduce the costs associated with \abbr{MC} sampling, a
number of techniques have been introduced.

\cite{ukhov2014} and \cite{ukhov2015}.
