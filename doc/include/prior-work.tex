Monte Carlo (\abbr{MC}) sampling \cite{xiu2010} is arguably the most famous and
versatile approach to the analysis of stochastic systems. The technique was
introduced in the middle of the twentieth century and since then has had a
tremendous impact both in academia and in terms of industrial breakthroughs. The
success of \abbr{MC} sampling is due to the ease of implementation, independence
of the stochastic dimensionality, and asymptotic behavior of the quantities
estimated using this approach (the law of large numbers). The major problem with
\abbr{MC} sampling, however, is in sampling: one should be able to collect a
sufficiently large number of realizations of the quantity of interest in order
to draw sound conclusions with respect to this quantity. The main concern here
is: How many samples can we afford drawing? How long does it take to obtain one
sample? How much does it cost? When the subject under analysis is expensive---in
any sense that matters to the problem at hand---\abbr{MC}-based methods are
rendered as slow and often infeasible.

In order to eliminate or reduce the costs associated with \abbr{MC} sampling, a
number of techniques have been introduced.

Probabilistic timing analysis.

Probabilistic power analysis. The analysis under process variation is considered
in \cite{ukhov2014}.

Let us turn to probabilistic temperature analysis. The analysis under process
variation is considered in \cite{ukhov2014} and \cite{ukhov2015}.
