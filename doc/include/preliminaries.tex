\subsection{System Model}
A multiprocessor system is defined as a tuple $(\platform, \application)$ where $\platform$ is a platform and $\application$ is an application. $\platform$ is composed of a set of processing elements $\{ \processor_i: i = 1, \dots, \Npe \}$ equipped with thermal package. All the components of $\platform$ are characterized by a thermal specification $\specification$ of the system defined as a collection of temperature-related information. The application $\application$ is given as a directed acyclic graph $(\tasks, \dependencies)$ where $\tasks = \{ \task_i: i = 1, \dots, \Nts \}$ is a set of tasks, and $\dependencies = \{ (\task_{i_1}, \task_{i_2}) \}$ is a set of data dependencies between $\tasks$. The binary matrix $\mapping \in \{ 0, 1 \}^{\Npe \times \Nts}$ denotes the mapping of $\application$ onto $\platform$ where the $(i, j)$th element is equal to one only if the $j$th task is to be executed on the $i$th processing element; otherwise, the element is zero.

\subsection{Uncertainty Model}
Let $(\O, \F, \P)$ be a complete probability space where $\O$ is a set of outcomes, $\F$ is a $\sigma$-algebra on $\O$, and $\P: \O \to [0, 1]$ is a probability measure. The system depends on a number of uncertain parameters denoted by a set of random variables (\rvs) $\vU(\o) = \{ \U_i(\o): i = 1, \dots, \Nup \}$, $\o \in \O$. Since the knowledge of the \emph{joint} distribution functions of $\vU(\o)$ is unrealistic in practice,\footnote{If the joint distribution function of $\vU(\o)$ is available, Rosenblatt's transformation is the most preferable choice.} only the \emph{marginal} distribution functions $\CDF_{\U_i}(\u)$ and a matrix of correlations $\mCorr_\vU$ are assumed to be given. $\mCorr_\vU$ captures either solely linear correlations, which is denoted by $\mLCorr_\vU$, via the Pearson's correlation coefficient (i.e., the matrix is the ordinary correlation matrix) or nonlinear correlations as well, which is denoted by $\mNCorr_\vU$, via a rank correlation coefficient (e.g., the Spearman's rho or Kendall's tau).

Each task $\task_i \in \tasks$ is characterized by the corresponding execution time, duration, denoted by a positive \rv\ $\D_i(\o) \in \real_+$; the \rvs\ $\D_i(\o)$, $\forall i$, form a random vector $\vD(\o) \in \real^\Nts_+$. $\vD(\o)$ are assumed to be given as (possibly implicit) functionals of $\vU(\o)$, i.e., $\vD(\o) = f(\vU(\o))$. In the simplest case, $\vD(\o) \equiv \vU(\o)$.
