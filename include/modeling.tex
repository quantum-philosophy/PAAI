The agenda for this section is as follows. In \sref{parameters}, the uncertain
parameters $\vu$ are transformed in a form suitable for the subsequent
calculations. In the rest of the subsections, \sref{time}, \sref{power}, and
\sref{temperature}, a number of system models are introduced, which give birth
to various quantities that are of interest to the designer of electronic
systems.

\subsection{Uncertainty Parameters} \slab{parameters}
The foremost step of our framework is to change the parameterization of the
problem from the random vector $\vu = (\u_i)_{i = 1}^\nu \sim \distribution_\vu$
to an auxiliary random vector $\vz = (\z_i)_{i = 1}^\nz \sim \distribution_\vz$
such that: 1) the support of $\distribution_\vz$ is the unit hypercube $[0,
1]^\nz$, and 2) $\nz \leq \nu$ has the smallest value needed to retain the
desired level of accuracy. The first is standardization, which is done primarily
for convenience. The second is model-order reduction, which identifies and
eliminates excessive complexity and, hence, speeds up the solution process. The
reduction is possible whenever there are dependencies between $(\u_i)_{i =
1}^\nu$, in which case one can find such $(\z_i)_{i = 1}^\nz$, $\nz < \nu$, that
each $\u_i$ can be recovered from $(\z_i)_{i = 1}^\nz$. We shall denote the
overall transformation by $\vu = \transformation(\vz)$ where
\begin{equation} \elab{transformation}
  \transformation: \real^\nu \to [0, 1]^\nz.
\end{equation}
For any point $\vz \in [0, 1]^\nz$, we are now able to compute the corresponding
$\vu$ and, consequently, the quantity of interest $\g$ as $(\g \circ
\transformation)(\vz) = \g(\transformation(\vz)) = \g(\vu)$; recall also
\sref{problem}.

Let us consider an example of $\transformation$ in order to understand the
concept better. To this end, we begin by assuming that the distribution of $\vu
= (\u_i)_{i = 1}^\nu$, $\distribution_\vu$, is given as a set of marginal
distribution functions $\{ \distribution_{\u_i} \}_{i = 1}^\nu$ and a copula
\cite{nelsen2006} (see also \sref{preliminaries}). Furthermore, the copula is
assumed to be a Gaussian copula whose correlation matrix is $\correlation \in
\real^{\nu \times \nu}$.

It is important to note the following. A set of marginals and a copula entirely
characterize the joint distribution of $\vu$, that is, $\distribution_\vu$.
However, we consider this distribution as an approximation rather than as the
true one. The knowledge of the true joint would be an impractical assumption to
make. A more realistic assumption is the availability of the marginals and
correlation matrix of $\vu$. In general, these two pieces are not sufficient to
recover the joint of $\vu$; however, the joint can be approximated well by
accompanying the available marginals by a Gaussian copula constructed based on
the available correlation matrix; see \cite{liu1986} and also \cite{ukhov2014}.
Hence, a set of marginals and a Gaussian copula are practical inputs to
probabilistic analysis.

The number of variables, which is so far $\nu$, has a significant impact on the
complexity of the problem at hand. Therefore, an important component of our
framework is model-order reduction, which we shall base on the discrete
Karhunen--Lo\`{e}ve decomposition, also known as the principal component
analysis. We proceed as follows. Since any correlation matrix is real and
symmetric, $\correlation$ admits the eigendecomposition: $\correlation = \m{V}
\m{\Lambda} \m{V}^T$ where $\m{V} \in \real^{\nu \times \nu}$ is an orthogonal
matrix whose columns are the eigenvectors of $\correlation$, and $\m{\Lambda} =
\diag(\lambda_i)_{i = 1}^\nu$ is a diagonal matrix whose diagonal elements are
the eigenvalues of $\correlation$. The eigenvalues $(\lambda_i)_{i = 1}^\nu$
correspond to the variances of the corresponding components revealed by the
decomposition. The model-order reduction boils down to selecting those major
components whose cumulative contributions to the total variance is above a
certain threshold. Formally, assuming that $(\lambda_i)_{i = 1}^\nu$ are sorted
in the descending order and given a threshold $\eta \in (0, 1]$ specifying the
fraction of the total variance to be preserved, we identify the smallest $\nz$
such that
\begin{equation} \elab{reduction}
  \frac{\sum_{i = 1}^\nz \lambda_i}{\sum_{i = 1}^\nu \lambda_i} \geq \eta.
\end{equation}
Denote by $\tilde{\m{V}} \in \real^{\nu \times \nz}$ and $\tilde{\m{\Lambda}}
\in\real^{\nz \times \nz}$ the matrices obtained by truncating $\m{V}$ and
$\m{\Lambda}$, respectively, to preserve only the first $\nz$ components where
$\nz$ is as shown above.

Now, the transformation $\transformation$ in \eref{transformation} is
\begin{equation} \elab{transformation-concrete}
  \vu = \distribution_\vu^{-1} \left( \Phi\left( \tilde{\m{V}} \tilde{\m{\Lambda}}^\frac{1}{2} \, \Phi^{-1}(\vz) \right) \right)
\end{equation}
where the \rvs\ $\vz = (\z_i)_{i = 1}^\nz$ are independent and uniformly
distributed on $[0, 1]^\nz$; $\Phi$ and $\Phi^{-1}$ are the distribution
function of the standard Gaussian distribution and its inverse, respectively,
which are applied elementwise; and $\distribution_\vu^{-1} =
\distribution_{\u_1}^{-1} \times \cdots \times \distribution_{\u_\nz}^{-1}$ is
the Cartesian product of the inverse marginal distributions of $\vu$, which are
applied to the corresponding element of the vector yielded by $\Phi$. In the
absence of correlations, \eref{transformation-concrete} is simply $\vu =
\distribution_\vu^{-1}(\vz)$, and no model-order reduction is possible ($\nu =
\nz$). It might be interesting to note that, using the above transformation, the
distribution $\vu$ gets ``baked'' in $\g$ because it ``disappears'' when $\g$ is
viewed as a function of $\vz$, distributed uniformly.

To summarize, we have found such a transformation $\transformation$ and the
corresponding random vector $\vz \sim \distribution_\vz$ that: 1)
$\distribution_\vz$ is supported by $[0, 1]^\nz$, and 2) $\vz$ has the smallest
number of dimensions $\nz$ needed to preserve $\eta$ portion of the variance.


\subsection{Application Timing} \slab{time}
Suppose the application is given as a directed acyclic graph. The vertices
represent tasks, and the edges data dependency between these tasks. Suppose
further that a static cyclic scheduling policy is utilized. Note, however, these
assumptions are orthogonal to our framework: the framework can be applied to any
application model and any scheduling policy.

Each task has a start and a finish time. For task $i$, denote these two time
moments by $\b_i$ and $\d_i$, respectively, and let $\vb = (\b_i)_{i=1}^\nt$ and
$\vd = (\d_i)_{i=1}^\nt$. Other timing characteristics of the application can be
derived from $(\vb, \vd)$. An example is the end-to-end delay, which is the
difference between the finish time of the latest task and the start time of the
earliest task:
\begin{equation} \elab{end-to-end-delay}
  \text{End-to-end delay} = \max_{i = 1}^\nt \, \d_i - \min_{i = 1}^\nt \, \b_i.
\end{equation}

Suppose the execution times of the tasks depend on $\vu$ (see \sref{problem}).
Then the tuple $(\vb, \vd)$ depends on $\vu$. \updated{Then the end-to-end delay
given in \eref{end-to-end-delay} depends on $\vu$ and constitutes a potential
metric $\g$.} Note that this $\g$ is nondifferentiable as the $\max$ and $\min$
functions are such. Hence, $\g$ is nonsmooth, which renders \up{PC} expansions
and similar techniques inadequate for this problem, as illustrated in
\sref{introduction}.

\begin{remark} \rlab{smoothness}
In general, the behavior of $\g$ with respect to continuity, differentiability,
and smoothness cannot be inferred from the behavior of $\vu$. Even when the
parameters are perfectly behaved, $\g$ can still and likely will exhibit
nondifferentiability or even discontinuity, which depends on how $\g$ works
internally. For example, as shown in \cite{tanasa2015}, even if execution times
of tasks are continuous, due to the actual scheduling policy, end-to-end delays
are very often discontinuous.
\end{remark}


\subsection{Power Consumption} \slab{power}
Denote the number of processing elements present on the platform by $\np$. Let
the dynamic power consumed by task $j$ when running on processing element $i$ be
fixed during the execution of the task and denote this dynamic power by
$\p^\dynamic_{ij}$. The fact that $\p^\dynamic_{ij}$ is constant might seem
restrictive. However, one should keep in mind that it is a example. Our
framework does not have such a restriction. Even in this simple model, the
modeling accuracy can be substantially improved by representing large tasks as
sequences of smaller tasks.

Let the vector $\vp(\t) = (\p_i(\t))_{i = 1}^{\np}$ capture the total power
consumption of the system at time $\t$. This vector is related to the dynamic
power introduced above as follows:
\begin{equation} \elab{power}
  \p_i(\t) = \sum_{j = 1}^\nt \p^\dynamic_{ij} \: \delta_{ij} (\t) + \p^\static_i(\t), \hspace{1em} \text{for $i = 1, \dots, \np$},
\end{equation}
where $\delta_{ij}(\t)$ is an indicator function (outputs either zero or one) of
the event that processing element $i$ executes task $j$ at time $\t$, and
$\p^\static_i(\t)$ is the static power consumed by processing element $j$ at
time $\t$. The last component depends on time because the leakage power and
temperature are interdependent \cite{liu2007}, and temperature changes over time
(see the next subsection).

Given a set of $\ns$ points on the timeline $\{ \t_i \}_{i = 1}^\ns$,
\eref{power} can be used to construct a power profile of the system as follows:
\[
  \mP = (\p_i(\t_j))_{i = 1, j = 1}^{\np, \ns} \in \real^{\np \times \ns}.
\]
The above is essentially a matrix where row $i$ captures the power consumed by
processing element $i$ at the $\ns$ time moments.

The total energy consumed by the system during an application run can be
computed by integrating \eref{power} over the time span of the
application---which is demarcated by the minuend and subtrahend in
\eref{end-to-end-delay}---and the corresponding integral can be estimated using
the power profile as follows:
\begin{equation} \elab{total-energy}
  \text{Total energy} = \sum_{i = 1}^\np \int \p_i(\t) \, \d\t \approx \sum_{i = 1}^\np \sum_{j = 1}^\ns \p_i(\t_j) \, \Delta\t_j
\end{equation}
where $\Delta\t_j$ is either $\t_j - \t_{j - 1}$ or $\t_{j + 1} - \t_j$,
depending on how power values are encoded in $\mP$. The assumption that
\eref{total-energy} is based on is that each $\Delta\t_i$ is sufficiently small
so that the power consumed within the interval does not change significantly.

Since the tuple $(\vb, \vd)$ depends on $\vu$, the power consumption of the
system depends on $\vu$ too. Consequently, the total energy given in
\eref{total-energy} depends on $\vu$ and is a candidate for $\g$.


\subsection{Heat Dissipation} \slab{temperature}
Based on the specification of the platform including its thermal package, an
equivalent thermal \up{RC} circuit is constructed \cite{skadron2004}. The
circuit comprises $\nn$ thermal nodes, and its structure depends on the intended
level of granularity, which impacts the resulting accuracy. For clarity, we
assume that each processing element is mapped onto one corresponding node, and
the thermal package is represented as a set of additional nodes. As before, let
us remind that the \up{RC} model is an example; other models can be used with
our framework equally well.

The thermal dynamics of the system are modeled using the following system of
differential-algebraic equations \cite{ukhov2014, ukhov2012}:
\begin{subnumcases}{\elab{thermal-system}}
  \mC \frac{\d\vs(\t)}{\d\t} + \mG \vs(\t) = \mM \vp(\t) \\
  \vq(\t) = \mM^T \vs(\t) + \vq_\ambient
\end{subnumcases}
The coefficients $\mC \in \real^{\nn \times \nn}$ and $\mG \in \real^{\nn \times
\nn}$ are a diagonal matrix of thermal capacitance and a symmetric,
positive-definite matrix of thermal conductance, respectively. The vectors
$\vp(\t) \in \real^\np$,  $\vq(\t) \in \real^\np$, and $\vs(\t) \in \real^\nn$
correspond the system's power, temperature, and internal state at time $\t$,
respectively. The vector $\vq_\ambient \in \real^\np$ contains the ambient
temperature. The matrix $\mM \in \real^{\nn \times \np}$ is a mapping that
distributes the power consumption of the processing elements across the thermal
nodes; without loss of generality, $\mM$ is a rectangular diagonal matrix whose
diagonal elements are equal to one.

Given a set of $\ns$ points on the timeline $\{ \t_i \}_{i = 1}^\ns$,
\eref{thermal-system} can be used to compute a temperature profile of the system
as follows:
\begin{equation*}
  \mQ = (\q_i(\t_j))_{i = 1, j = 1}^{\np, \ns} \in \real^{\np \times \ns}.
\end{equation*}
Then the maximum temperature of the system can be estimated using the
temperature profile as follows:
\begin{equation} \elab{maximum-temperature}
  \text{Max temperature} = \max_{i = 1}^\np \, \sup_{\t} \, \q_i(\t) \approx \max_{i = 1}^\np \max_{j = 1}^\ns \, \q_i(\t_j).
\end{equation}

Since the power consumption of the system is affected by $\vu$ (see
\sref{power}), the system's temperature is affected by $\vu$ as well. Therefore,
the temperature in \eref{maximum-temperature} can be considered as a quantity of
interest $\g$. Note that, due to the maximization involved, the quantity is
nondifferentiable and, hence, cannot be adequately addressed using polynomial
approximations, specially taking into account the concern in \rref{smoothness}.


To conclude, we have covered three facets of an electronic system, namely,
timing, power, and heat, and introduced a number of quantities associated with
them, which we will come back to in the section on experimental results,
\sref{experimentation}. We have also discussed the transformation that needs to
be applied to $\vu$ prior to the interpolation of $\g$, and now we are ready to
move on to the interpolation itself.
