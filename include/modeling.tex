The agenda for this section is as follows. In \sref{uncertain-parameters}, the
uncertain parameters $\vu$ are transformed in a form suitable for the subsequent
calculations. In the rest of the subsections, \sref{application-timing},
\sref{power-consumption}, and \sref{heat-dissipation}, a number of system models
are introduced, which give birth to various quantities that are of interest to
the designer of electronic systems.

\subsection{Uncertainty Parameters} \slab{uncertain-parameters}
The foremost step of our framework is to change the parameterization of the
problem from the random vector $\vu = (\u_i)_{i = 1}^\nu \sim \distribution_\vu$
to an auxiliary random vector $\vz = (\z_i)_{i = 1}^\nz \sim \distribution_\vz$
such that (i) the support of $\distribution_\vz$ is the unit hypercube $[0,
1]^\nz$, and (ii) $\nz \leq \nu$ has the smallest value needed to retain the
desired level of accuracy. Goal (i) is standardization, which is done primarily
for convenience. Goal (ii) is model-order reduction, which identifies and
eliminates excessive complexity and, hence, speeds up the proposed framework.
The reduction is possible whenever there are dependencies between $(\u_i)$, in
which case one can find such $(\z_i)_{i = 1}^\nz$, $\nz < \nu$, that each $\u_i$
can be recovered from $(\z_i)$. We shall denote the overall transformation by
$\vu = \transformation(\vz)$ where
\begin{equation} \elab{transformation}
  \transformation: \real^\nu \to [0, 1]^\nz.
\end{equation}
Now, for any point $\vz \in [0, 1]^\nz$, we are able to compute the
corresponding $\vu$ and, consequently, the quantity of interest $\g$ as $\g(\vu)
= \g(\transformation(\vz))$; see \sref{problem-formulation}.

Let us consider an example to get a better intuition about $\transformation$. To
this end, we begin by assuming that the distribution of $\vu = (\u_i)_{i =
1}^\nu$, $\distribution_\vu$, is given as a set of marginal distribution
functions
\[
  \{ \distribution_{\u_i}: i = 1, \dots, \nu \}
\]
and a copula \cite{nelsen2006}. The copula is a uniform distribution function on
$[0, 1]^\nu$ that captures the dependencies between $(\u_i)$. Furthermore, the
copula is assumed to be a Gaussian copula whose correlation matrix is
$\correlation \in \real^{\nu \times \nu}$.

\begin{remark}
A set of marginals and a copula entirely characterize the joint distribution of
$\vu$, $\distribution_\vu$. However, we consider this distribution to be an
approximation rather than the true one. The knowledge of the true joint would be
an impractical assumption to make. A more realistic assumption is the
availability of the marginals and correlation matrix of $\vu$. In general, these
two pieces are not sufficient to recover the joint of $\vu$; however, the joint
can be approximated well by accompanying the available marginals by a Gaussian
copula constructed based on the available correlation matrix; see \cite{liu1986}
and also \cite{ukhov2014}. Hence, a set of marginals and a Gaussian copula are
a typical input to probabilistic analysis.
\end{remark}

The number of variables, which is so far $\nu$, has a significant impact on the
complexity of the problem at hand. Therefore, an important component of our
framework is model-order reduction, which we shall base on the discrete
Karhunen--Lo\`{e}ve decomposition, also known as the principal component
analysis. We proceed as follows. Since any correlation matrix is real and
symmetric, $\correlation$ admits the eigendecomposition:
\[
  \correlation = \m{V} \m{\Lambda} \m{V}^T
\]
where $\m{V} \in \real^{\nu \times \nu}$ is an orthogonal matrix of the
eigenvectors of $\correlation$, and $\m{\Lambda} = \diag(\lambda_i)_{i = 1}^\nu$
is a diagonal matrix of the eigenvalues of $\correlation$. The eigenvalues
$(\lambda_i)$ correspond to the variances of the corresponding components
revealed by the decomposition. The model-order reduction boils down to selecting
those components whose cumulative contributions to the total variance is above a
certain threshold. Formally, assuming that $(\lambda_i)$ are sorted in the
descending order and given a threshold parameter $\eta \in (0, 1]$, we identify
the smallest $\nz$ such that
\[
  \frac{\sum_{i = 1}^\nz \lambda_i}{\sum_{i = 1}^\nu \lambda_i} \geq \eta.
\]
Denote by $\tilde{\m{V}} \in \real^{\nu \times \nz}$ and $\tilde{\m{\Lambda}}
\in\real^{\nz \times \nz}$ the matrices obtained by truncating $\m{V}$ and
$\m{\Lambda}$, respectively, according to the strategy delineated above.

Now, the transformation $\transformation$ in \eref{transformation} is
\begin{equation} \elab{transformation}
  \vu = \distribution_\vu^{-1} \left( \Phi\left( \tilde{\m{V}} \tilde{\m{\Lambda}}^\frac{1}{2} \, \Phi^{-1}(\vz) \right) \right)
\end{equation}
where the \rvs\ $\vz = (\z_i)_{i = 1}^\nz$ are independent and uniformly
distributed on $[0, 1]$; $\Phi$ and $\Phi^{-1}$ are the distribution function of
the standard Gaussian distribution and its inverse, respectively, which are
applied elementwise; and $\distribution_\vu^{-1} := \distribution_{\u_1}^{-1}
\otimes \cdots \otimes \distribution_{\u_\nz}^{-1}$ are the inverse marginal
distributions of $\vu$, each of which is applied to the corresponding element of
the vector yielded by $\Phi$. To summarize, we have found such a transformation
$\transformation$ and a random vector $\vz \sim \distribution_\vz$ that (i)
$\distribution_\vz$ is supported by $[0, 1]^\nz$, and (ii) $\vz$ has the
smallest number of dimensions $\nz$ needed to preserve $\eta$ portion of the
variance. Besides, $\distribution_\vz$ is trivial to draw samples from.

\begin{remark}
In the absence of correlations, \eref{transformation} is simply $\vu =
\distribution_\vu^{-1}(\vz)$, and no model-order reduction is possible.
\end{remark}


\subsection{Application Timing} \slab{application-timing}
Each task is ascribed two variables: one is the beginning of the task's
execution time interval, and the other is the duration of this interval. For the
$i$th task, denote the two variables by $\b_i$ and $\d_i$, respectively, and let
$\vb = (\b_i)_{i=1}^{\nt}$ and $\vd = (\d_i)_{i=1}^{\nt}$. The tuple $(\vb,
\vd)$ becomes completely known after the application has been scheduled and
executed.

All other timing characteristics of the system can be found or derived from the
tuple $(\vb, \vd)$. Examples include the completion times of the tasks and the
end-to-end delay of the entire application. The latter, for instance, is
\begin{equation} \elab{end-to-end-delay}
  \text{End-to-end delay} = \max_{i = 1}^\nt \, (\b_i + \d_i).
\end{equation}

Any of the quantities mentioned above (and their combinations) can be considered
as the quantity of interest $\g$.


\subsection{Power Consumption} \slab{power-consumption}
Let the dynamic power consumed by task $j$ when running on processing element
$i$ be fixed during the execution of the task and denote this dynamic power by
$\p^\dynamic_{ij}$.

\begin{remark}
The boundaries of a task have not been specified. The modeling can be performed
at the level of granularity that makes the most sense for a particular problem.
\end{remark}

Let the vector $\vp(\t) = (\p_i(\t))_{i = 1}^{\np}$ capture the total power
consumption of the system at time $\t$. This vector is related to the dynamic
power introduced above as follows:
\begin{equation} \elab{power}
  \p_i(\t) = \sum_{j = 1}^\nt \p^\dynamic_{ij} \: \delta_{ij} (\t) + \p^\static_i(\t), \hspace{1em} \text{for $i = 1, \cdots, \np$},
\end{equation}
where $\delta_{ij}(\t)$ is the indicator of the event that processing element
$i$ executes task $j$ at time $\t$, and $\p^\static_i(\t)$ is the static power
consumed by processing element $j$ at time $\t$. In general, the last component
varies with time due to the interdependence between the leakage power and
temperature \cite{liu2007}.

Given a set of $\ns$ points on the timeline $\{ \t_i \}_{i = 1}^\ns$,
\eref{power} can be used to construct a power profile of the system as follows:
\[
  \mP = (\p_i(\t_j))_{i = 1, j = 1}^{\np, \ns} \in \real^{\np \times \ns}.
\]
The above is essentially a matrix where row $i$ captures the power consumed by
processing element $i$ at the $\ns$ time moments.

The total energy consumed by the system during an application run can be
computed by integrating \eref{power} over the time span of the
application---which is demarcated by the minuend and subtrahend in
\eref{end-to-end-delay}---and the corresponding integral can be estimated using
the power profile as follows:
\begin{equation} \elab{total-energy}
  \text{Total energy} = \sum_{i = 1}^\np \int \p_i(\t) \, \d\t \approx \sum_{i = 1}^\np \sum_{j = 1}^\ns \p_i(\t_j) \, \Delta\t_j
\end{equation}
where $\Delta\t_j$ is either $\t_j - \t_{j - 1}$ or $\t_{j + 1} - \t_j$,
depending on how power values are encoded in $\mP$. The assumption that
\eref{total-energy} is based on is that each $\Delta\t_i$ is sufficiently small
so that the power consumed within the interval does not change much.

Assuming that the power consumption depends on $\vu$---which is the case
whenever, for instance, $(\vb, \vd)$ depends on $\vu$---the total energy given
in \eref{total-energy} is a candidate for $\g$.


\subsection{Heat Dissipation} \slab{heat-dissipation}
Based on the specification of the platform including its thermal package, an
equivalent thermal \abbr{RC} circuit is constructed \cite{skadron2004}. The
circuit comprises $\nn$ thermal nodes, and its structure depends on the intended
level of granularity, which impacts the resulting accuracy. For clarity, we
assume that each processing element is mapped onto one corresponding node, and
the thermal package is represented as a set of additional nodes.

The thermal dynamics of the system are modeled using the following system of
differential-algebraic equations \cite{ukhov2014, ukhov2012}:
\begin{subnumcases}{\elab{thermal-system}}
  \mC \frac{d\vs(\t)}{d\t} + \mG \vs(\t) = \mM \vp(\t) \elab{thermal-system-ode} \\
  \vq(\t) = \mM^T \vs(\t) + \vq_\ambient
\end{subnumcases}
The coefficients $\mC \in \real^{\nn \times \nn}$ and $\mG \in \real^{\nn \times
\nn}$ are a diagonal matrix of thermal capacitance and a symmetric,
positive-definite matrix of thermal conductance, respectively. The vector
$\vq(\t) \in \real^\np$ represents the temperature of the system at time $\t$
while $\vs(\t) \in \real^\nn$ is the system's internal state at that moment. The
vector $\vq_\ambient \in \real^\np$ contains the ambient temperature. The matrix
$\mM \in \real^{\nn \times \np}$ is a mapping that distributes the power
consumption of the processing elements across the thermal nodes; without loss of
generality, $\mM$ is a rectangular diagonal matrix whose diagonal elements are
equal to one.

Given a set of $\ns$ points on the timeline $\{ \t_i \}_{i = 1}^\ns$,
\eref{thermal-system} can be used to compute a temperature profile of the system
as follows:
\begin{equation*}
  \mQ = (\q_i(\t_j))_{i = 1, j = 1}^{\np, \ns} \in \real^{\np \times \ns}.
\end{equation*}
Then the maximum temperature of the system can be estimated using the
temperature profile as follows:
\begin{align}
  \text{Maximum temperature} &= \max_{i = 1}^\np \, \sup_{\t} \, \q_i(\t) \nonumber \\
                             &\approx \max_{i = 1}^\np \max_{j = 1}^\ns \, \q_i(\t_j). \elab{maximum-temperature}
\end{align}

Assuming that temperature depends on $\vu$---which is the case whenever, for
instance, the power consumption depends on $\vu$---a temperature profile as well
as the maximum temperature can be considered as quantities of interest $\g$.
Dynamic steady-state temperature analysis \cite{ukhov2012} can also be
considered in this regards, which is particularly useful for reliability
analysis \cite{ukhov2015}.


To conclude, we have covered three facets of an electronic system, namely,
timing, power, and heat, and introduced a number of quantities associated with
them, which we will come back to in the section on experimental results,
\sref{experimentation}. We have also discussed the transformation that needs to
be applied to $\vu$ prior to the interpolation of $\g$, and now we are ready to
move on to the interpolation itself.
