The agenda for this section is as follows. In \sref{uncertain-parameters}, the
uncertain parameters $\vu$ are transformed in a form suitable for the subsequent
calculations. In the rest of the subsections, \sref{application-timing},
\sref{power-consumption}, and \sref{heat-dissipation}, a number of system models
are introduced, which give birth to various quantities that are of interest to
the designer of electronic systems.

\subsection{Uncertainty Parameters} \slab{uncertain-parameters}
The foremost step of our framework is to change the parameterization of the
problem from the random vector $\vu = (\u_i)_{i = 1}^\nu \sim \distribution_\vu$
to an auxiliary random vector $\vz = (\z_i)_{i = 1}^\nz \sim \distribution_\vz$
such that (i) the support of $\distribution_\vz$ is the unit hypercube $[0,
1]^\nz$, and (ii) $\nz \leq \nu$ has the smallest value needed to retain the
desired level of accuracy. Goal (i) is standardization, which is done primarily
for convenience. Goal (ii) is model-order reduction, which identifies and
eliminates excessive complexity and, hence, speeds up the proposed framework.
The reduction is possible whenever there are dependencies between $(\u_i)$, in
which case one can find such $(\z_i)_{i = 1}^\nz$, $\nz < \nu$, that each $\u_i$
can be recovered from $(\z_i)$. We shall denote the overall transformation by
$\vu = \transformation(\vz)$ where
\begin{equation} \elab{transformation}
  \transformation: \real^\nu \to [0, 1]^\nz.
\end{equation}
Now, for any point $\vz \in [0, 1]^\nz$, we are able to compute the
corresponding $\vu$ and, consequently, the quantity of interest $\g$ as $\g(\vu)
= \g(\transformation(\vz))$; see \sref{problem-formulation}.

We do not impose any restrictions on $\transformation$; however, we would like
to make it specific in order to give a better intuition. To this end, we begin
by assuming that the distribution of $\vu = (\u_i)_{i = 1}^\nu$,
$\distribution_\vu$, is given as a set of marginal distribution functions
\[
  \{ \distribution_{\u_i}: i = 1, \dots, \nu \}
\]
and a copula \cite{nelsen2006}. The copula is a uniform distribution function on
$[0, 1]^\nu$ that captures the dependencies between $(\u_i)$. Furthermore, the
copula is assumed to be a Gaussian copula whose correlation matrix is denote by
$\correlation \in \real^{\nu \times \nu}$.

\begin{remark}
A set of marginals and a copula are sufficient to fully characterize the joint
distribution of $\vu$, $\distribution_\vu$. However, we consider this
distribution to be an approximation rather than the true one. The knowledge of
the true distribution would be an impractical assumption to make. A more
realistic assumption is the availability of the marginals and correlation matrix
of $\vu$, although they are not sufficient to recover the joint of $\vu$ in
general. One prominent solution is to approximate the joint by accompanying the
available marginals by a Gaussian copula constructed based on the available
correlation matrix (see, \eg, \cite{ukhov2014}). This common scenario motivates
our choice of marginals and a Gaussian copula for illustration.
\end{remark}

The number of variables, which is so far $\nu$, has a significant impact on the
computational complexity of the proposed framework. Therefore, an important
component of the framework is model-order reduction, which we shall base on the
discrete Karhunen--Lo\`{e}ve decomposition, also known as the principal
component analysis. We proceed as follows. Since any correlation matrix is real
and symmetric, $\correlation$ admits the eigendecomposition:
\[
  \correlation = \m{V} \m{\Lambda} \m{V}^T
\]
where $\m{V} \in \real^{\nu \times \nu}$ is an orthogonal matrix of the
eigenvectors of $\correlation$, and $\m{\Lambda} = \diag(\lambda_i)_{i = 1}^\nu$
is a diagonal matrix of the eigenvalues of $\correlation$. The eigenvalues
$(\lambda_i)$ correspond to the variances of the corresponding components
revealed by the decomposition. The model-order reduction boils down to selecting
those components whose cumulative contributions to the total variance is above a
certain threshold. Formally, assuming that $(\lambda_i)$ are sorted in the
descending order and given a threshold parameter $\eta \in (0, 1]$, we identify
the smallest $\nz$ such that
\[
  \frac{\sum_{i = 1}^\nz \lambda_i}{\sum_{i = 1}^\nu \lambda_i} \geq \eta.
\]
Denote by $\tilde{\m{V}} \in \real^{\nu \times \nz}$ and $\tilde{\m{\Lambda}}
\in\real^{\nz \times \nz}$ the matrices obtained by truncating $\m{V}$ and
$\m{\Lambda}$, respectively, according to the strategy delineated above.

Now, the transformation $\transformation$ in \eref{transformation} is
\begin{equation} \elab{transformation}
  \vu = \distribution_\vu^{-1} \left( \Phi\left( \tilde{\m{V}} \tilde{\m{\Lambda}}^\frac{1}{2} \, \Phi^{-1}(\vz) \right) \right)
\end{equation}
where the \rvs\ $\vz = (\z_i)_{i = 1}^\nz$ are independent and uniformly
distributed on $[0, 1]$; $\Phi$ and $\Phi^{-1}$ are the distribution function of
the standard Gaussian distribution and its inverse, respectively, which are
applied elementwise; and $\distribution_\vu^{-1} := \distribution_{\u_1}^{-1}
\otimes \cdots \otimes \distribution_{\u_\nz}^{-1}$ are the inverse marginal
distributions of $\vu$, each of which is applied to the corresponding element of
the vector yielded by $\Phi$. To summarize, we have found such a transformation
$\transformation$ and the corresponding distribution $\distribution_\vz$ such
that $\distribution_\vz$ (i) is defined on $[0, 1]^\nz$ and (ii) has the minimal
number of dimensions to preserve $\eta$ portion of the variance. Besides,
$\distribution_\vz$ is trivial to draw samples from.

\begin{remark}
  In the absence of correlations, \eref{transformation} is simply $\vu =
  \distribution_\vu^{-1}(\vz)$, and no model-order reduction is possible.
\end{remark}


\subsection{Application Timing} \slab{application-timing}
Each task in $\tasks$ has a start time and a finish time. Denote these two time
moments of the $i$th task by $\b_i$ and $\d_i$, respectively, and let $\vb =
(\b_i)_{i=1}^{\nt}$ and $\vd = (\d_i)_{i=1}^{\nt}$. All other timing
characteristics of the system can be derived from the tuple $(\vb, \vd)$. An
example is the end-to-end delay of the application, which is defined as the
difference between the finish time of the last task and the start time of the
first task:
\begin{equation} \elab{end-to-end-delay}
  \text{End-to-end delay} = \max_{i \in \tasks} \, \d_i - \min_{i \in \tasks} \, \b_i.
\end{equation}

Assuming that the timing of the tasks depends on $\vu$, the quantities mentioned
above are potential quantities of interest $\g$, that is, potential targets to
probabilistic analysis.


\subsection{Power Consumption} \slab{power-consumption}
Denote the mapping of the application $\tasks$ onto the platform $\procs$ by a
vector $\mapping = (m_i)_{i = 1}^{\nt}$ wherein $m_i \in \procs$ is the index of
the processing element that the $i$th task is assigned to. Let the dynamic power
consumption of the tasks be fixed during their execution and given by a vector
$\vp^\dynamic = (\p^\dynamic_i)_{i = 1}^{\nt}$ wherein $\p^\dynamic_i$ is the
dynamic power of the $i$th task.

\begin{remark}
The boundaries of a task have not been specified, and one can perform modeling
at the level of granularity that makes the most sense for a particular problem.
\end{remark}

Let a time-dependent vector $\vp(\t) = (\p_i(\t))_{i = 1}^{\np}$ capture the
total power consumption of the system at time $\t$. This vector should not be
confused with the vector $\vp^\dynamic$ introduced in the previous paragraph.
The two are related as follows:
\begin{equation} \elab{power}
  \p_i(\t) = \sum_{j = 1}^\nt \p^\dynamic_j \: \delta_{i\,m_j} \: \one_{[\b_j, \b_j + \d_j)}(\t) + \p^\static_i(\t, \q_i(\t)),
\end{equation}
for $i = 1, \dots, \np$, where $\delta_{ij}$ is the Kronecker delta, $\one_A(x)$
is the indicator function of a set $A$, and $\p^\static_i$ is the static power
consumption of the $i$th processing element, which depends \cite{liu2007} on the
instant temperature of that processing element denoted by $\q_i$ (see the next
subsection). In words, the above equation yields the dynamic power of the task
running on the $i$th processing element at time $\t$, if any, plus the static
power.

Given a set of time moments of interest, \eref{power} can be used to construct
the corresponding power profile $\mP \in \real^{\np \times \ns}$ of the system,
which is essentially a matrix whose $i$th row captures the evolution of the
power consumption of the $i$th processing element over $\ns$ time moments.

Another highly valuable quantity, which can be calculated at this point, is the
total energy consumption. To this end, \eref{power} has to be integrated over
the time span of the application:
\begin{align}
  \text{Total energy} &= \sum_{i = 1}^\np \int \p_i(\t) \, d\t \elab{total-energy} \\
                      &= \vd^T \vp^\dynamic + \sum_{i = 1}^\np \int \p^\static_i(\t, \q_i(\t)) \, d\t. \nonumber
\end{align}

A power profile $\mP$ as well as the above quantity are candidates for $\g$. In
the former case, $\g$ has $\np \ns$ outputs.


\subsection{Heat Dissipation} \slab{heat-dissipation}
Based on the specification of the platform including its thermal package, an
equivalent thermal \abbr{RC} circuit is constructed \cite{skadron2004}. The
circuit comprises $\nn$ thermal nodes, and its structure depends on the intended
level of granularity, which impacts the resulting accuracy. For clarity, we
assume that each processing element is mapped onto one corresponding node, and
the thermal package is represented as a set of additional nodes.

The thermal dynamics of the system are modeled using the following system of
differential-algebraic equations \cite{ukhov2014, ukhov2012}:
\begin{subnumcases}{\elab{thermal-system}}
  \mC \frac{d\vs(\t)}{d\t} + \mG \vs(\t) = \mM \vp(\t) \elab{thermal-system-ode} \\
  \vq(\t) = \mM^T \vs(\t) + \vq_\ambient
\end{subnumcases}
The coefficients $\mC \in \real^{\nn \times \nn}$ and $\mG \in \real^{\nn \times
\nn}$ are a diagonal matrix of thermal capacitance and a symmetric,
positive-definite matrix of thermal conductance, respectively. The vector
$\vq(\t) \in \real^\np$ represents the temperature of the system at time $\t$
while $\vs(\t) \in \real^\nn$ is the system's internal state at that moment. The
vector $\vq_\ambient \in \real^\np$ contains the ambient temperature. The matrix
$\mM \in \real^{\nn \times \np}$ is a mapping that distributes the power
consumption of the processing elements across the thermal nodes; without loss of
generality, $\mM$ is a rectangular diagonal matrix whose diagonal elements are
equal to unity.

As noted with respect to power, given a set of time instances $\{ \t_i \}_{i =
1}^\ns$, \eref{thermal-system} can be utilized to compute a temperature profile
$\mQ \in \real^{\np \times \ns}$ of the system. Another example is the maximal
temperature, which can be estimated as follows:
\[
  \text{Maximal temperature} = \max_{i \in \procs} \, \sup_{\t} \, \q_i(\t).
\]

Assuming that temperature depends on $\vu$---which is the case whenever, for
instance, the power consumption depends on $\vu$---a temperature profile as well
as the maximal temperature can be considered as quantities of interest $\g$.
Dynamic steady-state temperature analysis \cite{ukhov2012} can also be
considered in this regards, which is particularly useful for reliability
analysis \cite{ukhov2015}.


To conclude, we have covered three facets of an electronic system, namely,
timing, power, and heat, and introduced a number of quantities associated with
them, which we will come back to in the section on experimental results,
\sref{experimentation}. We have also discussed the transformation that needs to
be applied to $\vu$ prior to the interpolation of $\g$, and now we are ready to
move on to the interpolation itself.
