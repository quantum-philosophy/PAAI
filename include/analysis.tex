In \sref{modeling}, we formalized the uncertainty affecting the timing
characteristics of multiprocessor systems and introduced a number of quantities
$\g$ that the designer is typically interested in knowing. In the previous
section, \sref{interpolation}, we were concerned with constructing efficient
approximations of hypothetical multi-input multi-output functions $\f$. Now, we
shall amalgamate the ideas developed in the aforementioned two sections.

Our solution has already been delineated in \sref{solution}. Given a quantity of
interest $\g$ dependent on the uncertain parameters $\vu \in \real^\nu$, we
reparametrize it in terms of $\vz \in [0, 1]^\nz$ and treat the resulting
composition as a deterministic function $\f$ defined on the unit hypercube (see
\sref{uncertain-parameters}). In other words, we let $\g \circ \transformation$
in \sref{modeling} be $\f$ in \sref{interpolation}. Next, we construct a light
interpolant for $\f = \g \circ \transformation$. Finally, we proceed in the
usual \abbr{MC} fashion by picking one of MC methods, sampling the interpolant
accordingly, and estimating the desired statistics about $\g$. For instance,
having collected a sufficiently large number of samples, one can perform kernel
density estimation in order to estimate the probability density function of the
quantity of interest. Furthermore, probabilistic moments and probabilities of
certain events can be estimated in a straightforward manner by algebraic
averaging.

There is, however, one class of quantities that deserves separate attention.
This class comprises dynamic characteristics of multiprocessor systems. Imagine,
for instance, a temperature profile $\mQ \in \real^{\np \times \ns}$, which is a
matrix wherein each row tracks the evolution of temperature of a processing
element over time with a certain sample rate. With $\np$ processing elements and
$\ns$ time moments, the total (output) dimensionality of the quantity of
interest is $\np\ns$, which is also the dimensionality of the surpluses that
\aref{construct} and \aref{evaluate} operate with.
