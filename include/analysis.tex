In \sref{modeling}, we formalized the uncertainty affecting the timing
characteristics of multiprocessor systems and introduced a number of quantities
$\g$ that the designer is typically interested in knowing. In the previous
section, \sref{interpolation}, we were concerned with constructing efficient
approximations of hypothetical multi-input multi-output functions $\f$. Now, we
shall amalgamate the ideas developed in the aforementioned two sections.

Our solution has already been delineated in \sref{solution}. Given a quantity of
interest $\g$ dependent on the uncertain parameters $\vu \in \real^\nu$, we
reparametrize it in terms of the \rvs\ $\vz \in [0, 1]^\nz$ via the
transformation $\transformation$ and construct an interpolant for the resulting
composition $\g \circ \transformation$, treating it a deterministic function
$\f$ defined on the unit hypercube (see \sref{uncertain-parameters}). We then
proceed to designing and undertaking a computer experiment, which boils down to
sampling the interpolant and estimating the desired statistics about $\g$. For
instance, having collected a sufficiently large number of samples, one can
perform kernel density estimation in order to estimate the probability density
function of $\g$. Furthermore, probabilistic moments and probabilities of
certain events can be estimated in a straightforward manner by algebraic
averaging.

There is, however, one class of quantities that deserves separate attention.
This class comprises dynamic characteristics of multiprocessor systems. Imagine,
for instance, a temperature profile $\mQ \in \real^{\np \times \ns}$, which is a
matrix wherein each row tracks the evolution of temperature of a processing
element over time with a certain sample rate. With $\np$ processing elements and
$\ns$ time moments, the total (output) dimensionality of the quantity of
interest is $\np\ns$, which is also the dimensionality of the surpluses that
\aref{construct} and \aref{evaluate} operate with.
