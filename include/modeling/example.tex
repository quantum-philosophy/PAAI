Before we move on to interpolation, let us take a moment and apply the proposed
framework to a small problem in order to get a better feel for how all the
pieces of the framework fit together. A detailed description of our experimental
setup is given in \sref{configuration}; here we give only the bare minimum.

\newcommand{\cores}{\token{PE1} and \token{PE2}}
\newcommand{\tasks}{\token{T1}--\token{T4}}
The addressed problem is depicted in \fref{example}. We consider a platform with
two homogeneous processing elements, \cores, and an application with four tasks,
\tasks. The data dependencies between \tasks\ and their mapping onto \cores\ can
be seen in \fref{example}. This delay is also our quantity of interest $\g$. The
uncertain parameters are the execution times of \token{T2} and \token{T4}
denoted by $\u_1$ and $\u_2$, respectively; the parameters are correlated, and
their marginal distributions are beta distributions. In order to make the
problem more challenging, the (deterministic) execution time of \token{T3} is
chosen to lie in the range of the possible values of the execution time of
\token{T2}, which curtails the impact of $\u_1$ on $\g$.

The leftmost box in \fref{example} represents a simulator of the system. The
second box corresponds to the transformation $\transformation$ discussed in
\sref{parameters}, and the third one to our interpolation engine discussed in
the next section. Using 156 strategic invocations of the simulator, the
interpolation engine yields a light surrogate for the simulator (the slim box
with rounded corners). Having obtained such a surrogate, one proceeds to
sampling extensively the surrogate via a sampling method of choice (the
rightmost box). Recall that the computation cost of this extensive sampling is
negligible. The collected samples are then used to compute an estimate of the
distribution of $\g$.

In the graph on the right-hand side of \fref{example}, the blue line shows the
probability density function of $\g$ computed by applying kernel density
estimation to the samples obtained from our surrogate. The yellow line (barely
visible behind the blue line) shows the true density of $\g$; its calculation is
explained in \sref{experimentation}. It can be seen that our solution closely
matches the exact one. In addition, the orange line shows the estimation that
one would get if one sampled $\g$ directly 156 times and used only those samples
in order to calculate the density of $\g$. We see that, for the same budget of
simulations, the solution delivered by our framework is substantially closer to
the true one than the one delivered by naive sampling.
