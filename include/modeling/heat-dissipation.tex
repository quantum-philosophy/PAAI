Based on the specification of the platform at hand, an equivalent thermal
\abbr{RC} circuit is constructed \cite{skadron2004}. The circuit comprises $\nn$
thermal nodes, and its structure depends on the intended level of granularity,
which impacts the resulting accuracy. For clarity, we assume that each
processing element is mapped onto one corresponding node, and the thermal
package is represented as a set of additional nodes.

The thermal dynamics of the system are modeled using the following system of
differential-algebraic equations \cite{ukhov2014, ukhov2012}:
\begin{subnumcases}{\elab{thermal-system}}
  \mC \frac{d\vs(\t)}{d\t} + \mG \vs(\t) = \mM \vp(\t) \elab{thermal-system-ode} \\
  \vq(\t) = \mM^T \vs(\t) + \vq_\ambient
\end{subnumcases}
The coefficients $\mC \in \real^{\nn \times \nn}$ and $\mG \in \real^{\nn \times
\nn}$ are a diagonal matrix of thermal capacitance and a symmetric,
positive-definite matrix of thermal conductance, respectively. The vector
$\vq(\t) \in \real^\np$ represents the temperature of the system at time $\t$
while $\vs(\t) \in \real^\nn$ is the system's internal state at that moment. The
vector $\vq_\ambient \in \real^\np$ contains the ambient temperature. The matrix
$\mM \in \real^{\nn \times \np}$ is a mapping that distributes the power
consumption of the processing elements across the thermal nodes; without loss of
generality, $\mM$ is a rectangular diagonal matrix whose diagonal elements are
equal to unity.

As noted with respect to power, given a set of time instances $\{ \t_i \}_{i =
1}^\ns$, \eref{thermal-system} can be utilized to compute a temperature profile
$\mQ \in \real^{\np \times \ns}$ of the system. Another example is the maximal
temperature, which can be estimated as follows:
\[
  \text{Maximal temperature} = \max_{i, \t} \q_i(\t).
\]

Assuming that temperature depends on $\vu$---which is the case whenever, for
instance, the power consumption depends on $\vu$---a temperature profile as well
as the maximal temperature can be considered as quantities of interest $\g$.

\begin{remark}
We do not cover dynamic steady-state temperature analysis \cite{ukhov2012} in
this paper; however, an extension of the proposed techniques to this scenarios
is straightforward.
\end{remark}
