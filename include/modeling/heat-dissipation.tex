Based on the specification of the platform including its thermal package, an
equivalent thermal \up{RC} circuit is constructed \cite{skadron2004}. The
circuit comprises $\nn$ thermal nodes, and its structure depends on the intended
level of granularity, which impacts the resulting accuracy. For clarity, we
assume that each processing element is mapped onto one corresponding node, and
the thermal package is represented as a set of additional nodes.

The thermal dynamics of the system are modeled using the following system of
differential-algebraic equations \cite{ukhov2014, ukhov2012}:
\begin{subnumcases}{\elab{thermal-system}}
  \mC \frac{\d\vs(\t)}{\d\t} + \mG \vs(\t) = \mM \vp(\t) \elab{thermal-system-ode} \\
  \vq(\t) = \mM^T \vs(\t) + \vq_\ambient
\end{subnumcases}
The coefficients $\mC \in \real^{\nn \times \nn}$ and $\mG \in \real^{\nn \times
\nn}$ are a diagonal matrix of thermal capacitance and a symmetric,
positive-definite matrix of thermal conductance, respectively. The vector
$\vq(\t) \in \real^\np$ represents the temperature of the system at time $\t$
while $\vs(\t) \in \real^\nn$ is the system's internal state at that moment. The
vector $\vq_\ambient \in \real^\np$ contains the ambient temperature. The matrix
$\mM \in \real^{\nn \times \np}$ is a mapping that distributes the power
consumption of the processing elements across the thermal nodes; without loss of
generality, $\mM$ is a rectangular diagonal matrix whose diagonal elements are
equal to one.

Given a set of $\ns$ points on the timeline $\{ \t_i \}_{i = 1}^\ns$,
\eref{thermal-system} can be used to compute a temperature profile of the system
as follows:
\begin{equation*}
  \mQ = (\q_i(\t_j))_{i = 1, j = 1}^{\np, \ns} \in \real^{\np \times \ns}.
\end{equation*}
Then the maximum temperature of the system can be estimated using the
temperature profile as follows:
\begin{align}
  \text{Maximum temperature} &= \max_{i = 1}^\np \, \sup_{\t} \, \q_i(\t) \nonumber \\
                             &\approx \max_{i = 1}^\np \max_{j = 1}^\ns \, \q_i(\t_j). \elab{maximum-temperature}
\end{align}

Assuming that temperature depends on $\vu$---which is the case whenever, for
instance, the power consumption depends on $\vu$---a temperature profile as well
as the maximum temperature can be considered as quantities of interest $\g$.
Dynamic steady-state temperature analysis \cite{ukhov2012} can also be
considered in this regards, which is particularly useful for reliability
analysis \cite{ukhov2015}.
