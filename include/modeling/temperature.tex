Based on the specification of the platform including its thermal package, an
equivalent thermal \up{RC} circuit is constructed \cite{skadron2004}. The
circuit comprises $\nn$ thermal nodes, and its structure depends on the intended
level of granularity, which impacts the resulting accuracy. For clarity, we
assume that each processing element is mapped onto one corresponding node, and
the thermal package is represented as a set of additional nodes. As before, let
us remind that the \up{RC} model is an example; other thermal models can be used
with our framework equally well.

The thermal dynamics of the system are modeled using the following system of
differential-algebraic equations \cite{ukhov2014, ukhov2012}:
\begin{subnumcases}{\elab{thermal-system}}
  \mC \frac{\d\vs(\t)}{\d\t} + \mG \vs(\t) = \mM \vp(\t) \\
  \vq(\t) = \mM^T \vs(\t) + \vq_\ambient
\end{subnumcases}
The coefficients $\mC \in \real^{\nn \times \nn}$ and $\mG \in \real^{\nn \times
\nn}$ are a diagonal matrix of thermal capacitance and a symmetric,
positive-definite matrix of thermal conductance, respectively. The vectors
$\vp(\t) \in \real^\np$,  $\vq(\t) \in \real^\np$, and $\vs(\t) \in \real^\nn$
correspond the system's power, temperature, and internal state at time $\t$,
respectively. The vector $\vq_\ambient \in \real^\np$ contains the ambient
temperature. The matrix $\mM \in \real^{\nn \times \np}$ is a mapping that
distributes the power consumption of the processing elements across the thermal
nodes; without loss of generality, $\mM$ is a rectangular diagonal matrix whose
diagonal elements are equal to one.

Given a set of $\ns$ points on the timeline $\{ \t_i \}_{i = 1}^\ns$,
\eref{thermal-system} can be used to compute a temperature profile of the system
as follows:
\begin{equation*}
  \mQ = (\q_i(\t_j))_{i = 1, j = 1}^{\np, \ns} \in \real^{\np \times \ns}.
\end{equation*}
Then the maximum temperature of the system can be estimated using the
temperature profile as follows:
\begin{equation} \elab{maximum-temperature}
  \text{Max temperature} = \max_{i = 1}^\np \, \sup_{\t} \, \q_i(\t) \approx \max_{i = 1}^\np \max_{j = 1}^\ns \, \q_i(\t_j).
\end{equation}

Since the power consumption of the system is affected by $\vu$ (see
\sref{power}), the system's temperature is affected by $\vu$ as well. Therefore,
the temperature in \eref{maximum-temperature} can be considered as a quantity of
interest $\g$. Note that, due to the maximization involved, the quantity is
nondifferentiable and, hence, cannot be adequately addressed using polynomial
approximations, specially taking into account the concern in \rref{smoothness}.
