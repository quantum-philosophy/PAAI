Let the dynamic power consumed by the $j$th task when running on the $i$th
processing element be fixed during the execution of the task and given by
$\p^\dynamic_{ij}$.

\begin{remark}
The boundaries of a task have not been specified, and one can perform modeling
at the level of granularity that makes the most sense for a particular problem.
\end{remark}

Let a time-dependent vector $\vp(\t) = (\p_i(\t))_{i = 1}^{\np}$ capture the
total power consumption of the system at time $\t$. This vector is related to
the dynamic power introduced above as follows:
\begin{equation} \elab{power}
  \p_i(\t) = \sum_{j \in \tasks} \p^\dynamic_{ij} \: \delta_{ij} (\t) + \p^\static_i(\t), \hspace{1em} \text{for $i \in \procs$},
\end{equation}
where $\delta_{ij}(\t)$ is the indicator that the $i$th processing element
executes the $j$th task at time $\t$, and $\p^\static_i(\t)$ is the static power
consumed by the $i$th processing element at time $\t$. In general, the last
component varies over time due to the interdependence between the leakage power
and temperature \cite{liu2007}.

Given a set of points $\{ \t_j \}_{j = 1}^\ns$ on the timeline, \eref{power} can
be used to construct a power profile of the system
\[
  \mP = (\p_i(\t_j))_{i = 1, j = 1}^{i = \np, j = \ns} \in \real^{\np \times \ns},
\]
which is essentially a matrix whose $i$th row captures the evolution of the
power consumption of the $i$th processing element over the $\ns$ time moments.
Another valuable quantity, which can be calculated at this point, is the total
energy consumed by the system during one application run. To this end,
\eref{power} has to be integrated over the time span of the application:
\begin{equation} \elab{total-energy}
  \text{Total energy} = \sum_{i \in \procs} \int \p_i(\t) \, d\t
\end{equation}

Assuming that the power consumption depends on $\vu$---which is the case when,
for instance, $(\vb, \vd)$ depends on $\vu$---a power profile as well as the
total energy are candidates for $\g$.
