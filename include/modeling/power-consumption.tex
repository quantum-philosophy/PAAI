Let the dynamic power consumed by task $j$ when running on processing element
$i$ be fixed during the execution of the task and denote this dynamic power by
$\p^\dynamic_{ij}$.

\begin{remark}
The boundaries of a task have not been specified. The modeling can be performed
at the level of granularity that makes the most sense for a particular problem.
\end{remark}

Let the vector $\vp(\t) = (\p_i(\t))_{i = 1}^{\np}$ capture the total power
consumption of the system at time $\t$. This vector is related to the dynamic
power introduced above as follows:
\begin{equation} \elab{power}
  \p_i(\t) = \sum_{j = 1}^\nt \p^\dynamic_{ij} \: \delta_{ij} (\t) + \p^\static_i(\t), \hspace{1em} \text{for $i = 1, \dots, \np$},
\end{equation}
where $\delta_{ij}(\t)$ is the indicator of the event that processing element
$i$ executes task $j$ at time $\t$, and $\p^\static_i(\t)$ is the static power
consumed by processing element $j$ at time $\t$. In general, the last component
varies with time due to the interdependence between the leakage power and
temperature \cite{liu2007}.

Given a set of $\ns$ points on the timeline $\{ \t_i \}_{i = 1}^\ns$,
\eref{power} can be used to construct a power profile of the system as follows:
\[
  \mP = (\p_i(\t_j))_{i = 1, j = 1}^{\np, \ns} \in \real^{\np \times \ns}.
\]
The above is essentially a matrix where row $i$ captures the power consumed by
processing element $i$ at the $\ns$ time moments.

The total energy consumed by the system during an application run can be
computed by integrating \eref{power} over the time span of the
application---which is demarcated by the minuend and subtrahend in
\eref{end-to-end-delay}---and the corresponding integral can be estimated using
the power profile as follows:
\begin{equation} \elab{total-energy}
  \text{Total energy} = \sum_{i = 1}^\np \int \p_i(\t) \, \d\t \approx \sum_{i = 1}^\np \sum_{j = 1}^\ns \p_i(\t_j) \, \Delta\t_j
\end{equation}
where $\Delta\t_j$ is either $\t_j - \t_{j - 1}$ or $\t_{j + 1} - \t_j$,
depending on how power values are encoded in $\mP$. The assumption that
\eref{total-energy} is based on is that each $\Delta\t_i$ is sufficiently small
so that the power consumed within the interval does not change much.

Assuming that the power consumption depends on $\vu$---which is the case
whenever, for instance, $(\vb, \vd)$ depends on $\vu$---the total energy given
in \eref{total-energy} is a candidate for $\g$.
