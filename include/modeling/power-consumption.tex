Denote the mapping of the application $\tasks$ onto the platform $\procs$ by a
vector $\mapping = (m_i)_{i = 1}^{\nt}$ wherein $m_i \in \procs$ is the index of
the processing element that the $i$th task is assigned to. Let the dynamic power
consumption of the tasks be fixed during their execution and given by a vector
$\vp^\dynamic = (\p^\dynamic_i)_{i = 1}^{\nt}$ wherein $\p^\dynamic_i$ is the
dynamic power of the $i$th task.

\begin{remark}
The boundaries of a task have not been specified, and one can perform modeling
at the level of granularity that makes the most sense for a particular problem.
\end{remark}

Let a time-dependent vector $\vp(\t) = (\p_i(\t))_{i = 1}^{\np}$ capture the
total power consumption of the system at time $\t$. This vector should not be
confused with the vector $\vp^\dynamic$ introduced in the previous paragraph.
The two are related as follows:
\begin{equation} \elab{power}
  \p_i(\t) = \sum_{j = 1}^\nt \p^\dynamic_j \: \delta_{i\,m_j} \: \one_{[\b_j, \b_j + \d_j)}(\t) + \p^\static_i(\t),
\end{equation}
for $i = 1, \dots, \np$, where $\delta_{ij}$ is the Kronecker delta, $\one_A(x)$
is the indicator function of a set $A$, and $\p^\static_i(\t)$ is the static
power consumed at time $\t$. The last one varies over time due to the
interdependence between the leakage power and temperature \cite{liu2007}. In
words, the above equation yields the dynamic power of the task running on the
$i$th processing element at time $\t$ (if any) plus the static power at that
moment.

Given a set of points $\{ \t_i \}_{i = 1}^\ns$ on the timeline, \eref{power} can
be used to construct a power profile $\mP \in \real^{\np \times \ns}$ of the
system, which is essentially a matrix whose $i$th row captures the evolution of
the power consumption of the $i$th processing element over the $\ns$ time
moments. Another valuable quantity, which can be calculated at this point, is
the total energy consumed by the system. To this end, \eref{power} has to be
integrated over the time span of the application:
\begin{align}
  \text{Total energy} &= \sum_{i = 1}^\np \int \p_i(\t) \, d\t \elab{total-energy} \\
                      &= \vd^T \vp^\dynamic + \sum_{i = 1}^\np \int \p^\static_i(\t) \, d\t. \nonumber
\end{align}

Assuming that the power consumption depends on $\vu$---which is the case when,
for instance, $(\vb, \vd)$ depends on $\vu$---a power profile as well as the
total energy are candidates for $\g$.
