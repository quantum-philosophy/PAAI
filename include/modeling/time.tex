Suppose the application is given as a directed acyclic graph. The vertices
represent tasks, and the edges data dependency between these tasks. Suppose
further that a static cyclic scheduling policy is utilized. Note, however, these
assumptions are orthogonal to our framework: the framework can be applied to any
application model and any scheduling policy.

Each task has a start and a finish time. For task $i$, denote these two time
moments by $\b_i$ and $\d_i$, respectively, and let $\vb = (\b_i)_{i=1}^\nt$ and
$\vd = (\d_i)_{i=1}^\nt$. Other timing characteristics of the application can be
derived from $(\vb, \vd)$. An example is the end-to-end delay, which is the
difference between the finish time of the latest task and the start time of the
earliest task:
\begin{equation} \elab{end-to-end-delay}
  \text{End-to-end delay} = \max_{i = 1}^\nt \, \d_i - \min_{i = 1}^\nt \, \b_i.
\end{equation}

Suppose the execution times of the tasks depend on $\vu$ (see \sref{problem}).
Then the tuple $(\vb, \vd)$ depends on $\vu$. Then the end-to-end delay given in
\eref{end-to-end-delay} depends on $\vu$ and is a potential metric $\g$; it is
the one used in \fref{example}. Note that this $\g$ is nondifferentiable as the
$\max$ and $\min$ functions are such. Hence, $\g$ is nonsmooth, which renders
\up{PC} expansions and similar techniques inadequate for this problem, as
illustrated in \sref{introduction}.

\begin{remark} \rlab{smoothness}
In general, the behavior of $\g$ with respect to continuity, differentiability,
and smoothness cannot be inferred from the behavior of $\vu$. Even when the
parameters are perfectly behaved, $\g$ can still and likely will exhibit
nondifferentiability or even discontinuity, which depends on how $\g$ works
internally. For example, as shown in \cite{tanasa2015}, even if execution times
of tasks are continuous, due to the actual scheduling policy, end-to-end delays
are very often discontinuous.
\end{remark}
