Suppose the application is given as a directed acyclic graph. The vertices
correspond to tasks, and the edges to data dependency between the tasks. Each
task has a start and a finish time. For task $i$, denote these two time moments
by $\b_i$ and $\d_i$, respectively, and let $\vb = (\b_i)_{i=1}^\nt$ and $\vd =
(\d_i)_{i=1}^\nt$. Other timing characteristics of the application can be
derived from $(\vb, \vd)$. An example is the end-to-end delay, which, in this
scenario, is the difference between the finish time of the latest task and the
start time of the earliest task:
\begin{equation} \elab{end-to-end-delay}
  \text{End-to-end delay} = \max_{i = 1}^\nt \, \d_i - \min_{i = 1}^\nt \, \b_i.
\end{equation}

Suppose the execution times of the tasks depend on $\vu$ (see \sref{problem}).
Then the end-to-end delay given in \eref{end-to-end-delay} is a potential
quantity of interest $\g$, that is, a potential target for probabilistic
analysis. Note that this quantity is nondifferentiable in general. The
nonsmoothness makes the problem difficult for such techniques as \up{PC}
expansions, as illustrated in \sref{introduction}.
