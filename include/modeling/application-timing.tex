Each task in $\tasks$ has a start time and a finish time. Denote these two time
moments of the $i$th task by $\b_i$ and $\d_i$, respectively, and let $\vb =
(\b_i)_{i=1}^{\nt}$ and $\vd = (\d_i)_{i=1}^{\nt}$. All other timing
characteristics of the system can be derived from the tuple $(\vb, \vd)$. An
example is the end-to-end delay of the application, which is defined as the
difference between the finish time of the last task and the start time of the
first task:
\begin{equation} \elab{end-to-end-delay}
  \text{End-to-end delay} = \max_{i \in \tasks} \, \d_i - \min_{i \in \tasks} \, \b_i.
\end{equation}

Assuming that the timing of the tasks depends on $\vu$, the quantities mentioned
above are potential quantities of interest $\g$, that is, potential targets to
probabilistic analysis.
