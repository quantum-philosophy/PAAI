Each task is ascribed two variables: one is the beginning of the task's
execution time interval, and the other is the duration of this interval. For the
$i$th task, denote the two variables by $\b_i$ and $\d_i$, respectively, and let
$\vb = (\b_i)_{i=1}^{\nt}$ and $\vd = (\d_i)_{i=1}^{\nt}$. The tuple $(\vb,
\vd)$ becomes completely known after the application has been scheduled and
executed.

All other timing characteristics of the system can be found or derived from the
tuple $(\vb, \vd)$. Examples include the completion times of the tasks and the
end-to-end delay of the entire application. The latter, for instance, is
\[
  \text{End-to-end delay} = \max_{i = 1}^\nt \, (\b_i + \d_i).
\]

Any of the quantities mentioned above (and their combinations) can be considered
as the quantity of interest $\g$.
