Each task occupies a certain interval on the timeline. Denote by $\b_i$ and
$\d_i$ the beginning and duration of the execution interval of the $i$th task,
respectively, and let $\vb = (\b_i)_{i=1}^{\nt}$ and $\vd = (\d_i)_{i=1}^{\nt}$.
All other timing characteristics of the system can be derived from the tuple
$(\vb, \vd)$. The examples include the completion times of the tasks and the
end-to-end delay of the application. The latter, for instance, is
\begin{equation} \elab{end-to-end-delay}
  \text{End-to-end delay} = \max_{i = 1}^\nt \, (\b_i + \d_i).
\end{equation}

Assuming that the timing of the tasks depends on $\vu$, the quantities mentioned
above are potential quantities of interest $\g$, that is, potential targets to
probabilistic analysis.
