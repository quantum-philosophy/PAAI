Each task of the application $\tasks = \{ i \}_{i = 1}^\nt$ has a start and a
finish time. Denote these two moments of task $i$ by $\b_i$ (for \emph{begin})
and $\d_i$ (for \emph{done}), respectively, and let $\vb = (\b_i)_{i=1}^\nt$ and
$\vd = (\d_i)_{i=1}^\nt$. All other timing characteristics of the system can be
derived from the tuple $(\vb, \vd)$. An example is the end-to-end delay, which
is defined as the difference between the finish time of the latest task and the
start time of the earliest task:
\begin{equation} \elab{end-to-end-delay}
  \text{End-to-end delay} = \max_{i \in \tasks} \, \d_i - \min_{i \in \tasks} \, \b_i.
\end{equation}

Assuming that the timing of the tasks depends on $\vu$, the end-to-end delay
given in \eref{end-to-end-delay} is a potential quantity of interest $\g$, that
is, a potential target for probabilistic analysis.
