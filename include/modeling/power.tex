Denote the number of processing elements present on the platform by $\np$. Let
the dynamic power consumed by task $j$ when running on processing element $i$ be
fixed during the execution of the task and denote this dynamic power by
$\p^\dynamic_{ij}$.

The fact that $\p^\dynamic_{ij}$ is constant might seem restrictive. However,
one should keep in mind that it is just an example. Our framework does not have
such a restriction. Even in this simple power model, the modeling accuracy can
be improved by representing large tasks as sequences of smaller tasks.

Let the vector $\vp(\t) = (\p_i(\t))_{i = 1}^{\np}$ capture the total power
consumption of the system at time $\t$. This vector is related to the dynamic
power introduced above as follows:
\begin{equation} \elab{power}
  \p_i(\t) = \sum_{j = 1}^\nt \p^\dynamic_{ij} \: \delta_{ij} (\t) + \p^\static_i(\t), \hspace{1em} \text{for $i = 1, \dots, \np$},
\end{equation}
where $\delta_{ij}(\t)$ is an indicator function (outputs either zero or one) of
the event that processing element $i$ executes task $j$ at time $\t$, and
$\p^\static_i(\t)$ is the static power consumed by processing element $j$ at
time $\t$. The last component depends on time because the leakage power and
temperature are interdependent \cite{liu2007}, and temperature changes over time
(see the next subsection).

Given a set of $\ns$ points on the timeline $\{ \t_i \}_{i = 1}^\ns$,
\eref{power} can be used to construct a power profile of the system as follows:
\[
  \mP = (\p_i(\t_j))_{i = 1, j = 1}^{\np, \ns} \in \real^{\np \times \ns}.
\]
The above is essentially a matrix where row $i$ captures the power consumed by
processing element $i$ at the $\ns$ time moments.

The total energy consumed by the system during an application run can be
computed by integrating \eref{power} over the time span of the
application---which is demarcated by the minuend and subtrahend in
\eref{end-to-end-delay}---and the corresponding integral can be estimated using
the power profile as follows:
\begin{equation} \elab{total-energy}
  \text{Total energy} = \sum_{i = 1}^\np \int \p_i(\t) \, \d\t \approx \sum_{i = 1}^\np \sum_{j = 1}^\ns \p_i(\t_j) \, \Delta\t_j
\end{equation}
where $\Delta\t_j$ is either $\t_j - \t_{j - 1}$ or $\t_{j + 1} - \t_j$,
depending on how power values are encoded in $\mP$. The assumption that
\eref{total-energy} is based on is that each $\Delta\t_i$ is sufficiently small
so that the power consumed within the interval does not change significantly.

Since the tuple $(\vb, \vd)$ depends on $\vu$, the power consumption of the
system depends on $\vu$ too. Consequently, the total energy given in
\eref{total-energy} depends on $\vu$ and is a candidate for $\g$.
