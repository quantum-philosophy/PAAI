The careful reader has noted a problem with the calculation of variance in the
previous subsection: $h$ is vector valued. More generally, the metric $\g$ in
\sref{modeling} and the function $\f$ in \sref{interpolation} have been depicted
as having one-dimensional codomains. This, however, has been done only for the
sake of clarity. All the mathematics and pseudocodes stay the same for
vector-valued functions. The only except is that, since a surplus
$\surplus(\vx_{\vi\vj})$ naturally inherits the output dimensionality of $\f$,
the operations that involve $\surplus(\vx_{\vi\vj})$ should be adequately
adjusted. If the outputs are on different scales and/or have different accuracy
requirements, one might want to have different $\aerror$ and $\rerror$ in
\eref{stopping-condition} for different outputs. In that case, one also needs to
device a more sensible strategy for scoring collocation nodes in \eref{score}
such as rescaling individual outputs and then calculating the uniform norm
$\norm{\cdot}_\infty$ or $L^2$ norm $\norm{\cdot}_2$. Our code \cite{sources}
has been written with multiple outputs in mind.
