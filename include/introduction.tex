Computer experiments \cite{santner2003} are a powerful approach to uncertainty
quantification. The most popular strategies include Monte Carlo (\abbr{MC})
sampling, quasi-\abbr{MC} sampling, and Latin hypercube sampling. Compared to
other techniques for probabilistic analysis, computer experiments are
straightforward to undertake. The system at hand is treated as completely
opaque, and one only has to simulate it a sufficient number of times in order to
draw cogent conclusions about the system's behavior.

Consider, for instance, the classical \abbr{MC} sampling, which is arguably the
most famous and versatile approach to the analysis of stochastic systems. The
technique was introduced in the middle of the twentieth century and since then
has expanded into a broad family of methods that have had a tremendous impact
both in academia and in terms of industrial breakthroughs. The success of
\abbr{MC} sampling is due to the ease of implementation, independence of the
stochastic dimensionality, and asymptotic behavior of the quantities estimated
using this approach (the law of large numbers).

The major problem with sampling techniques, however, is in sampling: one should
be able to collect a sufficiently large number of realizations of the quantity
of interest in order to draw sound conclusions with respect to this quantity.
The main concerns in this regard are: How many samples can one afford to
collect? How long does it take to obtain one sample? How much does one sample
cost? When the subject under analysis is expensive---as measured by any metric
that makes the most sense to the problem at hand---computer experiments are
rendered as slow and often infeasible.

The work is based on the work in \cite{klimke2006} and \cite{ma2009}.

The approach belongs to the class of stochastic collocation techniques
\cite{xiu2010}. The major distinctive feature of stochastic collocation is the
usage of interpolation as a means of uncertainty quantification, which should be
contrasted with other techniques such as polynomial chaos expansions relying on
regression. The application of popular polynomial expansions is limited in this
case due to the non-smoothness of the response surface.

The remainder of the paper is organized as follows. Section~\ref{sec:prior-work}
provides an overview of the prior work. In \sref{present-work}, we summarize the
contribution of the present paper. The preliminaries are given in
\sref{preliminaries}. The objective of our study is formulated in
\sref{problem-formulation}. The proposed framework is presented in
\sref{modeling}, \sref{interpolation}, and \sref{analysis}. The experimental
results are reported and discussed in \sref{experimental-results}, and
\sref{conclusion} concludes the paper.
