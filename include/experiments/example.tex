Last but not least, we investigate the viability of deploying the proposed
framework in a real environment. It means that we need to couple the framework
with a battle-proven simulator, which is used in both academia and industry, and
let it simulate a real application running on a real platform. Before we
proceed, we would like to remind that all the implementation and configuration
details including the infrastructure developed for this example can be found at
\cite{sources}.

The scenario that we consider is the same as the one depicted in \fref{example}
except for the fact that an industrial-standard simulator is put in place of the
``black box'' on the left side, and that the metric of interest $\g$ is now the
total energy. Unlike the previous examples, there is no true solution to compare
with due to the prohibitive expense of the simulator, which is exactly why our
framework is needed in such cases.

The simulator of choice is the well-known and widely used combination of Sniper
\cite{carlson2011} and McPAT \cite{li2009}. The architecture that we simulate is
Intel's Nehalem-based Gainestown series. Sniper is distributed with a
configuration file for this architecture, and we use it without any changes. The
platform is configured to have three \up{CPU}s sharing one \up{L3} cache.

The application that has been chosen for simulation is \up{VIPS}, which is an
image-processing piece of software taken from the \up{PARSEC} benchmark suite
\cite{bienia2011}. In this scenario, \up{VIPS} applies a fixed set of operations
to a given image. The width and height of the image to process are considered as
the uncertain parameters $\vu$ (see \sref{problem}), which are assumed to be
distributed uniformly within certain ranges.

The real-life deployment has fulfilled our expectations. The interpolation
process successfully finished and delivered a surrogate after 78 invocations of
the simulator. Each invocation took 40 minutes on average. The probability
distribution of the total energy was then estimated by sampling the constructed
surrogate $10^5$ times. These many samples would take months to obtain if one
sampled the simulator directly. Using the proposed technique, the whole
procedure took several hours.
