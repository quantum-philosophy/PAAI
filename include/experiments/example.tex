Last but not least, we investigate the viability of deploying the framework in a
real environment. It means that we need to couple the framework with a
battle-proven simulator used in academia and industry and let it simulate a real
application running on a real platform. The scenario we consider is the same as
the one depicted in \fref{example} except for the fact that an
industrial-standard simulator is put in place of the leftmost box, and the
metric of interest $\g$ is now the total energy. Unlike the previous examples,
there is no true solution to compare with due to the prohibitive expense of the
simulator, which is exactly why our framework is needed in such cases.

The simulator of choice is the widely used combination of Sniper
\cite{carlson2011} and McPAT \cite{li2009}. The simulated architecture is
Intel's Nehalem-based Gainestown series. The platform is configured to have
three \up{CPU}s sharing one \up{L3} cache. The simulated application is
\up{VIPS}, which is an image-processing piece of software taken from \up{PARSEC}
\cite{bienia2011}. In this scenario, \up{VIPS} applies a fixed set of operations
to a given image, and the width and height of the image to process are the
uncertain parameters $\vu$, which are assumed to be distributed uniformly within
certain ranges. All other details including the infrastructure developed for
this example can be found at \cite{sources}.

The real-life deployment has fulfilled our expectations. The interpolation
process successfully finished and delivered a surrogate after 78 invocations of
the simulator. Each invocation took 40 minutes on average. The probability
distribution of the total energy was then estimated by sampling the constructed
surrogate $10^5$ times. \updated{These many samples would take months to obtain
if one sampled the simulator directly.} Using the proposed technique, the whole
procedure took several hours.
