\begin{reviewer}
The paper presents an adaptive interpolation approach for estimating the
distribution of a quantity of interest g that depends on uncertain parameters u.
The paper is motivated by the fact that electronic systems often  exhibit
uncertain run-time behaviors due to variation in both the physical and cyber
world. The proposed approach treats the system as a black box, and estimates the
probability distribution g(u) by first reducing g(u) to a surrogate g(z) where z
is extracted from u and then evaluating g(z) using using interpolation on a set
of adaptively chosen sample points.

\clabel{3}{1}
Overall, the paper is well organized. The English is colloquial at times and
should be improved in the revised version. I am not an expert in probability
distribution estimation so I cannot comment on the novelty of the estimation
approach. However, the motivation and results of the proposed approach are
convincing. Compared to existing methods, the approach proposed addresses
problems such as nondifferentiability and discontinuity in the dependence on the
uncertain parameters, and also requires lesser number of samples to reach a
certain degree of accuracy for estimating the probability distribution of the
quantity of interest.
\end{reviewer}

\begin{authors}
We agree.

\begin{actions}
  \action{Well done.}
\end{actions}
\end{authors}

\begin{reviewer}
\clabel{3}{2}
There are several clarification I would like the authors to make in the
revision. First, I think the claim of treating the system as a blackbox is
somewhat misleading. In particular, the approach needs to use the dependency
among the uncertain parameters to construct the surrogate g(z). This is also
evident in the example after Remark 1 where the correlation matrix is assumed to
be given. This can be unrealistic and I would like to understand how this
restriction impacts the proposed approach. Another problem I have is with Figure
5. Why is the initial error (at 0 sample) lower for blue than red? In fact, why
is it non-zero for the proposed approach?
\end{reviewer}

\begin{authors}
We agree.

\begin{actions}
  \action{Well done.}
\end{actions}
\end{authors}
