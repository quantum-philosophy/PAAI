\begin{reviewer}
The paper presents an adaptive interpolation approach for estimating the
distribution of a quantity of interest g that depends on uncertain parameters u.
The paper is motivated by the fact that electronic systems often  exhibit
uncertain run-time behaviors due to variation in both the physical and cyber
world. The proposed approach treats the system as a black box, and estimates the
probability distribution g(u) by first reducing g(u) to a surrogate g(z) where z
is extracted from u and then evaluating g(z) using using interpolation on a set
of adaptively chosen sample points.

\clabel{3}{1}
Overall, the paper is well organized. The English is colloquial at times and
should be improved in the revised version.
\end{reviewer}

\begin{authors}
The reviewer has not specified the expressions whose usage is not appropriate;
however, based on our own judgment, we have tried to identify and eliminate or
replace the expressions that might be considered colloquial. For example, the
following words and phrases have been revisited:

\begin{itemize}
  \item \emph{broad umbrella} (Sec.~I)
  \item \emph{How much does it cost \dots?} (Sec.~I)
  \item \emph{How many \dots do we need?} (Sec.~I)
  \item \emph{How many can we afford?} (Sec.~I)
  \item \emph{figure it out} (Sec.~I)
  \item \emph{let alone} (Sec.~I)
  \item \emph{switching gears} (Sec.~II)
  \item \emph{quandary} (Sec.~VI)
  \item \emph{backed in} (Sec.~VIII-A)
  \item \emph{facets} (Sec.~VIII)
  \item \emph{starts to shine} (Sec.~IX-B)
  \item \emph{comes in two flavors} (Sec.~IX-C)
  \item \emph{go hand in hand} (Sec.~IX-D)
  \item \emph{thanks to} (Sec.~IX-E)
  \item \emph{get to grips with} (Sec.~XI)
  \item \emph{vanilla} (Sec.~XI-A)
  \item \emph{walk through} (Sec.~XI-B)
\end{itemize}

See also \cref{0}{1}.

\begin{actions}
  \action{The language has been made more formal and scientific.}
\end{actions}
\end{authors}

\begin{reviewer}
I am not an expert in probability distribution estimation so I cannot comment on
the novelty of the estimation approach. However, the motivation and results of
the proposed approach are convincing. Compared to existing methods, the approach
proposed addresses problems such as nondifferentiability and discontinuity in
the dependence on the uncertain parameters, and also requires lesser number of
samples to reach a certain degree of accuracy for estimating the probability
distribution of the quantity of interest.

There are several clarification I would like the authors to make in the
revision.

\clabel{3}{2}
First, I think the claim of treating the system as a blackbox is somewhat
misleading. In particular, the approach needs to use the dependency among the
uncertain parameters to construct the surrogate g(z). This is also evident in
the example after Remark~1 where the correlation matrix is assumed to be given.
This can be unrealistic and I would like to understand how this restriction
impacts the proposed approach.
\end{reviewer}

\begin{authors}
The proposed framework has two main inputs: a specification of the probability
distribution of the uncertain parameters and a means of evaluating the desired
metric for a given assignment of the parameters. What is referred to as a
``black box'' is the second input only, that is, the simulator of the metric.
The reviewer is absolutely right that the probability distribution of the
uncertain parameters needs to be given explicitly in order to construct a
surrogate. We have addressed this concern by clarifying the problem formulation
given in Sec.~V. We have also adjusted the figure of the illustrative example
given in Sec.~VII, Fig.~2, so that the ``black-box'' component is apparent.

Lastly, we would like to note that the way the information about the probability
distribution of the uncertain parameters is obtained is outside of the scope of
this work. However, since we are well aware that the knowledge of the join
distribution of the parameters is a rather impractical assumption to make, the
framework has been tailored to work with an incomplete knowledge of the
distribution, which we elaborate on in Remark~1 made in Sec.~VIII-A.

\begin{actions}
  \action{The problem formulation in Sec.~V has been revisited.}

  \action{The part that is treated as a ``black box'' inside the proposed
  framework has been emphasized in Fig.~2 both graphically and textually.}
\end{actions}
\end{authors}

\begin{reviewer}
\clabel{3}{3}
Another problem I have is with Figure~5. Why is the initial error (at 0 sample)
lower for blue than red? In fact, why is it non-zero for the proposed approach?
\end{reviewer}

\begin{authors}
The leftmost points in the graphs given in Fig.~5 correspond to one sample, not
to zero samples. Due to the scaling, it was indeed a source of confusion. In the
revised version of the figure, the zeros on the horizontal axes have been
replaced with ones.

Let us now turn to the observation that the blue curve (our approach) is lower
than the red one (direct sampling) at the starting point where the number of
samples equals one. In the case of direct sample, the distribution is estimated
based on one sample. In the case of our technique, the distribution is estimated
based on $10^5$ drawn from the corresponding surrogate. However, since a
surrogate with one collocation node is a constant (see $e_{00}$ in Fig.~4), all
these samples are the same. The difference between the two approaches is then in
the positions of their samples.

It can be seen in (26) that the first collocation node of our technique is 0.5,
which is the center of the unit interval. Due to the transformation $\mathbb{T}$
described in Sec.~VIII-A, the interpolation takes place in a uniform probability
space. Consequently, point 0.5 is the median (the probability masses to the left
and right from the point are equal) of the distribution before and after the
transformation. We suppose that, in our setting, this point is a better
representative of the distribution than the first point of the Sobol sequence
underpinning direct sampling. The Sobol sequence starts from a noncentral
location due to the nonzero seed (scramble) set in the experiments, and this
point is unlikely to have a special meaning for the distribution of the
uncertain parameters.

The main purpose of Fig.~5 is to show the accuracy trends of the proposed
technique and those of direct sampling. In our opinion, the figure serves this
purpose sufficiently well.

\begin{actions}
  \action{The starting points of the graphs given in Figure~5 have been
  clarified by adjusting the leftmost labels on the horizontal axes.}
\end{actions}
\end{authors}
