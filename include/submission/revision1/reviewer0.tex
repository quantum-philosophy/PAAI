We have identified several common concerns among the reviewers, and we would
like to address them in this section. It might not be applicable to all the
reviewers to its full extent; however, we encourage all the reviewers to read
this section before they proceed to their individual sections. In order to ease
future referencing of this common section, we have titled and structured it in
such a way that it appears to belonged to one additional reviewer, whom we refer
to as Reviewer~0.

\vspace{2em}

\begin{reviewer}
\clabel{0}{1} The paper is difficult to read.
\end{reviewer}

\begin{authors}
In order to make the paper more readable, a number of changes have been made.

First of all, the language has been made more straightforward throughout the
paper. Certain sentences and paragraphs have been rephrased in order to make
their messages easier to comprehend. We have also made an adjustment to the
terminology that we used: \emph{quantity}, as in \emph{quantity of interest},
has been replaced by \emph{metric}, which is considered more appropriate for
this paper; see \cref{1}{3}. In addition, several colloquial expressions have
been eliminated; see \cref{3}{1}.

Second, the illustrative example that was originally given at the end of
the section titled ``Modeling,'' has been moved higher up in the flow of the
paper and resides now in a separate section titled ``Example.'' We expect this
change to have a considerable positive impact on the paper since it builds up
the reader's intuition much earlier, prior to any in-depth discussions. The
corresponding figure, Fig.~2, has been also referenced better from the main text
so that the reader is encouraged to recall this concrete example when it is
likely to be of great help for understanding.

\begin{actions}
  \action{The language has been adjusted throughout the paper.}

  \action{The illustrative example has been moved higher up (Sec.~VII), and the
  corresponding figure (Fig.~2) has been referenced better from the subsequent
  sections.}
\end{actions}
\end{authors}

\begin{reviewer}
\clabel{0}{2} The paper lacks a real-life example.
\end{reviewer}

\begin{authors}
We agree that a real-life example is needed. As it is emphasized in the paper,
our work is concerned with developing a design-time system-level framework. We
target the scenarios wherein the designer resorts to various simulators in order
to investigate the potential runtime behavior of the system under development.
Our goal is to reduce the cost associated with running those simulators, which
are extremely time consuming. Therefore, in the scope of this work, a real-life
example means that we couple our framework with an industrial-standard simulator
and let the simulator evaluate a real application running on a real platform, as
orchestrated by the framework. This is exactly what we have added to the revised
manuscript.

The simulator that we consider is Sniper [Carlson,~2011], which is widely used
in both academia and industry. The simulator is based on the Pin tool by
Intel.\footnote{\texttt{https://software.intel.com/en-us/articles/pintool}} It
aims at systems based on the IA-32 and x86-64 instruction sets and has been
validated against Intel Core and Nehalem.

The exact architecture used in our experiments is Intel's Nehalem-based
Gainestown series; Sniper is shipped with a configuration file (see
\texttt{config/gainestown.cfg}) for this architecture, and we use it without any
changes. The simulated platform is set up to have three CPUs sharing one L3
cache; other details can be found in the above configuration file. Three CPUs
are the minimum requirement imposed by the application that we would like to
simulate (to be discussed shortly).

We have decided to perform uncertainty quantification of the total energy
consumed by the system. In other words, this energy is our metric of interest.
Consequently, we also need a power simulation in order to calculate the power
consumed by the system over time. This job is delegated to McPAT by HP
[Li,~2009], which is another battle-proven and well-known tool.

Regarding a real application, we have decided to choose one from the popular
PARSEC benchmark suite by Princeton University [Bienia,~2011]. PARSEC contains
around a dozen of benchmarks, and the one that we consider is
VIPS,\footnote{\texttt{http://www.vips.ecs.soton.ac.uk}} which is an
image-processing library and command-line tool. VIPS is asked to perform a fix
set of complex operations on a given image. The scenario that one might want to
keep in mind is that the application is provided as a service, and this service
performs a set of standard enhancements on the images submitted by the users.

The width and height of the image to process are considered as uncertain
parameters. We do not assume any prior knowledge of these parameters except that
they lie within certain ranges, and, therefore, the parameters are assumed to be
uniformly distributed. Given this information, our objective is to find the
distribution of the total energy, which we approach using our framework.

Sniper is an accurate simulator, and it also has good performance compared to
other simulators. However, running the simulator is still time consuming, which
is exacerbated even further by the computations undertaken by McPAT.
Consequently, sampling techniques as a means of uncertainty quantification are
prohibitively expensive. As a result, we no longer have the ability to obtain
the true solution, which we could potentially use for assessing the accuracy of
the proposed framework. In the case of the synthetic examples discussed in
Sec.~XI, the true solutions are obtained by extensively sampling our light
in-house simulator; each solution is estimated based on a Sobol sequence with
$10^5$ points. In the case of the real-life example, obtaining such a reliable
reference solution would take weeks if not months as a single simulation takes
40 minutes on average.

This state of affairs is exactly the motivation behind our work. As we emphasize
in the paper, if it was computationally cheap to run such simulators as Sniper
an arbitrary number of times, one would not needed any auxiliary framework such
as ours; direct Monte Carlo sampling and alike would suffice. However,
simulations are expensive. Our framework allows one to construct a surrogate for
the system under consideration by invoking the simulator at a few points chosen
adaptively. One can then proceed to studying the surrogate instead of the
original system, which will be very efficient. It should also be mentioned that
the chosen real-life example is relatively small. It is likely that the
framework will be applied to larger systems, in which case the inadequacy of
direct sampling and the benefit of using the proposed framework will be much
more apparent.

The result of the real-life example is as follows. The construction of a
surrogate finished after 78 invocations of the simulator (the accuracy criteria
were similar to those given in Sec.~XI).

The additional infrastructure written to facilitate the real-life example
described above is open source and available online, as is the rest of our code.
We have also included a virtual machine for better reproducibility of the
results reported in the paper.

\begin{actions}
  \action{A real-life example has been added to the paper, Sec.~XI-C.}

  \action{The infrastructure underpinning the real-life example has been
  included in the publicly accessible supplementary materials.}
\end{actions}
\end{authors}
