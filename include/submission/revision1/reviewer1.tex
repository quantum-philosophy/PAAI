\begin{reviewer}
The article proposes a system-level framework for early estimation and analysis
of an electronic system given a task graph and a heterogeneous platform. The
framework proposes a general approach which is based on hybrid adaptivity on a
sparse grid  for early estimation for the different quality of interest (e.e
power, delay).

Here are my comments to the author:

\clabel{1}{1}
The most problem of this article is the style of presentation. it seems that the
article tries to make the problem and approach more complicated than what it is.
The style of writing in particular approach and implementation is very confusing
and completed. It makes the paper very difficult to follow. I do like the
motivation example in the introduction, however, the rest of paper significantly
lack the real examples and illustrative figures. My suggestion is to build a
story with a real example of a task graph and a platform when describing the
approach and implementation.
\end{reviewer}

\begin{authors}
In order to make the paper more readable, a number of changes have been made.
First, the language has been adjusted throughout the paper in order to make it
more straightforward; see also \cref{1}{3}. Second, the illustrative example
that was originally given at the end of Sec.~VII, ``Modeling,'' has been moved
higher up and resides now in a separate section titled ``Example.'' We expect
this change to have a significant positive impact since it builds up the
reader's intuition much earlier, prior to any in-depth discussions. The
corresponding figure has been also referenced more extensively so that the
reader has always in mind a concrete example to relate to.

The aforementioned example is synthetic; however, its synthetic nature allows us
to make the example easy to follow and yet complex enough to ensure that the
example diligently serves its intended purpose. Specifically, the example gives
the reader the big picture of the proposed framework. It is a road map that the
reader can always refer to whenever the reader needs a clarification about how
certain pieces of the framework fit together. We believe that it is the primary
concern of the reviewer in this comment, and giving the example more attention
in the revised version of the paper addresses the concern sufficiently well; see
also \cref{1}{2}.

\begin{actions}
  \action{The language has been adjusted throughout the paper.}
  \action{The illustrative example has been moved higher up, and the
  corresponding figure has been better referenced from the subsequent sections.}
\end{actions}
\end{authors}

\begin{reviewer}
\clabel{1}{2}
The second major problem of the article is the experimental results. It claimed
in the article, the approach can be applied to different platforms with
different heterogeneous processors, the experimental results stay as a random
statistical model of platform and task graphs with no insight correlated to the
challenges of a real platform. I expect authors can apply their approach their
to measure a metric when an abstract but real application mapped to an abstract
model of a real platform.
\end{reviewer}

\begin{authors}
We agree that a real-life example is needed. As it is emphasized in the paper,
our work is concerned with developing a design-time system-level framework. We
target the scenarios wherein the designer resorts to various simulators in order
to investigate the potential runtime behavior of the system under development.
Our goal is to reduce the cost associated with running those simulators, which
are extremely time consuming. Therefore, in the scope of this work, a real-life
example means that we couple our framework with an industrial-standard simulator
and let the simulator evaluate a real application running on a real platform, as
orchestrated by the framework. This is exactly what we have added to the revised
manuscript.

The simulator that we consider is Sniper [Carlson, 2011], which is widely used
in both academia and industry. The simulator is based on the Pin tool by
Intel.\footnote{\texttt{https://software.intel.com/en-us/articles/pintool}} It
aims at systems based on the IA-32 and x86-64 instruction sets and has been
validated against Intel Core and Nehalem. The exact architecture used in our
experiments is Intel's Nehalem-based Gainestown series; Sniper is shipped with a
configuration file for this architecture, and we use it without any changes.

We have decided to perform uncertainty quantification of the total energy
consumed by the system at hand; in other words, this energy is our metric of
interest. To this end, we also need a power simulation in order to calculate the
power consumed by the system over time. This job is delegated to McPAT by
HP,\footnote{\texttt{http://www.hpl.hp.com/research/mcpat}} which is another
battle-proven and well-known tool.

Regarding a real application, we have decided to pick one from the popular
PARSEC benchmark suite by Princeton University [Bienia, 2011]; Sniper provides a
smooth integration with this suite. PARSEC contains around a dozen of
benchmarks, and the one that we consider is
VIPS,\footnote{\texttt{http://www.vips.ecs.soton.ac.uk}} which is an
image-processing library and command-line tool. In the scope of the benchmark,
VIPS is asked to perform a fix set of complex operations on a given image.

The scenario that one might consider is that the application, VIPS, is provided
as a service, and this service performs a set of standard enhancements on the
images submitted by the users. The width and height of the image to process are
considered as uncertain parameters. We do not assume any prior knowledge about
these parameters except that they lie within certain ranges. Thus, the
parameters are assumed to be uniformly distributed. Given this information, at
it has already been mentioned, our objective is to find the distribution of the
total energy.

\begin{actions}
  \action{A real-life example has been added to the paper.}
\end{actions}
\end{authors}

\begin{reviewer}
\clabel{1}{3}
Wording suggestion, please replace the ``quality of interest'' with ``metric''.
It is more understandable and less confusing.
\end{reviewer}

\begin{authors}
The reason we chose the term \emph{quantity of interest} is that it is the one
frequently used in the general literature on uncertainty quantification.
However, we agree with the reviewer that, in this particular context, it might
not be that common and, therefore, might lead to an unnecessary increase in
cognitive load; \emph{metric} is a better choice.

\begin{actions}
  \action{The term \emph{quantity of interest} has been replaced by
  \emph{metric}.}
\end{actions}
\end{authors}

\begin{reviewer}
\clabel{1}{4}
It is mentioned in the article that the implementation is  based on [4],
[5],[6]. I like to see the difference between the proposed implementation and
the ones presented in [4][5][6]. There is no comparison in the related work.
They all are based on hybrid adaptivity on a sparse grid.
\end{reviewer}

\begin{authors}
We agree.

\begin{actions}
  \action{Well done.}
\end{actions}
\end{authors}

\begin{reviewer}
\clabel{1}{5}
Authors see the heterogeneous platform as a heterogeneous processor elements
with behavior modeled as Gaussian distribution. Real platforms are the
combination of processors as well as communication fabric and shared memory with
run-time and data-dependent arbitration. It is not clear how the approach can
capture the effect of communication and shared memory when estimating a certain
metric.
\end{reviewer}

\begin{authors}
We agree.

\begin{actions}
  \action{Well done.}
\end{actions}
\end{authors}
