\begin{reviewer}
The article proposes a system-level framework for early estimation and analysis
of an electronic system given a task graph and a heterogeneous platform. The
framework proposes a general approach which is based on hybrid adaptivity on a
sparse grid  for early estimation for the different quality of interest (e.e
power, delay).

Here are my comments to the author:

\clabel{1}{1}
The most problem of this article is the style of presentation. it seems that the
article tries to make the problem and approach more complicated than what it is.
The style of writing in particular approach and implementation is very confusing
and completed. It makes the paper very difficult to follow. I do like the
motivation example in the introduction, however, the rest of paper significantly
lack the real examples and illustrative figures. My suggestion is to build a
story with a real example of a task graph and a platform when describing the
approach and implementation.
\end{reviewer}

\begin{authors}
In order to make the paper more readable, a number of changes have been made.
First, the language has been adjusted throughout the paper in order to make it
more intelligible; see also \cref{1}{3}. Second, the illustrative example that
was originally given at the end of Sec.~VII, ``Modeling,'' has been moved higher
up and resides now in a separate section titled ``Example.'' We expect this
change to have a positive impact since it builds up the reader's intuition much
earlier, prior to any in-depth discussions. The corresponding figure has been
also referenced to more extensively so that the reader always keeps in mind and
relates to a concrete example.

The aforementioned example is synthetic; however, its synthetic nature allows us
to make the example as simple as possible and yet complex enough to ensure that
the example serves well its intended purpose. Namely, the example gives the
reader the big picture of the proposed framework. It is a road map that the
reader can always refer to whenever the reader feels lost or unsure about how
certain pieces of the framework fit together. We believe it is the primary
concern of the reviewer in this comment, and, by giving the example more
attention in the revised version of the paper, we address the concern
sufficiently well.

\begin{actions}
  \action{The language has been adjusted throughout the paper.}
  \action{The illustrative example has been moved higher up, and the
  corresponding figure has been better referenced from the main text.}
\end{actions}
\end{authors}

\begin{reviewer}
\clabel{1}{2}
The second major problem of the article is the experimental results. It claimed
in the article, the approach can be applied to different platforms with
different heterogeneous processors, the experimental results stay as a random
statistical model of platform and task graphs with no insight correlated to the
challenges of a real platform. I expect authors can apply their approach their
to measure a metric when an abstract but real application mapped to an abstract
model of a real platform.
\end{reviewer}

\begin{authors}
We agree.

\begin{actions}
  \action{Well done.}
\end{actions}
\end{authors}

\begin{reviewer}
\clabel{1}{3}
Wording suggestion, please replace the ``quality of interest'' with ``metric''.
It is more understandable and less confusing.
\end{reviewer}

\begin{authors}
The reason we chose the term \emph{quantity of interest} is that it is the one
used in the literature on uncertainty quantification. However, we agree with the
reviewer that, in this particular context, it might not well known and, hence,
might lead to unnecessary confusion; \emph{metric} is a better choice.

\begin{actions}
  \action{The term \emph{quantity of interest} has been replaced by
  \emph{metric}.}
\end{actions}
\end{authors}

\begin{reviewer}
\clabel{1}{4}
It is mentioned in the article that the implementation is  based on [4],
[5],[6]. I like to see the difference between the proposed implementation and
the ones presented in [4][5][6]. There is no comparison in the related work.
They all are based on hybrid adaptivity on a sparse grid.
\end{reviewer}

\begin{authors}
We agree.

\begin{actions}
  \action{Well done.}
\end{actions}
\end{authors}

\begin{reviewer}
\clabel{1}{5}
Authors see the heterogeneous platform as a heterogeneous processor elements
with behavior modeled as Gaussian distribution. Real platforms are the
combination of processors as well as communication fabric and shared memory with
run-time and data-dependent arbitration. It is not clear how the approach can
capture the effect of communication and shared memory when estimating a certain
metric.
\end{reviewer}

\begin{authors}
We agree.

\begin{actions}
  \action{Well done.}
\end{actions}
\end{authors}
