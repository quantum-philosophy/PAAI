\begin{reviewer}
The article proposes a system-level framework for early estimation and analysis
of an electronic system given a task graph and a heterogeneous platform. The
framework proposes a general approach which is based on hybrid adaptivity on a
sparse grid  for early estimation for the different quality of interest (e.e
power, delay).

Here are my comments to the author:

\clabel{1}{1}
The most problem of this article is the style of presentation. it seems that the
article tries to make the problem and approach more complicated than what it is.
The style of writing in particular approach and implementation is very confusing
and completed. It makes the paper very difficult to follow. I do like the
motivation example in the introduction, however, the rest of paper significantly
lack the real examples and illustrative figures. My suggestion is to build a
story with a real example of a task graph and a platform when describing the
approach and implementation.
\end{reviewer}

\begin{authors}
Please refer to \cref{0}{1} for our response.

Regarding the last sentence of this comment, the example mentioned in
\cref{0}{1} is synthetic; however, its synthetic nature allows us to make the
corresponding story with a platform and a task graph simple to follow and yet
sufficiently complex to ensure that the example diligently serves its intended
purpose, which is to give the reader the big picture of the proposed framework.
It is a road map that the reader can always refer to whenever the reader needs a
clarification or confirmation about how certain parts of the framework. We
suppose that it is the primary concern of the reviewer here, and giving the
example more attention in the revised version of the paper addresses the concern
sufficiently well.

\begin{actions}
  \action{Please refer to \cref{0}{1} for our actions.}
\end{actions}
\end{authors}

\begin{reviewer}
\clabel{1}{2}
The second major problem of the article is the experimental results. It claimed
in the article, the approach can be applied to different platforms with
different heterogeneous processors, the experimental results stay as a random
statistical model of platform and task graphs with no insight correlated to the
challenges of a real platform. I expect authors can apply their approach their
to measure a metric when an abstract but real application mapped to an abstract
model of a real platform.
\end{reviewer}

\begin{authors}
Please refer to \cref{0}{2} for our response.

\begin{actions}
  \action{Please refer to \cref{0}{2} for our actions.}
\end{actions}
\end{authors}

\begin{reviewer}
\clabel{1}{3}
Wording suggestion, please replace the ``quality of interest'' with ``metric''.
It is more understandable and less confusing.
\end{reviewer}

\begin{authors}
The reason we chose the term \emph{quantity of interest} (and its shortcut
\emph{quantity}) is that it is the one frequently used in the general literature
on uncertainty quantification. However, we agree with the reviewer that, in this
particular context, it might not be that common and, therefore, might lead to an
unnecessary increase in cognitive load. The usage of \emph{metric} in place of
\emph{quantity} is more appropriate and less confusing. In the revised version,
we still occasionally use the \emph{of interest} ending in order to add weight
to the term and make it stand out when it is needed.

\begin{actions}
  \action{The term \emph{quantity of interest} and its occasional shortcut
  \emph{quantity} have been replaced by the term \emph{metric of interest} or
  simply by \emph{metric}.}
\end{actions}
\end{authors}

\begin{reviewer}
\clabel{1}{4}
It is mentioned in the article that the implementation is  based on [4],
[5],[6]. I like to see the difference between the proposed implementation and
the ones presented in [4][5][6]. There is no comparison in the related work.
They all are based on hybrid adaptivity on a sparse grid.
\end{reviewer}

\begin{authors}
The framework relies on the mathematics developed in [Klimke,~2006], [Ma,~2009],
and [Jakeman,~2012]. We do not alter this mathematics; we use it for our
purposes. These three bodies of research are rather theoretical, and they cover
different aspects of what we present in the paper.

\begin{actions}
  \action{Well done.}
\end{actions}
\end{authors}

\begin{reviewer}
\clabel{1}{5}
Authors see the heterogeneous platform as a heterogeneous processor elements
with behavior modeled as Gaussian distribution. Real platforms are the
combination of processors as well as communication fabric and shared memory with
run-time and data-dependent arbitration. It is not clear how the approach can
capture the effect of communication and shared memory when estimating a certain
metric.
\end{reviewer}

\begin{authors}
We agree.

\begin{actions}
  \action{Well done.}
\end{actions}
\end{authors}
