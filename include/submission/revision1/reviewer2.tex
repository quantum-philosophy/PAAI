\begin{reviewer}
\clabel{2}{1}
This paper proposes a probabilistic analysis framework that in principle should
enable very rapid sensitivity analysis of complex processing systems, under
various uncertainties. While the idea seems interesting that one framework is
generalized to handle any source of uncertainty to study any metric (as the
authors claim), I find that hard to be realized without heavy manual work. This
is also mentioned that modeling new metric is a manual work, and it is not clear
whether it will be an easy task to integrate it to this framework. This is
already a major challenge to the actual work.
\end{reviewer}

\begin{authors}
We agree.

\begin{actions}
  \action{Well done.}
\end{actions}
\end{authors}

\begin{reviewer}
In addition, I have the following major concerns:

\clabel{2}{2}
The paper is hard to follow. Clear examples include Sec VII. A, Sec VIII, and
Sec. IX. I could not smoothly follow the flow in those sections, sometimes I
could not figure out the purpose of those sections. For example, what is meant
by the level of Smolyak's interpolation, and how is it useful? What is the
Newton-Cotes rule, and how is it useful to this work? There is a very good
chance that reader may not know those rules/algorithms. Maybe auxiliary material
can be added to explain those aspects in a very simple way so readers can follow
your work.
\end{reviewer}

\begin{authors}
Please refer to \cref{0}{1} for our response.

\begin{actions}
  \action{Please refer to \cref{0}{1} for our actions.}
\end{actions}
\end{authors}

\begin{reviewer}
\clabel{2}{3}
It is mentioned that you follow the black-box approach to analyze the
uncertainty. To this, I do not quite follow how can I analyze the uncertainty
due low-level variabilities such as process variation and degradation of
transistors. Also, what will happen if wanted to analyze the impact on the
lifetime of the system? The selection of your case studies cannot reflect that,
and it is not mentioned how can someone reuse this framework for those
scenarios.
\end{reviewer}

\begin{authors}
We agree.

\begin{actions}
  \action{Well done.}
\end{actions}
\end{authors}

\begin{reviewer}
\clabel{2}{4}
There is no mentioning on the accuracy of this overall approach or the
validation against real systems or using actual application traces, while this
can be very easy. Simulation of various processing systems can be exhaustive,
but variation already exists among various of-the-shelf computing products. One
can run some applications on those platforms, and even vary the input data size
to see the final outcome (energy, performance,...). Verification against
experimental works will do two things: 1) solidifies your findings
significantly, 2) gives insight on the accuracy of the framework (do you have
accurate metrics from your framework and experiments, or you observe the same
trend when examining various scenarios?).
\end{reviewer}

\begin{authors}
Please refer to \cref{0}{2} for our response.

\begin{actions}
  \action{Please refer to \cref{0}{2} for our actions.}
\end{actions}
\end{authors}
