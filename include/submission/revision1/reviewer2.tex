\begin{reviewer}
\clabel{2}{1}
This paper proposes a probabilistic analysis framework that in principle should
enable very rapid sensitivity analysis of complex processing systems, under
various uncertainties. While the idea seems interesting that one framework is
generalized to handle any source of uncertainty to study any metric (as the
authors claim), I find that hard to be realized without heavy manual work. This
is also mentioned that modeling new metric is a manual work, and it is not clear
whether it will be an easy task to integrate it to this framework. This is
already a major challenge to the actual work.
\end{reviewer}

\begin{authors}
We hope that the real-life example, which has been added to the paper (see
\cref{0}{2}), responds well to this concern. Specifically, in the revised
version of the paper, there is a new section, Sec.~XI-C, that elaborates on the
deployment of the proposed framework in a real environment. The main challenge
for us was installing and configuring the simulator, and this issue is
orthogonal to the proposed framework. Having done this work, that is, having a
working simulator of the metric, the actual integration with the framework is
straightforward based on our experience.

The steps that we have taken in order to integrate our framework with a
third-party simulator and apply it to a real-life problem can be found online;
this part is open source as is the rest of our code. In the supplementary
materials, we have also included configuration scripts that take a clean machine
to a fully functioning simulation environment needed for the example.

\begin{actions}
  \action{Please refer to \cref{0}{2} for our actions.}
\end{actions}
\end{authors}

\begin{reviewer}
In addition, I have the following major concerns:

\clabel{2}{2}
The paper is hard to follow. Clear examples include Sec VII. A, Sec VIII, and
Sec. IX. I could not smoothly follow the flow in those sections, sometimes I
could not figure out the purpose of those sections. For example, what is meant
by the level of Smolyak's interpolation, and how is it useful? What is the
Newton-Cotes rule, and how is it useful to this work? There is a very good
chance that reader may not know those rules/algorithms. Maybe auxiliary material
can be added to explain those aspects in a very simple way so readers can follow
your work.
\end{reviewer}

\begin{authors}
Please refer to \cref{0}{1} for our response.

Let us now turn to the term \emph{Smolyak level}. The meaning of the term is
rather mechanical. The interpolation presented in Sec.~IX-B is a step-by-step
process. The more steps are taken, the more refined the corresponding
approximation is. The Smolyak level is the index of the interpolation step
starting from zero. Therefore, the Smolyak level can be used to tell how
sophisticated the corresponding surrogate is in terms of the underlying
interpolation grid. We have realized that we did not put enough stress on the
term when we introduced it in Sec.~IX-B in relation to (17), and that we did not
remind sufficiently well to the reader of the term's definition in the
subsequent discussions. We have addressed these concerns in the revised version
of the paper.

Regarding the Newton--Cotes rule, it is one of the components of the
interpolation algorithm. It is described in Sec.~IX-C titled ``Collocation
Nodes.'' The rule is set of coordinates, which can be seen in (26). These
coordinates define a grid upon which the interpolation process is carried out.
Each node of the grid is a potential collocation node, that is, a point at which
the value of the target function might be calculated and added to the surrogate
under construction. The utility of the rule is in the induction of a grid that
satisfies the requirements identified in Sec.~IX-B and summarized at the
beginning of Sec.~IX-C. Again, based on the received feedback, we have realized
that we did not refer sufficiently well to the description of the rule after its
introduction in Sec.~IX-C, which has been improved in the revised manuscript.

\begin{actions}
  \action{Please refer to \cref{0}{1} for our actions.}

  \action{The definition and usage of the term \emph{Smolyak level}, which is
  introduced in Sec.~IX-B in relation to (17), have been revised in Sec.~IX.}

  \action{The description of the Newton--Cotes rule, which is given in
  Sec.~IX-C, has been referred to more extensively from the subsequent
  sections.}
\end{actions}
\end{authors}

\begin{reviewer}
\clabel{2}{3}
It is mentioned that you follow the black-box approach to analyze the
uncertainty. To this, I do not quite follow how can I analyze the uncertainty
due low-level variabilities such as process variation and degradation of
transistors. Also, what will happen if wanted to analyze the impact on the
lifetime of the system? The selection of your case studies cannot reflect that,
and it is not mentioned how can someone reuse this framework for those
scenarios.
\end{reviewer}

\begin{authors}
We agree.

\begin{actions}
  \action{Well done.}
\end{actions}
\end{authors}

\begin{reviewer}
\clabel{2}{4}
There is no mentioning on the accuracy of this overall approach or the
validation against real systems or using actual application traces, while this
can be very easy. Simulation of various processing systems can be exhaustive,
but variation already exists among various of-the-shelf computing products. One
can run some applications on those platforms, and even vary the input data size
to see the final outcome (energy, performance,...). Verification against
experimental works will do two things: 1) solidifies your findings
significantly, 2) gives insight on the accuracy of the framework (do you have
accurate metrics from your framework and experiments, or you observe the same
trend when examining various scenarios?).
\end{reviewer}

\begin{authors}
Please refer to \cref{0}{2} for our response.

\begin{actions}
  \action{Please refer to \cref{0}{2} for our actions.}
\end{actions}
\end{authors}
