\begin{reviewer}
The revised version is much improved compared to the first submission. Most of
my comments have been addressed in the new version.
\end{reviewer}

\begin{authors}
We are glad to hear this. Thank you.
\end{authors}

\begin{reviewer}
\clabel{2}{1}
In terms of organization, I think there are too many sections and I recommend
merging some of them and use subsections when appropriate.
\end{reviewer}

\begin{authors}
We agree with the reviewer. In the revised version of the manuscript, we have
reduced the number of sections from 12 to 8. To elaborate, the two old sections
titled ``Prior Work'' (Sec.~II) and ``Present Work'' (Sec.~III) have been merged
into one section titled ``Prior Work and Our Contribution'' (Sec.~II).
Furthermore, the four old sections titled ``Preliminaries'' (Sec.~IV),
``Problem'' (Sec.~V), ``Solution'' (Sec.~VI), and ``Example'' (Sec.~VII) have
been merged into one section titled ``Problem Formulation and Our Solution''
(Sec.~III). The old sections are now subsections of the corresponding new
sections. In addition, we have refined the names of these subsections.
Specifically, ``Present Work,'' ``Problem,'' ``Solution,'' and ``Example'' are
now called ``Our Contribution,'' ``Problem Formulation,'' ``Our Solution,'' and
``Illustrative Example,'' respectively. We believe that the aforementioned new
names reflect more accurately the content of the corresponding subsections and,
therefore, improve further the reading experience.

We would like to note that, even though the illustrative example has been turned
from a section (Sec.~VII) into a subsection (Sec.~III-D), it still resides in
its own well-demarcated unit, and it is still given prior to any in-depth
discussions, respecting the changes of the first revision.

\begin{actions}
  \action{The number of sections has been reduced from 12 to 8.}

  \action{The names of the merged old sections or the new subsections have been
  refined.}
\end{actions}
\end{authors}

\begin{reviewer}
\clabel{2}{2}
In terms of experimental evaluation, I find the ``Real-life Example'' section a
bit unsatisfying. If running the simulator is expensive, wouldn't it help to
compare with the measurements of an actual processor running the target image
processing application? The simulator can be used in the same way with
appropriate configuration to the processor.
\end{reviewer}

\begin{authors}
The reviewer is right that the proposed framework can be applied to an actual
platform instead of a simulator of such a platform. However, comparing the
results obtained by applying the framework to an actual platform with the
results obtained by applying the framework to a simulator of that platform would
validate the simulator, not our technique. The adequacy of the simulator at hand
is an important concern; however, it is outside of the scope of this work.

Let us now elaborate on the fact that the real-life example revolves around a
simulator instead of an actual platform. This is done due to the following
reason: the use case described in the example is exactly the one that this work
is targeted at. Our main objective is to assist the simulation-based
design-space exploration by providing the designer with an efficient tool for
analyzing the uncertainty present in the system under development. It can be
seen in the paper that the word \emph{simulate} and its derivatives are
ubiquitous. We talk about simulation and simulators in such key parts of the
manuscript as the ones presenting the prior work (Sec.~III-A), problem
formulation (Sec.~III-B), and illustrative example (Sec.~III-D), including a
graphical representation of the workflow of the framework (Fig.~2). Then, in our
opinion, the focal point of the real-life example is apposite: the example flows
naturally from the preceding discussions and keeps the paper coherent.

Furthermore, as we discussed in our response to the first revision and emphasize
throughout the paper (see, for instance, Sec.~III-B), we address the scenario
wherein the evaluation of the metric of interest is expensive, which is the case
with simulations. If they were computationally cheap, one would not needed any
auxiliary framework such as ours; direct Monte Carlo sampling and alike would
suffice. However, simulations are expensive, which necessitates our approach.
\end{authors}

\begin{reviewer}
\clabel{2}{3}
Here are some minor mistakes that should be fixed:

\noindent- ``perse'' should be ``per se''

\noindent- ``These many samples'' instead of ``This many samples''.
\end{reviewer}

\begin{authors}
We appreciate your careful observations. The typos have been eliminated in the
new version.

\begin{actions}
  \action{The typos pointed out by the reviewer have been eliminated.}
\end{actions}
\end{authors}
