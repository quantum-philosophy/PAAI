\begin{reviewer}
The revised version is much improved compared to the first submission. Most of
my comments have been addressed in the new version.
\end{reviewer}

\begin{authors}
We are glad to hear this. Thank you.
\end{authors}

\begin{reviewer}
\clabel{2}{1}
In terms of organization, I think there are too many sections and I recommend
merging some of them and use subsections when appropriate.
\end{reviewer}

\begin{authors}
We agree with the reviewer. In the revised version of the manuscript, we have
reduced the number of sections from 12 to 8. To elaborate, the two old sections
titled ``Prior Work'' (Sec.~II) and ``Present Work'' (Sec.~III) have been merged
into one section titled ``Prior Work and Our Contribution'' (Sec.~II).
Furthermore, the four old sections titled ``Preliminaries'' (Sec.~IV),
``Problem'' (Sec.~V), ``Solution'' (Sec.~VI), and ``Example'' (Sec.~VII) have
been merged into one section titled ``Problem Formulation and Our Solution''
(Sec.~III). The old sections are now subsections of the corresponding new
sections. In addition, we have refined the names of these subsections.
Specifically, ``Present Work,'' ``Problem,'' ``Solution,'' and ``Example'' are
now called ``Our Contribution,'' ``Problem Formulation,'' ``Our Solution,'' and
``Illustrative Example,'' respectively. We believe that the new names reflect
better the content of the corresponding subsections and, therefore, will further
improve the reading experience.

We would like to note that, even though the illustrative example has been turned
from a section (Sec.~VII) into a subsection (Sec.~III-D), it still resides in
its own well demarcated unit, and it is still given prior to any in-depth
discussions, respecting the changes of the first revision.

\begin{actions}
  \action{The number of sections has been reduced from 12 to 8.}

  \action{The names of the grouped old sections or new subsections have been
  refined.}
\end{actions}
\end{authors}

\begin{reviewer}
\clabel{2}{2}
In terms of experimental evaluation, I find the ``Real-life Example'' section a
bit unsatisfying. If running the simulator is expensive, wouldn't it help to
compare with the measurements of an actual processor running the target image
processing application? The simulator can be used in the same way with
appropriate configuration to the processor.
\end{reviewer}

\begin{authors}
The reviewer is right that the proposed framework can be applied to an actual
platform instead of a simulator of a platform. However, the scenario that the
real-life example portrays revolves around a simulator due to the following
reason. The use case that the example describes is the one that our work is
primarily oriented toward. It can be seen that the word \emph{simulate} and its
derivatives are ubiquitous in the paper. We talk about simulation and simulators
in such key parts of the manuscript as the ones presenting the prior work
(Sec.~III-A), problem formulation (Sec.~III-B), and illustrative example
(Sec.~III-D) including a graphical representation of the workflow of the
framework (Fig.~2). Then the focal point of the real-life example is apposite in
our opinion. The example flows naturally from the preceding discussions and
keeps the paper coherent.

Please refer also to our response to \cref{3}{2}.
\end{authors}

\begin{reviewer}
\clabel{2}{3}
Here are some minor mistakes that should be fixed:

\noindent- ``perse'' should be ``per se''

\noindent- ``These many samples'' instead of ``This many samples''.
\end{reviewer}

\begin{authors}
We appreciate your careful observations. The typos have been eliminated in the
new version.

\begin{actions}
  \action{The typos pointed out by the reviewer have been eliminated.}
\end{actions}
\end{authors}
