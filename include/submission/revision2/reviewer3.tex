\begin{reviewer}
The authors have made major changes to the paper to address my comments, thanks
for your effort.
\end{reviewer}

\begin{authors}
Thank \emph{you} for your feedback.
\end{authors}

\begin{reviewer}
\clabel{3}{1}
However, I am a bit confused on the addition of Section XI.C. This section is
valuable, but it is written in an extremely compact way. I was expecting a set
of graphs similar to Figure 5. It would be great if less is shown from the
synthetic example, and more from actual simulations.
\end{reviewer}

\begin{authors}
We agree that the subsection in question---which is now Sec.~VII-C due to the
changes associated with \cref{2}{1}---feels rather dense, which is primarily due
to the space limit. In order to make it more welcoming, we have revisited the
language that is used in that subsection, reconsidered the subsection's
organization in terms of paragraphs, and added a number of clarifying details.

Let us now turn to the graphs mentioned by the reviewer. Each of those graphs
shows how our accuracy changes as the number of interpolation points increases.
This accuracy is measured with respect to the reference solution that is
obtained by exhaustive sampling. As we discussed in our response to the first
revision and write in Sec.~VII-C, obtaining such a solution is prohibitively
expensive in the case of the real-life example due to the large simulation
times. Consequently, we were not able to provide a graph similar to those in
Fig.~5 for the real-life example.

We would like to emphasize that the main purpose of the real-life example is to
ensure that it is indeed straightforward to deploy the proposed framework, which
is what we claim in the paper. Moreover, since the corresponding infrastructure
has been open-sourced, potential practitioners can go over and study the exact
steps that need to be taken in order to apply the framework; they can also reuse
the whole infrastructure as well as make use of a subset of its components.

\begin{actions}
  \action{Section~VII-C has been updated in order to make the reading experience
  better.}
\end{actions}
\end{authors}

\begin{reviewer}
\clabel{3}{2}
Here is another suggestion to increase the results, feel free to use it. Use
VIPS on actual machines, and rely on the Intel PCM to measure the processor
stats/energy. This should enhance your analysis as you will rely on real-time
measurements rather than simulations.
\end{reviewer}

\begin{authors}
Thank you for the suggestion. We do appreciate it; however, we would like to
leave it for the future work as we currently do not have the resources and, more
importantly, the skills that are needed in order to diligently undertake such an
experiment.

Please refer also to our response to \cref{2}{2}.
\end{authors}
