Sampling methods would be a reasonable solution to probabilistic analysis of
electronic systems if electronic systems were inexpensive (with respect to the
computation time) to simulate. In order to eliminate or reduce the costs
associated with direct sampling, a number of techniques have been introduced.

The techniques are diverse and can be classified in many ways. Two natural
distinguishing features of a technique are 1)~the source of uncertainty and
2)~the quantity of interest that the technique is tailored for. The technique
gives a probabilistic characterization of the latter while taking into account
the deteriorating effect of the former. 3)~The solution that the technique
proposes in order to tackle the problem is another feature to keep an eye on.
The solution can be based on a mathematical framework or a statistical method,
for instance. In what follows, we shall discuss the prior work from the
standpoint of the aforementioned three features.

Let us first discuss physical sources of uncertainty and, more concretely,
process variation as it has been extensively studied. Circuit-level timing and
power analyses under process variation are undertaken in \cite{bhardwaj2008} by
means of polynomial chaos (\up{PC}) expansions \cite{xiu2010}. The work in
\cite{juan2012} models static steady-state temperature and accounts for process
variation by leveraging the linearity of Gaussian distributions and
time-invariant systems. A stochastic collocation \cite{xiu2010} approach to
static steady-state temperature analysis is given in \cite{lee2013}, which
relies on global interpolation using Newton polynomials. In \cite{ukhov2014},
transient temperature analysis is considered, and process variation is addressed
via \up{PC} expansions. The machinery of \up{PC} expansions is also utilized in
\cite{ukhov2015} in order to model dynamic steady-state temperature
\cite{ukhov2012} and to enhance reliability models.

Let us now turn to digital sources of uncertainty. In this context, timing
analysis has drawn the major attention \cite{quinton2012}. A seminal work on
response time analysis of periodic tasks with random execution times on
uniprocessors is reported in \cite{diaz2002}. A novel analytical solution to
this problem is given in \cite{tanasa2015}, which makes milder assumptions and
allows for addressing larger, previously unsolvable problems. The framework
proposed in \cite{santinelli2011} facilitates task scheduling by providing
probabilistic bounds on the resource given to a task flow and the resource
needed by that task flow; the approach is based on real-time calculus and is
applicable to electronic systems.

Studying the literature on probabilistic analysis of electronic systems, one can
note a pronounced trend: the generality and straightforwardness of sampling
methods tend to be lost. To elaborate, a technique typically 1)~requires
restrictive conditions to be fulfilled such as the absence of dependencies,
2)~is tailored to one concrete quantity such as the response time, and
3)~requires substantial effort to be deployed.

However, one should keep in mind what is practical. First of all, although
additional assumptions might make the mathematics analytically solvable, they
often do not hold and oversimplify the model. An exact analytical solution might
also be extremely complex, requiring a lot of computational resources upon
evaluation. Furthermore, it is often the case that there has been developed a
robust simulator of the quantity of interest for the utopian deterministic
scenario. Switching gears to probabilistic analysis typically means discarding
this battle-tested code all together and implementing something else from
scratch, which is wasteful and not desirable.

Some of the techniques listed earlier in this section, in fact, preserve the
generality and straightforwardness of sampling methods. An example is the
uncertainty analysis presented in \cite{ukhov2015}. The reason is that the
construction of \up{PC} expansions in \cite{ukhov2015} is undertaken by means of
so-called nonintrusive spectral projections \cite{xiu2010}, which do not need to
look inside the ``black box,'' similar to sampling methods. However, as
motivated in \sref{introduction}, nonsmoothness is a serious problem for global
approximation based on polynomials. The convergence of \up{PC} expansions, for
instance, deteriorates substantially in such cases, requiring partitioning of
the stochastic space in order to alleviate the problem. Therefore, it is not
straightforward to apply such techniques as the one given in \cite{ukhov2015} in
the context of digital sources of uncertainty exhibiting nonsmoothness.

To conclude, the available techniques for probabilistic analysis of electronic
systems are restricted in use. Flexible, capable, and easy-to-deploy frameworks
are needed.
