The techniques developed for probabilistic analysis of multiprocessor systems
are diverse as the area itself and can be classified in many ways. Two natural
classifications originate from the inputs and outputs of the systems that the
techniques have been tailored for. An input, in this context, refers to a source
of uncertainty, and an output refers to a quantity of interest. A technique
tries to give a probabilistic characterization of the latter while taking into
account the effect of the former. The solution that the technique proposes to
solve the problem gives birth to another classification. A solution can be
powered by, for instance, a mathematical framework, statistical method, or
computational algorithm. Let us give an overview of some of the available
techniques from the perspective of the aforementioned classifications.

One of the concerns of an electronic system's designer is process variation
\cite{srivastava2005}. The work in \cite{juan2012} models static steady-state
temperature and accounts for process variation by leveraging the linearity of
Gaussian distributions and time-invariant systems. A stochastic collocation
approach \cite{xiu2010} to static steady-state temperature analysis powered by
Newton polynomials is given in \cite{lee2013}. In \cite{ukhov2014}, transient
temperature analysis is considered, and process variation is address by means of
polynomial-chaos (\abbr{PC}) expansions \cite{xiu2010}. \abbr{PC} expansions
have also been utilized in \cite{ukhov2015} to analyze dynamic steady-state
temperature and parameters of reliability models under process variation.

In order to eliminate or reduce the costs associated with compute experiments in
the context of multiprocessor system design, a number of techniques have been
introduced.

Let us begin with probabilistic timing analysis.

Regarding probabilistic power analysis, the analysis under process variation is
considered in \cite{ukhov2014}.

Let us turn to probabilistic temperature analysis. The analysis under process
variation is considered in \cite{ukhov2014} and \cite{ukhov2015}.
