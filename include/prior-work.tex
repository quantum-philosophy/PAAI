The major problem with sampling techniques is in sampling: one should be able to
collect a sufficiently large number of realizations of the quantity of interest
in order to draw sound conclusions with respect to that quantity
\cite{diaz-emparanza2002}. Then the main concerns are: How many samples do we
need? How many samples can we afford? How long does it take to obtain one
sample? How much does one sample cost? When the subject under analysis is
expensive---as measured by any metric that makes the most sense to the problem
at hand---computer experiments are rendered as slow and often infeasible.

In order to eliminate or reduce the costs associated with compute experiments in
the context of probabilistic analysis of multiprocessor systems, a number of
techniques have been introduced. The techniques are diverse as the area itself
and can be classified in many ways. Two natural features, distinguishing the
techniques, originate from the inputs and outputs of the systems that the
techniques have been tailored for. An input, in this context, refers to a source
of uncertainty, and an output refers to a quantity of interest. A technique
tries to give a probabilistic characterization of the latter while taking into
account the effect of the former. The solution that the technique proposes to
solve the problem is another feature to pay attention to. A solution can be
powered by, for instance, a mathematical framework, statistical method, or
computational algorithm. Let us give an overview of some of the available
techniques by focusing on the aforementioned traits.

Sources of uncertainty can be divided into analog and digital. The major
representative of the former is process variation \cite{srivastava2005}, which
has been central for many research projects as we discuss next. The work in
\cite{juan2012} models static steady-state temperature and accounts for process
variation by leveraging the linearity of Gaussian distributions and
time-invariant systems. A stochastic collocation approach \cite{xiu2010} to
static steady-state temperature analysis powered by Newton polynomials is given
in \cite{lee2013}. In \cite{ukhov2014}, transient temperature analysis is
considered, and process variation is address by means of polynomial-chaos
(\abbr{PC}) expansions \cite{xiu2010}. \abbr{PC} expansions are also utilized in
\cite{ukhov2015} to model dynamic steady-state temperature and to enhance
reliability models. Circuit-level timing and power analysis under process
variation is undertaken in \cite{bhardwaj2008} via \abbr{PC} expansions.

Sources of uncertainty labeled as ``digital'' are phenomena of the digital world
rather than physical. Many applications running on modern devices exhibit
nondeterministic behaviors: their characteristics change from one phase or
invocation to another, which can be due to, for example, the runtime environment
or input data. Such behaviors are essentially random, and accounting for them is
highly beneficial if not essential; consider, for instance, energy efficiency
and safety. In this context, timing analysis has drawn a lot of attention
\cite{quinton2012}. A seminal work on response time analysis of periodic tasks
with random execution times is reported in \cite{diaz2002}. A novel analytical
solution to this problem is given in \cite{tanasa2015}, which makes milder
assumptions and allows for addressing larger, previously infeasible problems.

The framework that we propose in this paper is mainly oriented towards digital
sources of uncertainty, and our experimental setup has been tailored
accordingly. However, the framework is expected to perform well when modeling
other uncertainties such as those due to process variation.
