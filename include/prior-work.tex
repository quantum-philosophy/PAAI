Computer experiments together with Monte Carlo (\abbr{MC}) methods would be a
reasonable solution to probabilistic analysis of multiprocessor systems if
multiprocessor systems were inexpensive to simulate (in terms of time, money, or
other resources). In order to eliminate or reduce the costs associated with
direct sampling, a number of techniques have been introduced. The techniques are
diverse as the area itself and can be classified in many ways. Two natural
characteristics, distinguishing the techniques, originate from the sources of
uncertainty and quantities of interests that the techniques have been tailored
for. A technique tries to give a probabilistic characterization of the latter
while taking into account the deteriorating effect of the former. The solution
that a technique proposes to tackle the corresponding problem is another
characteristic to pay attention to. A solution can be based on, for instance, a
mathematical framework, statistical method, or computational algorithm. In what
follows, we shall focus on the three aforementioned characteristics.

Let us first discuss analog sources of uncertainty and, more concretely, process
variation as it by far dominates. Circuit-level timing and power analyses under
process variation are undertaken in \cite{bhardwaj2008} by means of polynomial
chaos (\abbr{PC}) expansions \cite{xiu2010}. The work in \cite{juan2012} models
static steady-state temperature and accounts for process variation by leveraging
the linearity of Gaussian distributions and time-invariant systems. A stochastic
collocation \cite{xiu2010} approach to static steady-state temperature analysis
is given in \cite{lee2013}, which relies on global interpolation using Newton
polynomials. In \cite{ukhov2014}, transient temperature analysis is considered,
and process variation is address via \abbr{PC} expansions. The machinery of
\abbr{PC} expansions is also utilized in \cite{ukhov2015} in order to model
dynamic steady-state temperature and to enhance reliability models.

Let us now turn to digital sources of uncertainty. In this context, timing
analysis has drawn the major attention \cite{quinton2012}. A seminal work on
response time analysis of periodic tasks with random execution times on
uniprocessors is reported in \cite{diaz2002}. A novel analytical solution to
this problem is given in \cite{tanasa2015}, which makes milder assumptions and
allows for addressing larger, previously infeasible problems. The framework
proposed in \cite{santinelli2011} facilitates task scheduling by providing
probabilistic bounds on the resource given to a task flow and the resource
needed by that task flow; the approach is based on real-time calculus and is
applicable to multiprocessor systems. Temperature variations due to
uncertainties in timing have also been extensively studied, and the accent has
been primarily on the worst case. An example is the work presented in
\cite{yang2013}, which builds on real-time calculus and studies the maximal
temperature.

Studying the literature on probabilistic analysis of multiprocessor systems, one
can note a pronounced trend: the generality and straightforwardness of \abbr{MC}
methods tend to be lost. First of all, a technique usually requires restrictive
conditions to be fulfilled, such as the absence of dependencies or the usage of
a certain scheduling policy, and is applicable to one particular quantity, such
the response time or maximal temperature. Second, the designer typically has to
invest much effort into applying a chosen technique; the battle-tested
simulators that have been developed over the years for the ideal deterministic
case have to be substantially modified or abandoned all together. This trend is,
of course, reasonable: the techniques try to excel by narrowing down the scope
and asking for more.

The framework that we propose in this paper is mainly oriented towards digital
sources of uncertainty, and our experimental setup has been tailored
accordingly. However, the framework is expected to perform well when modeling
other uncertainties such as those due to process variation.
