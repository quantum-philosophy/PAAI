Consider an electronic system composed of two major components: a platform and
an application. The platform is a collection of heterogeneous processing
elements, and the application is a collection of interdependent tasks.

\updated{The designer is interested in evaluating a metric $\g$ that
characterizes the electronic system under consideration from a certain
perspective.} Examples of $\g$ include the execution delay of the application or
a task, energy consumption of the platform or a processing element, and maximum
temperature of the platform or a processing element.

\updated{The metric $\g$ depends on a set of parameters $\vu$ that are uncertain
at the design stage.} Examples of $\vu$ include the amount of data the
application needs to process, execution times of the tasks, and properties of
the environment.

\updated{The parameters $\vu$ are given as a random vector $\vu = (\u_i)_{i =
1}^\nu$ with an arbitrary but known distribution $\distribution_\vu$.} The
dependency of $\g$ on $\vu$, written as $\g(\vu)$, implies that $\g$ is random
to the designer. For a given outcome of $\vu$, however, the evaluation of $\g$
is purely deterministic. \updated{This operation is typically undertaken by an
adequate simulator of the system at hand, and it is assumed to be doable but
computationally expensive.}

\updated{Our objective is to develop an efficient framework for estimating the
probability distribution of the metric of interest $\g$ dependent on the
uncertain parameters $\vu$.} The framework is required to be able to handle
nondifferentiable and even discontinuous dependencies between $\g$ and $\vu$ as
they constitute an important class of problems for electronic-system design.
