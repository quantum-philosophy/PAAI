Consider a multiprocessor system composed of two major components: a platform
and an application. The platform consists of a set of heterogeneous processing
elements $\procs = \{ 1, \dots, \np \}$ and is equipped with a thermal package.
The application is given as a set of tasks $\tasks = \{ 1, \dots, \nt \}$. In
what follows, the system will be denoted by $(\procs, \tasks)$.

The designer is assumed to be interested in studying a quantity $\g$ that
characterizes the system described above. The examples include the end-to-end
delay, total energy, maximal temperature, and power and temperature profiles.
The quantity $\g$, referred to as the quantity of interest, depends on a set of
parameters that are uncertain at the design stage. In this work, we focus on
system-level parameters such as the execution times of the tasks. The uncertain
parameters are modeled as a random vector $\vu = (\u_i)_{i = 1}^\nu$ with
distribution $\distribution_\vu$. It is important to note that the dependency of
$\g$ on $\vu$, written as $\g(\vu)$, inevitably implies that $\g$ is random to
the designer; however, for a given $\vu$, $\g$ is purely deterministic. Then,
the characterization of $\g$ refers to the estimation of the probability
distribution of $\g$, which implies the availability of such quantities as the
expected value, variance, and probabilities of arbitrary events.

Lastly, for a given $\vu$, the evaluation of $\g$ is assumed to be doable but
computationally or otherwise expensive. If the cost of $\g$ was negligible, one
could collect thousands of independent samples of $\g$ and compute the required
statistics about $\g$ without involving any framework like the one presented in
this paper.

We pursue the following major objectives:
\begin{itemize}

\item \textbf{Objective~1.} Develop an efficient framework for probabilistic
  analysis of the quantity of interest $\g$ characterizing the system $(\procs,
  \tasks)$ such that the uncertainty originating from the parameters $\vu$ is
  taken into consideration.

\end{itemize}
