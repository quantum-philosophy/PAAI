As noted earlier, making use of a sampling method is a compelling approach to
uncertainty quantification. We would readily apply such a method to study our
metric $\g$ if only evaluating $\g$ had a small cost, which it does not.

Our solution to the above predicament is to construct a light representation of
the heavy $\g$ and study this representation instead of $\g$. The surrogate that
we build is based on adaptive interpolation: $\g$ is evaluated at a number of
strategically chosen collocation nodes, and any other values of $\g$ are
reconstructed on demand (without involving $\g$) using a set of basis functions
mediating between the collected values of $\g$. The benefit of this approach is
in the number of invocations of the metric $\g$: only a few evaluations of $\g$
are needed, and the rest of our probabilistic analysis is powered by the
constructed interpolant, which, in contrast to $\g$, has a negligible cost.

Let us delineate the steps involved in the solution process. Recall that $\g$ is
parameterized by the uncertain parameters $\vu$, and these variables are the
only source of randomness. 1)~The metric $\g$ is reparameterized in terms of an
auxiliary random vector $\vz$ extracted from $\vu$; the necessity of this stage
will become clear later on. 2)~An interpolant of $\g$ is constructed by
considering $\g$ as a deterministic function of $\vz$ and evaluating $\g$ at a
small set of carefully chosen points. 3)~The probability distribution of $\g$ is
then estimated by applying an arbitrary sampling method to the constructed
interpolant of $\g$.

The first two of the above steps should be undertaken with a great care as
interpolation of multivariate functions is a challenging task. This aspect will
be discussed in detail in \sref{modeling} and \sref{interpolation}. However,
before we proceed to those sections, we would like to give an illustrative
example.
