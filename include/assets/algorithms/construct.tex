\begin{algorithm}
  \caption{\emph{Construct} an interpolant for a function.}
  \alab{construct}
  \begin{algorithmic}[1]
    \vspace{0.4em}

    \Require{target} \Comment{A function to approximate}
    \Ensure{surrogate} \Comment{The approximating object}

    \vspace{0.4em}

    \Let{level}{0}
    \Let{indices}{ComputeIndices(level)}
    \Let{surrogate}{InitializeEmpty()}

    \Loop
      \Let{nodes}{ComputeNodes(indices)}
      \Let{values}{Execute(target, nodes)}
      \Let{interpolated\_values}{Evaluate(surrogate, nodes)}
      \Let{surpluses}{$\text{values} - \text{interpolated\_values}$}
      \State Append(surrogate, indices, surpluses)
      \If{IsLimitReached(surrogate)}
        \State \textbf{break}
      \EndIf
      \For{k \textbf{in range of} indices}
        \If{IsAccurateEnough(surpluses[k])}
          \State Exclude(indices, k)
        \EndIf
      \EndFor
      \If{IsEmpty(indices)}
        \State \textbf{break}
      \EndIf
      \Let{indices}{ComputeChildren(indices)}
      \Let{level}{$\text{level} + 1$}
    \EndLoop

    \State \textbf{return} surrogate

    \vspace{0.4em}
  \end{algorithmic}
\end{algorithm}
