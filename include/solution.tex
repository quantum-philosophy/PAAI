Monte Carlo (\abbr{MC}) sampling is a simple yet powerful approach to
uncertainty quantification. We would readily apply \abbr{MC} sampling to our
quantity of interest $\g$ if only $\g$ had a negligible computational cost. Now,
recall that $\g$ is parametrized by the uncertain parameters $\vu$, and these
variables are the only source of randomness. Then, our solution to the
aforementioned quandary is to (i) reparametrize $\g$ in terms of an auxiliary
random vector $\vz$ extracted from $\vu$ (the necessary will become clear later
on); (ii) construct a light interpolant of $\g$, considering $\g$ as a
deterministic function of $\vz$, by evaluating $\g$ only at a few strategically
chosen nodes; and (iii) use the constructed interpolant instead of $\g$ to
perform \abbr{MC} sampling.

Interpolation of high-dimensional functions is a challenging task \perse, and it
should be approached with a great care. However, before doing so, one should
also have a lucid understand of what the final goal of the subsequent analysis
is and choose the subject function accordingly. The quantity of interest should
be generally preprocessed in order to make interpolation efficient and easy to
undertake. In this regard, domain-specific knowledge is invaluable. All these
aspects will be discussed in detail in the rest of the paper.
