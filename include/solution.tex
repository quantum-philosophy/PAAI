As noted earlier, performing Monte Carlo (\abbr{MC}) simulations is a compelling
approach to uncertainty quantification. We would readily apply an \abbr{MC}
method to study our quantity of interest $\g$ if only $\g$ had a negligible
cost, which it does not.

Recall that $\g$ is parameterized by the uncertain parameters $\vu$, and these
variables are the only source of randomness. Then, our solution to the
aforementioned quandary is as follows. First, we reparameterize $\g$ in terms of
an auxiliary random vector $\vz$ extracted from $\vu$; the necessity of this
stage will become clear later on. Second, we construct a light interpolant of
$\g$ by considering $\g$ as a deterministic function of $\vz$ and evaluating
$\g$ at a small number of strategically chosen points. Finally, we use the
constructed interpolant instead of $\g$ to perform sampling. The benefit of this
approach is in the number of invocations of $\g$: only a few evaluations of $\g$
are needed, and the rest of the analysis is powered by the interpolant, which
does have a negligible cost as opposed to $\g$.

Interpolation of high-dimensional functions is a challenging task, which should
be approached with a great care. This aspect will be discussed in detail in
\sref{interpolation}. Before we proceed to interpolation, however, we need flesh
out further our uncertain parameters and quantities of interest.
