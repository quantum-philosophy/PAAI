Our work brings the following major contribution: We develop a framework for
probabilistic analysis of electronic systems that is straightforward to use and
is applicable to a wide range of uncertainty-quantification problems.

Let us substantiate the above statement. The usage of the framework is
streamlined because it has the same low entrance requirements as sampling
techniques: one only has to be able to evaluate the quantity of interest given a
set of deterministic parameters. Moreover, the framework can be utilized in
scenarios with limited knowledge about the joint probability distribution of the
uncertain parameters, which are common in practice. The scope of the framework
is wide because the framework has a powerful approximation engine. The approach
that we take belongs to the class of stochastic collocation techniques
\cite{xiu2010}. Stochastic collocation leverages interpolation as a means of
uncertainty quantification, which should be contrasted with other techniques
such as \up{PC} expansions relying on regression. More concretely, our framework
makes use of adaptive hierarchical interpolation on sparse grids
\cite{jakeman2012, klimke2006, ma2009}, which enables the framework to
substantially reduce the costs associated with uncertainty quantification, as
discussed and illustrated in \sref{introduction}.

In addition to the above contribution, we open-source our implementation of the
ideas presented in this paper \cite{sources}, which also includes the whole
experimental setup described in \sref{experimentation}. The implementation is
broken down into multiple libraries, which makes it well disposed to
cherry-picked reusage.

Lastly, let us note that, in this paper, we are mainly oriented toward digital
sources of uncertainty introduced earlier, and our examples and experiments are
tailored accordingly. However, the framework is expected to perform well when
modeling other uncertainties such as those due to process variation.
