The life cycle of interpolation has roughly two phases: construction and usage.
The construction phase involves sampling the target function $\f$ at collocation
points and produces a set of artifacts needed for the actual interpolation at
the usage phase. Regarding the aforementioned artifacts, it can be seen in
\eref{tensor-surplus-1d} that an approximant is entirely characterized by a set
of indexing pairs $\{ (\vi, \vj) \}$ and the corresponding surpluses $\{
\surplus(\vx^\vi_\vj) \}$. Recall that the multi-index $\vi$ captures the levels
of interpolation with respect to each dimension, and $\vj$ captures the
corresponding orders (see \sref{tensor-product}). Each pair $(\vi, \vj)$
unambiguously maps to a collocation point and a basis function, which, together
with the corresponding surplus, allow one to evaluate the interpolant at any
point of interest.

\begin{algorithm}
  \caption{\emph{Construct} an interpolant of a function.}
  \alab{construct}
  \begin{algorithmic}[1]
    \vspace{0.4em}

    \Require{target} \Comment{A function to interpolate}
    \Ensure{surrogate} \Comment{Artifacts of interpolation}

    \vspace{0.4em}

    \Let{surrogate}{New()}

    \For{s = First(); !Check(s); s = Next(s)}
      \Let{s.nodes}{Locate(s.indices)}
      \Let{s.values}{Invoke(target, s.nodes)}
      \Let{s.estimates}{Evaluate(surrogate, s.nodes)}
      \Let{s.surpluses}{Subtract(s.values, s.estimates)}
      \Let{s.scores}{Assess(s.surpluses)}
      \State Append(surrogate, s.indices, s.surpluses)
    \EndFor

    \State \textbf{return} surrogate

    \vspace{0.4em}
  \end{algorithmic}
\end{algorithm}

The conceptual code corresponding to the construction phase is given in
\aref{construct} called \token{Construct}.

\begin{remark}
In the pseudocodes presented in this paper, many implementation details---such
as memory management---have been purposely omitted in order to distill the core
ideas. In addition, some of the auxiliary subroutines that the algorithms make
use of are not described as being self-explanatory.
\end{remark}

The input to \token{Construct}, \token{target}, is the function to be
approximated. The output, \token{surrogate}, is a structure containing the
artifacts mentioned earlier: indices and surpluses. The key steps of
\token{Construct} are as follows.

\begin{compactlist}

\point{Line~4:} The outermost loop corresponds to the increasing level of
Smolyak's interpolation, which is denoted by $l$ in \eref{smolyak} and
\eref{smolyak-incremental} and by \token{level} in the code. The \token{indices}
variable is a working set containing the indexing pairs specific to the current
level. The set is initially populated on line~2 by the indices of the zeroth
level. For the open Newton--Cotes rule, it is the zero pair $(\v{0}, \v{0})$,
which correspond to the root node $(0.5)_{i = 1}^\nin$ and the basis function
$\e^\v{0}_\v{0}(\vx) = 1$.

\point{Line~5:} \token{ComputeNodes} takes a set of indexing pairs and returns
the corresponding (multidimensional) nodes of the underlying sparse grid; see
\sref{collocation-nodes}.

\point{Line~6:} \token{Execute} calls the target function at each of the given
nodes and returns the corresponding values. This is a prominent candidate for
parallelization since each node can be evaluated independently from the rest and
the target function is by far the most computationally intensive part.

\point{Line~7:} \token{Evaluate} utilizes the surrogate constructed so far in
order to calculate an approximation to the values of the target function
obtained on line~6; \token{Evaluate} is \aref{evaluate}, and it will be
described separately.

\point{Line~9:} \token{Append} ameliorates the surrogate by extending it with
the indices and surpluses of the current iteration.

\point{Line~10:} The check is to stop the algorithm when it reaches a limit on a
resource such as the interpolation level and number of \token{target}'s
invocations. Note that the latter example might not be precise to check at this
particular point; a better precision can be achieved by trimming \token{indices}
before \token{Execute}.

\point{Line~13:} The loop iterates over the surpluses of the current level and
identifies those indices that need refinement. \token{IsAccurateEnough}
represents a refinement strategy and might not necessarily be solely based on
the surpluses. In our experiments, we use the technique shown in
\eref{absolute-error} and \eref{relative-error}.

\point{Line~18:} The check is to stop the algorithm when there is nothing left
to refine, which is dictated by \token{IsAccurateEnough}.

\point{Line~21:} \token{ComputeChildren} takes the indices selected for
refinement and returns the corresponding child indices of the next level; see
\fref{rule} and \sref{adaptivity}.

\end{compactlist}

\begin{algorithm}
  \caption{\emph{Evaluate} an interpolant at a set of points.}
  \alab{evaluate}
  \begin{algorithmic}[1]
    \vspace{0.4em}

    \Require{surrogate, points} \Comment{Interpolant and points}
    \Ensure{estimates} \Comment{Approximated values}

    \vspace{0.4em}

    \Let{estimates}{New()}

    \For{point \textbf{in} points}
      \Let{estimate}{0}
      \For{(index, surplus) \textbf{in} surrogate}
        \Let{weight}{basis.Compute(index, point)}
        \Let{estimate}{$\text{estimate} + \text{surplus} \times \text{weight}$}
      \EndFor
      \State Append(estimates, estimate)
    \EndFor

    \State \textbf{return} estimates

    \vspace{0.6em}
  \end{algorithmic}
\end{algorithm}

Let us now turn to the usage phase of an interpolant. The corresponding
pseudocode is given in \aref{evaluate}, which is called \token{Evaluate}. This
algorithm is also involved in the construction phase; see \aref{construct},
line~7. The main steps of \token{Evaluate} are given below.

\begin{compactlist}

\point{Line~5:} The loop directly corresponds to \eref{smolyak-incremental} and
\eref{tensor-surplus-1d} with the only exception that, due to the adaptivity of
the construction procedure, the loop is likely to contain only a subset of the
summands in the aforementioned equations.

\point{Line~6:} \token{ComputBasis} takes a pair $(\vi, \vj)$ and a point and
returns the value of the (multidimensional) basis function corresponding to the
pair evaluated at the point; see \sref{basis-functions}.

\end{compactlist}

To recapitulate, in this section, we have presented the key component of the
proposed framework for probabilistic analysis of multiprocessor systems: an
efficient approach to multidimensional interpolation. The interpolation
technique is based on the Smolyak algorithm, which enables the interpolation to
be performed in an adaptive hierarchical manner, highly beneficial for practical
computations. The overall technique has been consolidated in \aref{construct}
and \aref{evaluate}.
