The life cycle of interpolation has roughly two phases: construction and usage.
The construction phase involves evaluating the target function $\f$ at
collocation points and produces a set of artifacts needed for the actual
interpolation at the usage phase. Regarding the aforementioned artifacts, it can
be seen in \eref{smolyak-hierarchical} that an approximant is entirely
characterized by a set of indexing pairs $\{ (\vi_k, \vj_k) \}_k$ and the
corresponding surpluses $\{ \surplus(\vx^{\vi_k}_{\vj_k}) \}_k$. Recall that the
multi-index $\vi$ captures the levels of interpolation with respect to each
dimension, and $\vj$ captures the corresponding orders (see
\sref{tensor-product}). Each pair $(\vi, \vj)$ unambiguously identifies a
collocation point and a basis function, which, together with the corresponding
surplus, allow one to evaluate the interpolant at any point of interest.

\begin{algorithm}
  \caption{\emph{Construct} an interpolant of a function.}
  \alab{construct}
  \begin{algorithmic}[1]
    \vspace{0.4em}

    \Require{target} \Comment{A function to approximate}
    \Ensure{surrogate} \Comment{Artifacts of approximation}

    \vspace{0.4em}

    \Let{level}{0}
    \Let{indices}{grid.ComputeIndices(level)}
    \Let{surrogate}{NewSurrogate()}

    \Loop
      \Let{nodes}{grid.ComputeNodes(indices)}
      \Let{values}{Invoke(target, nodes)}
      \Let{approximations}{Evaluate(surrogate, nodes)}
      \Let{surpluses}{$\text{values} - \text{approximations}$}
      \State Append(surrogate.Indices, indices)
      \State Append(surrogate.Surpluses, surpluses)
      \If{IsEnough(surrogate)}
        \State \textbf{break}
      \EndIf
      \For{i \textbf{in range of} indices}
        \If{IsAccurate(surpluses[i])}
          \State Exclude(indices, i)
        \EndIf
      \EndFor
      \If{IsEmpty(indices)}
        \State \textbf{break}
      \EndIf
      \Let{indices}{grid.ComputeNeighbors(indices)}
      \Let{level}{$\text{level} + 1$}
    \EndLoop

    \State \textbf{return} surrogate

    \vspace{0.4em}
  \end{algorithmic}
\end{algorithm}

The conceptual code corresponding to the construction phase is given in
\aref{construct} called \token{Construct}.

\begin{remark}
In the pseudocodes presented in this paper, many implementation details such as
memory management have been purposely omitted in order to distill the core
ideas. In addition, some of the auxiliary subroutines that the algorithms make
use of are not described as being self-explanatory.
\end{remark}

The input \token{target} is a function $\f$ to be approximated. The output
\token{surrogate} is a structure containing the artifacts mentioned earlier:
indexing pairs and surpluses; hereafter, the former are referred to as just
indices. The key steps of the \token{Construct} algorithm are as follows.

\begin{compactlist}

\point{Line~4:} Each iteration of the outermost loop corresponds to a level of
Smolyak's interpolation, which is denoted by $l$ in \eref{smolyak-hierarchical}
and by \token{level} in the code. The \token{indices} variable is a working set
containing the indices of the current (in progress) level. The set is initially
populated on line~2 by the indices of level~0, which is just $\{ (\v{0}, \v{0})
\}$ for the Newton--Cotes grid.

\point{Line~5:} \token{grid.ComputeNodes} takes a set of indices and returns the
corresponding (multidimensional) nodes of the underlying sparse grid; see
\sref{collocation-nodes}.

\point{Line~6:} \token{Invoke} exercises the target function at each of the
given nodes and returns the corresponding values. This is a prominent candidate
for parallelization since each collocation node can be evaluated independently
from the rest.

\point{Line~7:} \token{Evaluate} utilizes the interpolant constructed so far in
order to calculate approximations to the true values of the target function
obtained on line~6. The \token{Evaluate} function is \aref{evaluate}, and it
will be discussed separately.

\point{Line~9--10:} \token{Append} ameliorates the surrogate by extending it
with the indices and surpluses of the current iteration.

\point{Line~11:} The check is to stop the algorithm when it reaches a
user-defined limit such as the maximal level of interpolation, number of
\token{target}'s invocations, or time spent on interpolation.

\point{Line~14:} The loop iterates over the surpluses of the current level and
identifies those indices that need refinement. The \token{IsAccurate} function
represents a refinement strategy and might not necessarily be solely based on
the surpluses. In our experiments, we use the formula given in \eref{error}.

\point{Line~18:} The check is to stop the algorithm when there is nothing left
to refine, which is dictated by \token{IsAccurate}.

\point{Line~21:} \token{grid.ComputeNeighbors} takes the indices selected for
refinement and returns the corresponding child indices of the next level; see
\fref{rule} and \sref{adaptivity}.

\end{compactlist}

\begin{algorithm}
  \caption{\emph{Evaluate} an interpolant at a set of points.}
  \alab{evaluate}
  \begin{algorithmic}[1]
    \vspace{0.4em}

    \Require{surrogate} \Comment{Artifacts of approximation}
    \Require{points} \Comment{Points of interest}
    \Ensure{values} \Comment{Approximated values}

    \vspace{0.4em}

    \Let{indices}{surrogate.Indices}
    \Let{surpluses}{surrogate.Surpluses}
    \Let{values}{NewArray()}

    \For{i \textbf{in range of} points}
      \Let{value}{0}
      \For{j \textbf{in range of} indices}
        \Let{weight}{basis.Compute(indices[j], points[i])}
        \Let{value}{$\text{value} + \text{weight} \times \text{surpluses[j]}$}
      \EndFor
      \State Append(values, value)
    \EndFor

    \State \textbf{return} values

    \vspace{0.4em}
  \end{algorithmic}
\end{algorithm}

Let us now turn to the usage phase of interpolant. The corresponding pseudocode
is given in \aref{evaluate}, which is called \token{Evaluate}. This algorithm is
also involved in the construction phase; see \aref{construct}, line~7. The main
steps of the \token{Evaluate} algorithm are given below.

\begin{compactlist}

\point{Line~6:} The inner loop directly corresponds to
\eref{smolyak-hierarchical} with the exception that, due to adaptivity, the loop
generally iterates over a subset of the summands in the aforementioned equation,
which we discussed in \sref{adaptivity}.

\point{Line~6:} \token{basis.Compute} takes a pair $(\vi, \vj)$ and a point and
returns the value of the (multidimensional) basis function corresponding to the
pair evaluated at the point; see \sref{basis-functions}.

\end{compactlist}

To recapitulate, in this section, we have presented the key component of the
proposed framework for probabilistic analysis of electronic systems: an
efficient approach to multidimensional interpolation. The interpolation
technique is based on the Smolyak algorithm, which enables the interpolation to
be performed in an adaptive hierarchical manner, highly beneficial for practical
computations. The overall technique has been consolidated in \aref{construct}
and \aref{evaluate}.
