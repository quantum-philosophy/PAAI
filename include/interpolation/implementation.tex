The life cycle of interpolation has roughly two stages: construction and usage.
The construction stage invokes $\f$ at a set of collocation nodes and produces
certain artifacts. The usage stage estimates the values of $\f$ at a set of
arbitrary points by manipulating the artifacts. In this subsection, we shall
have a look at the pseudocodes of the two stages. The purpose here is to give
the big picture; a lot of implementation details are purposely omitted. All the
details, however, can be found and studied at online as our implementation is
open-source \cite{sources}.

Let us first make a general note. We found it beneficial to the clarity and ease
of implementation to collapse the two sums in \eref{approximation} into one.
This requires storing a level index $\vi = (i_k)_{k = 1}^\nin$ and an order
index $\vj = (j_k)_{k = 1}^\nin$ for each interpolation element. Furthermore, it
is advantageous to encode each pair $(i_k, j_k)$ as a single unsigned integer,
which, in particular, eliminates excessive memory usage. In multiple dimensions,
this results in a single vector $\vl = (\iota_k)_{k = 1}^\nin$, which we simply
call an index. The encoding that we utilize is as follows:
\[
  \iota_k = i_k \lor (j_k \ll \n_\text{bits})
\]
where $\lor$ and $\ll$ are the bitwise \up{OR} and logical left shift,
respectively, and $\n_\text{bits}$ is the number of bits reserved for storing
Smolyak levels, which can be adjusted according to the maximum permitted
deepness of interpolation.

\begin{algorithm}
  \caption{\emph{Construct} an interpolant of a function.}
  \alab{construct}
  \begin{algorithmic}[1]
    \vspace{0.4em}

    \Require{target} \Comment{A function to approximate}
    \Ensure{surrogate} \Comment{Artifacts of approximation}

    \vspace{0.4em}

    \Let{level}{0}
    \Let{indices}{grid.ComputeIndices(level)}
    \Let{surrogate}{NewSurrogate()}

    \Loop
      \Let{nodes}{grid.ComputeNodes(indices)}
      \Let{values}{Invoke(target, nodes)}
      \Let{approximations}{Evaluate(surrogate, nodes)}
      \Let{surpluses}{$\text{values} - \text{approximations}$}
      \State Append(surrogate.Indices, indices)
      \State Append(surrogate.Surpluses, surpluses)
      \If{IsEnough(surrogate)}
        \State \textbf{break}
      \EndIf
      \For{i \textbf{in range of} indices}
        \If{IsAccurate(surpluses[i])}
          \State Exclude(indices, i)
        \EndIf
      \EndFor
      \If{IsEmpty(indices)}
        \State \textbf{break}
      \EndIf
      \Let{indices}{grid.ComputeNeighbors(indices)}
      \Let{level}{$\text{level} + 1$}
    \EndLoop

    \State \textbf{return} surrogate

    \vspace{0.4em}
  \end{algorithmic}
\end{algorithm}

The pseudocode of the construction stage is given in \aref{construct} called
\token{Construct}. The \token{target} input is a function $\f$ to be
approximated. The \token{surrogate} output is a structure containing the
artifacts of interpolation, which are a set of tuples $\{ (\vl_k,
\surplus(\vx_{\vl_k}) \}_k$ giving a comprehensive description of an
interpolant. The routine works as follows.

\begin{compactlist}

\point{Line~2:} Each iteration is an interpolation step in \eref{approximation},
and it has a state captured by a structure denoted by \token{s}. The
\token{strategy} object represents an adaptivity strategy utilized and works as
described in \sref{adaptivity}. The \token{First} method of \token{strategy}
returns the initial state of the first step so that the \token{indices} field of
\token{s} is initialized with the indices of that step. The body of the loop
populates the rest of the fields of \token{s} so that \token{strategy.Next} can
adequately produce the initial state of the next iteration. The process
terminates when a stopping condition is satisfied, in which case \token{Next}
returns a null state.

\point{Line~3:} The \token{grid} object represents the interpolation grid
utilized (see \sref{grid}), and its \token{Compute} method converts the step's
indices into the coordinates of the corresponding collocation nodes, that is,
$\{ \vl_k \}_k$ into $\{ \vx_{\vl_k} \}_k$.

\point{Line~4:} \token{Invoke} exercises \token{target} at the collocation
nodes. This function a prominent candidate for parallelization since each node
can be evaluated independently from the rest.

\point{Line~5:} \token{Evaluate} exercises the interpolant constructed so far at
the collocation nodes, approximating the values obtained on line~4. This
function will be discussed separately.

\point{Line~6:} \token{Subtract} computes the difference between the true and
approximated values of \token{target}, which yields the step's hierarchical
surpluses $\{ \surplus(\vx_{\vl_k}) \}_k$, similar to \eref{surplus}.

\point{Line~7:} \token{strategy.Score} calculates the scores of the new
collocation nodes based on their surpluses; see \eref{score}.

\point{Line~8:} \token{Append} ameliorates the interpolant by extending it with
the indices and surpluses of the current iteration.

\end{compactlist}

\begin{algorithm}
  \caption{\emph{Evaluate} an interpolant at a set of points.}
  \alab{evaluate}
  \begin{algorithmic}[1]
    \vspace{0.4em}

    \Require{surrogate} \Comment{Artifacts of approximation}
    \Require{points} \Comment{Points of interest}
    \Ensure{values} \Comment{Approximated values}

    \vspace{0.4em}

    \Let{indices}{surrogate.Indices}
    \Let{surpluses}{surrogate.Surpluses}
    \Let{values}{NewArray()}

    \For{i \textbf{in range of} points}
      \Let{value}{0}
      \For{j \textbf{in range of} indices}
        \Let{weight}{basis.Compute(indices[j], points[i])}
        \Let{value}{$\text{value} + \text{weight} \times \text{surpluses[j]}$}
      \EndFor
      \State Append(values, value)
    \EndFor

    \State \textbf{return} values

    \vspace{0.4em}
  \end{algorithmic}
\end{algorithm}

Let us now turn to the usage stage of an interpolant. The pseudocode is given in
\aref{evaluate} called \token{Evaluate}. This algorithm is also involved in
\aref{construct}; see line~5. Let us make a couple of observations regarding
\token{Evaluate}.

\begin{compactlist}

\point{Line~4:} The inner loop is an unfolded version of \eref{approximation}
(there is no separation between individual interpolation steps taken).

\point{Line~5:} The \token{basis} object represents the interpolation basis
utilized (see \sref{basis}), and its \token{Compute} method evaluates a single
(multidimensional) basis function at a single point.

\end{compactlist}

It is worth noting that the \token{basis}, \token{grid}, and \token{strategy}
objects conform to certain interfaces and can be easily swapped out. This makes
the two algorithms very general, reusable with different configurations. In
particular, the adaptivity strategy can be fine-tuned for each particular
problem.
