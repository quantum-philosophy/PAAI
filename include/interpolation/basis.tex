The basis functions that go hand in hand with the open Newton--Cotes rule are
the following piecewise linear functions. For $i = 0$ and $j = 0$, we have that
$\e_{00}(\x) = 1$. For $i > 0$ and $j = 0$ (close to the left endpoint),
\[
  \e_{i0}(\x) = \begin{cases}
    2 - \left( \n_i + 1 \right) \x, & \text{if } \x < \frac{2}{\n_i + 1}, \\
    0, & \text{otherwise}.
  \end{cases}
\]
For $i > 0$ and $j = \n_i - 1$ (close to the right endpoint),
\[
  \e_{i(\n_i - 1)}(\x) = \begin{cases}
    \left( \n_i + 1 \right) \x - \n_i + 1, & \text{if } \x > \frac{\n_i - 1}{\n_i + 1}, \\
    0, & \text{otherwise}.
  \end{cases}
\]
In other cases,
\[
  \e_{ij}(\x) = \begin{cases}
    1 - \left( \n_i + 1 \right)|\x - \x_{ij}|, & \text{if } |\x - \x_{ij}| < \frac{1}{\n_i + 1}, \\
    0, & \text{otherwise}.
  \end{cases}
\]
The basis functions corresponding to the first three levels of one-dimensional
interpolation are depicted in \fref{basis}. Note that $\e_{11}$, $\e_{21}$,
$\e_{23}$, and $\e_{25}$ are not depicted as they are not involved in the
hierarchical construction. In multiple dimensions, the basis functions are
formed as shown in \eref{basis-functions}.

Lastly, let us mention the volumes (integrals over the whole domain) of the
basis functions denoted by $\w_{ij}$; they will be needed in continuation.
Namely, $\w_{00} = 1$ and, for $i > 0$,
\begin{equation} \elab{volume}
  \w_{ij} = \int_0^1 \e_{ij}(\x) \, \d\x = \begin{cases}
    \frac{2}{\n_i + 1}, & \text{if } j \in \{ 0, \n_i - 1 \}, \\
    \frac{1}{\n_i + 1}, & \text{otherwise}.
  \end{cases}
\end{equation}
In multiple dimensions, the volumes are products of the one-dimensional
components, analogous to \eref{basis-function}.

\begin{remark}
Instead of piecewise linear functions, one can also utilize locally supported
polynomials of higher orders \cite{jakeman2012}. However, we did not observe
much improvement and, therefore, do not discuss this alternative in the paper.
\end{remark}

Imagine now a function that is nearly flat on the first half of $[0, 1]$ and
rather irregular on the other. Under these circumstances, it is natural to
expect that, in order to attain the same accuracy, the first half would require
much fewer collocation nodes than the other one; recall \fref{motivation}.
However, if we followed the construction procedure described so far, we would
not be able to benefit from this peculiar behavior: we would treat both sides
equally and would add all the nodes of each level. The solution to the above
problem is to make the interpolation algorithm adaptive, which we shall discuss
next.
