In one dimension ($\nin = 1$), $\f$ is approximated by virtue of the following
interpolation formula:
\begin{equation} \elab{tensor-1d}
  \tensor{i}(\f) := \sum_{j \in \index_i} \f(\x_{ij}) \, \e_{ij}.
\end{equation}
The superscript $i \in \natural$ signifies the level of interpolation; $\X_i =
\{ \x_{ij} \}_{j \in \index_i} \subset [0, 1]$ are the collocation nodes; $\E_i
= \{ \e_{ij} \}_{j \in \index_i} \subset \continuous([0, 1])$ are the basis
functions; and $\index_i = \{ j - 1 \}_{j = 1}^{n_i}$ is an index set
enumerating (starting from zero) the nodes and functions that belong to level
$i$. We shall refer to the subscript $j \in \index_i$ as the order of a node or
function. The choice of the collocation nodes and basis functions is an
important concern, and it will be discussed thoroughly later on.

In multiple dimensions ($\nin > 1$), $\f$ is approximated by the tensor product
of $\nin$ one-dimensional interpolants:
\begin{equation} \elab{tensor}
  \tensor{\vi}(\f) := \left( \bigotimes_{k = 1}^\nin \tensor{i_k} \right)(\f) = \sum_{\vj \in \index_\vi} \f(\vx_{\vi\vj}) \, \e_{\vi\vj}
\end{equation}
where $\vi = (i_k)_{k = 1}^\nin \in \natural^\nin$ and $\vj = (j_k)_{k = 1}^\nin
\in \natural^\nin$ are multi-indices specifying levels and orders, respectively,
for each of the dimensions, and $\index_\vi := \index_{i_1} \times \cdots \times
\index_{i_\nin}$ is a multi-index set obtained by computing the tensor product
of one-dimensional index sets. In the above formula,
\begin{align}
  \X_\vi &= \X_{i_1} \times \cdots \times \X_{i_\nin} \elab{collocation-nodes} \\
         &= \left\{ \vx_{\vi\vj} = (\x_{i_k j_k})_{k = 1}^\nin \right\}_{\vj \in \index_\vi} \subset [0, 1]^\nin \nonumber
\end{align}
and
\begin{equation} \elab{basis-functions}
  \E_\vi = \bigotimes_{k = 1}^\nin \E_{i_k}
         = \left\{ \e_{\vi\vj} = \bigotimes_{k = 1}^\nin \e_{i_k j_k} \right\}_{\vj \in \index_\vi} \subset \continuous([0, 1]^\nin)
\end{equation}
are the collocation nodes and basis functions, respectively, corresponding to
multi-index $\vi$. In \eref{basis-functions}, for any $\vx \in [0, 1]^\nin$,
\begin{equation} \elab{basis-function}
  \e_{\vi\vj}(\vx) = \left( \bigotimes_{k = 1}^\nin \e_{i_k j_k} \right)(\vx) := \prod_{k = 1}^\nin \e_{i_k j_k}(\x_k).
\end{equation}
Lastly, the cardinality of $\index_\vi$ is as follows:
\begin{equation} \elab{tensor-cardinality}
  \n_\vi = \prod_{k = 1}^\nin \n_{i_k}.
\end{equation}
Equation \eref{tensor-cardinality} elucidates the prohibited expense of the
tensor-product construction shown in \eref{tensor} for multidimensional
problems: the number of nodes grows exponentially as $\nin$ increases. However,
\eref{tensor} serves well as a building block for more efficient algorithms,
which we discuss next.

\begin{remark}
Each dimension can have its own rule defining the distribution of collocation
node with respect to each level. Similarly, the basis functions of one dimension
can differ from the basis functions of another. For simplicity and clarity of
presentation, this aspect is not covered in this paper.
\end{remark}
