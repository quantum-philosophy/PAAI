One of the central algorithms in the field of high-dimensional integration and
interpolation is the Smolyak algorithm \cite{smolyak1963}. The technique was
developed in the 1960s by Sergey Smolyak, and its impact is comparable to the
one of Monte Carlo sampling. Intuitively speaking, the algorithm takes a number
of small tensor-product structures and composes them in such a way that the
resulting grid has a drastically reduced number of nodes while preserving the
approximating power of the full tensor-product construction for the classes of
functions that one is typically interested in integrating or interpolating
\cite{klimke2006}.

The Smolyak interpolant for $\f$ is as follows:
\begin{equation} \elab{smolyak}
  \smolyak{l}(\f) := \sum_{l - \nin + 1 \leq |\vi| \leq l} (-1)^{l - |\vi|} \, {\nin - 1 \choose l - |\vi|} \, \tensor{\vi}(\f)
\end{equation}
where $l \in \natural$ is the level of Smolyak's interpolation, and $|\vi| :=
i_1 + \dots + i_\nin$. We see that the algorithm is indeed just a peculiar
composition of cherry-picked tensor products. However, the formula has an
implication of paramount importance: the quantity of interest needs to be
evaluated only at the nodes of the sparse grid underpinning \eref{smolyak}:
\begin{equation} \elab{smolyak-grid}
  \Y^l = \bigcup_{l - \nin + 1 \leq |\vi| \leq l} \X^\vi.
\end{equation}
The cardinality of the above set does not have a general closed-form formula;
however, it can be several orders of magnitude smaller than the one of the full
tensor product given in \eref{tensor-cardinality}, which depends on the
dimensionality of the problem at hand and the one-dimensional rules utilized.

A better intuition about the properties of Smolyak's construction can be
obtained by rewriting \eref{smolyak} in an incremental form. To this end, let
$\Delta\tensor{-1}(\f) := 0$,
\begin{align}
  & \Delta\tensor{i}(\f) := (\tensor{i} - \tensor{i - 1})(\f), \text{ and} \elab{tensor-delta-1d} \\
  & \Delta\tensor{\vi}(\f) := (\Delta\tensor{i_1} \otimes \cdots \otimes
  \Delta\tensor{i_\nin})(\f). \nonumber
\end{align}
Then, \eref{smolyak} is identical to
\begin{equation} \elab{smolyak-incremental}
  \smolyak{l}(\f) = \sum_{|\vi| \leq l} \Delta\tensor{\vi}(\f) = \smolyak{l - 1}(\f) + \sum_{|\vi| = l} \Delta\tensor{\vi}(\f)
\end{equation}
where $\smolyak{-1}(\f) := 0$. It can be seen that a Smolyak interpolant can be
refined efficiently: the work done to attain one accuracy level can be entirely
recycled to go one level up.

The sparsity and incremental refinement of Smolyak's approach, which are shown
in \eref{smolyak-grid} and \eref{smolyak-incremental}, respectively, are
remarkable properties \perse, but they can be taken even further. To this end,
let $\Delta\X^{-1} := \emptyset$,
\begin{align*}
  & \Delta\X^i := \X^i \setminus \X^{i - 1}, \text{ and} \\
  & \Delta\X^\vi := \Delta\X^{i_1} \times \cdots \times \Delta\X^{i_\nin}.
\end{align*}
Then, \eref{smolyak-grid} can be rewritten as
\begin{equation} \elab{smolyak-grid-incremental}
  \Y^l = \bigcup_{|\vi| \leq l} \Delta\X^\vi = \Y^{l - 1} \cup \bigcup_{|\vi| = l} \Delta\X^\vi,
\end{equation}
which is analogous to \eref{smolyak-incremental}. It can be seen now that it is
beneficial for refinement to have $\X^{i - 1}$ be entirely included in $\X^i$
since, in that case, the cardinality of $\Y^l \setminus \Y^{l - 1} =
\bigcup_{|\vi| = l} \Delta\X^\vi$ derived from \eref{smolyak-grid-incremental}
decreases. In words, the values of $\f$ obtained on lower levels can be reused
to attain higher levels if the grid grows without abandoning its previous
structure. With this in mind, the rule used for generating successive sets of
points $\{ \X^i \}_{i \in \natural}$ should be chosen to be nested, that is, in
such a way that $\X^i$ contains all nodes of $\X^{i - 1}$. By doing so, the
incremental refinement really starts to shine.

Lastly, we need to impose one additional property in order to rewrite
\eref{smolyak-incremental} in a hierarchical form. Namely, we require the
interpolants of higher levels to represent exactly the interpolants of lower
levels. In one dimension, it means that
\begin{equation} \elab{tensor-exactness}
  \tensor{i - 1}(\f) = \tensor{i}(\tensor{i - 1}(\f)).
\end{equation}
The condition in \eref{tensor-exactness} can be satisfied by an appropriate
choice of collocation nodes and basis functions, which will be discussed later.
If \eref{tensor-exactness} holds, using \eref{tensor-1d} and
\eref{tensor-delta-1d},
\[
  \Delta\tensor{i}(\f) = \sum_{j \in \Delta\index(i)} \left( \f(\x^i_j) - \tensor{i - 1}(\f)(\x^i_j) \right) \, \e^i_j
\]
where $\Delta\index(i) := \{ j \in \index(i): \x^i_j \in \Delta\X^i \}$. The
above sum is over $\Delta\X^i$ due to the fact that the difference in the
parentheses is zero whenever $\x^i_j \in \X^{i - 1}$ since $\X^{i - 1} \subset
\X^i$. In multiple dimensions, using the Smolyak formula, we have
\begin{equation} \elab{tensor-delta}
  \Delta\tensor{\vi}(\f) = \sum_{\vj \in \Delta\index(\vi)} \left( \f(\vx^\vi_\vj) - \smolyak{|\vi| - 1}(\f)(\vx^\vi_\vj) \right) \, \e^\vi_\vj
\end{equation}
where $\Delta\index(\vi) := \{ \vj \in \index(\vi): \vx^\vi_\vj \in \Delta\X^\vi
\}$. The delta
\begin{equation} \elab{surplus}
  \surplus(\vx^\vi_\vj) := \f(\vx^\vi_\vj) - \smolyak{|\vi| - 1}(\f)(\vx^\vi_\vj)
\end{equation}
is called a hierarchical surplus. When increasing the interpolation level, this
surplus is nothing but the difference between the actual value of $\f$ at a new
node and the approximation of this value computed by the interpolant constructed
so far.

The final formula for nonadaptive hierarchical interpolation is obtained by
substituting \eref{tensor-delta} into \eref{smolyak-incremental}:
\begin{align} 
  \smolyak{l}(\f) &= \sum_{|\vi| \leq l} \, \sum_{\vj \in \Delta\index(\vi)} \surplus(\vx^\vi_\vj) \, \e^\vi_\vj \elab{smolyak-hierarchical} \\
                  &= \smolyak{l-1}(\f) + \sum_{|\vi| = l} \, \sum_{\vj \in \Delta\index(\vi)} \surplus(\vx^\vi_\vj) \, \e^\vi_\vj \nonumber
\end{align}
where $\surplus(\vx^\vi_\vj)$ is computed according to \eref{surplus}.
