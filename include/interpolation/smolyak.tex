One of the central algorithms in the field of multidimensional integration and
interpolation is the Smolyak algorithm \cite{smolyak1963}. Intuitively speaking,
the algorithm takes a number of small tensor-product structures and composes
them in such a way that the resulting grid has a drastically reduced number of
nodes while preserving the approximating power of the full tensor-product
construction for the classes of functions that one is typically interested in
integrating or interpolating \cite{klimke2006}.

The Smolyak interpolant for $\f$ is as follows:
\begin{equation} \elab{smolyak-original}
  \smolyak{\l}(\f) = \sum_{\l - \nin + 1 \leq |\vi| \leq \l} (-1)^{\l - |\vi|} \, {\nin - 1 \choose \l - |\vi|} \, \tensor{\vi}(\f)
\end{equation}
where $\l \geq 0$ is the index of the interpolation step, which we shall refer
to as the Smolyak level, and $|\vi| = i_1 + \dots + i_\nin$. It can be seen that
the algorithm is indeed just a peculiar composition of cherry-picked tensor
products. However, the formula has an implication of paramount importance.
The function $\f$ needs to be evaluated only at the nodes of the sparse grid
underpinning \eref{smolyak-original}:
\begin{equation} \elab{smolyak-grid}
  \Y_l = \bigcup_{\l - \nin + 1 \leq |\vi| \leq \l} \X_\vi.
\end{equation}
The cardinality of the above set does not have a general closed-form formula;
however, it can be several orders of magnitude smaller than the one of the full
tensor product given in \eref{tensor-cardinality}, which depends on the
dimensionality of the problem at hand and the one-dimensional rules utilized
(\sref{tensor}).

A better intuition about the properties of the Smolyak construction can be
obtained by rewriting \eref{smolyak-original} in an incremental form. To this
end, let $\Delta\tensor{-1}(\f) = 0$,
\begin{align}
  & \Delta\tensor{i}(\f) = (\tensor{i} - \tensor{i - 1})(\f), \text{ and} \elab{tensor-delta-1d} \\
  & \Delta\tensor{\vi}(\f) = \left( \bigotimes_{k = 1}^\nin \Delta\tensor{i_k} \right)(\f). \nonumber
\end{align}
Then, \eref{smolyak-original} is identical to
\begin{equation} \elab{smolyak-incremental}
  \smolyak{\l}(\f) = \sum_{\vi \in \lindex_\l} \Delta\tensor{\vi}(\f) = \smolyak{\l - 1}(\f) + \sum_{\vi \in \Delta\lindex_\l} \Delta\tensor{\vi}(\f)
\end{equation}
where $\smolyak{-1}(\f) = 0$, and we let $\lindex_\l = \{ \vi: |\vi| \leq \l
\}$ and $\Delta\lindex_\l = \{ \vi: |\vi| = \l \}$. It can be seen that a
Smolyak interpolant can be efficiently refined: the work done in order to attain
one Smolyak level $\l$ is entirely recycled to go to the next.

The sparsity and incremental refinement of the Smolyak approach, which are shown
in \eref{smolyak-grid} and \eref{smolyak-incremental}, respectively, are
remarkable properties \perse, but they can be taken even further. To this end,
let $\Delta\X_{-1} = \emptyset$,
\[
  \Delta\X_i = \X_i \setminus \X_{i - 1}, \text{ and } \Delta\X_\vi = \Delta\X_{i_1} \times \cdots \times \Delta\X_{i_\nin}.
\]
Then, \eref{smolyak-grid} can be rewritten as
\begin{equation} \elab{smolyak-grid-incremental}
  \Y_\l = \bigcup_{\vi \in \lindex_\l} \Delta\X_\vi = \Y_{\l - 1} \cup \bigcup_{\vi \in \Delta\lindex_\l} \Delta\X_\vi,
\end{equation}
which is analogous to \eref{smolyak-incremental}. It can be seen now that it is
beneficial to the refinement to have $\X_{i - 1}$ be entirely included in $\X_i$
since, in that case, the cardinality of $\Y_l \setminus \Y_{\l - 1} =
\bigcup_{\vi \in \Delta\lindex_\l} \Delta\X_\vi$ derived from
\eref{smolyak-grid-incremental} decreases. In words, the values of $\f$ obtained
on lower levels (lower $\l$) can be reused to attain higher levels (higher $\l$)
if the grid grows without abandoning its previous structure. With this in mind,
the rule used for generating successive sets of points $\{ \X_i \}_i$ should be
chosen to be nested, that is, in such a way that $\X_i$ contains all nodes of
$\X_{i - 1}$.

The final step in this subsection is to rewrite \eref{smolyak-incremental} in a
hierarchical form. To this end, we require the interpolants of higher levels to
represent exactly the interpolants of lower levels. In one dimension, it means
that
\begin{equation} \elab{tensor-exactness}
  \tensor{i - 1}(\f) = \tensor{i}(\tensor{i - 1}(\f)).
\end{equation}
The condition in \eref{tensor-exactness} can be satisfied by an appropriate
choice of collocation nodes and basis functions, which will be discussed later.
If \eref{tensor-exactness} holds, using \eref{tensor-1d} and
\eref{tensor-delta-1d},
\[
  \Delta\tensor{i}(\f) = \sum_{j \in \Delta\oindex_i} \left( \f(\x_{ij}) - \tensor{i - 1}(\f)(\x_{ij}) \right) \, \e_{ij}
\]
where $\Delta\oindex_i = \{ j \in \oindex_i: \x_{ij} \in \Delta\X_i \}$. The
above sum is over $\Delta\X_i$ due to the fact that the difference in the
parentheses is zero whenever $\x_{ij} \in \X_{i - 1}$ since $\X_{i - 1} \subset
\X_i$.

In multiple dimensions, using the Smolyak formula,
\begin{equation} \elab{tensor-delta}
  \Delta\tensor{\vi}(\f) = \sum_{\vj \in \Delta\oindex_\vi} \left( \f(\vx_{\vi\vj}) - \smolyak{|\vi| - 1}(\f)(\vx_{\vi\vj}) \right) \, \e_{\vi\vj}
\end{equation}
where $\Delta\oindex_\vi = \{ \vj \in \oindex_\vi: \vx_{\vi\vj} \in
\Delta\X_\vi \}$. The delta
\begin{equation} \elab{surplus}
  \surplus(\vx_{\vi\vj}) = \f(\vx_{\vi\vj}) - \smolyak{|\vi| - 1}(\f)(\vx_{\vi\vj})
\end{equation}
is called a hierarchical surplus. When increasing the interpolation level, this
surplus is nothing but the difference between the actual value of $\f$ at a new
node and the approximation of this value computed by the interpolant constructed
so far.

The final formula for nonadaptive hierarchical interpolation is obtained by
substituting \eref{tensor-delta} into \eref{smolyak-incremental}:
\begin{align}
  \smolyak{\l}(\f) &= \sum_{\vi \in \lindex_\l} \, \sum_{\vj \in \Delta\oindex_\vi} \surplus(\vx_{\vi\vj}) \, \e_{\vi\vj} \nonumber \\
                   &= \smolyak{\l-1}(\f) + \sum_{\vi \in \Delta\lindex_\l} \, \sum_{\vj \in \Delta\oindex_\vi} \surplus(\vx_{\vi\vj}) \, \e_{\vi\vj} \elab{smolyak-hierarchical}
\end{align}
where $\surplus(\vx_{\vi\vj})$ is computed according to \eref{surplus}.
